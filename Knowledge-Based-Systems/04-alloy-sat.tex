\section{Alloy for Declarative Process Mining}

The problems presented in the previous section can be cast into
various problems
typical of databases. This can be done because in all such problems, we deal
with a finite interpretation domain.
As described above, solving these
problems requires in practice either constructing a set of models
or building
a relation that satisfies certain requirements.
%%
In this work, we explore the use of Alloy~\cite{alloy} as a tool for
these tasks.

Alloy is a tool for the construction of models of FO theories over
finite interpretation domains. In its essence, the tool takes
as input a FO theory and an interpretation domain (or a bound
on the size of the interpretation domain) and returns,
if any, a model of the input theory.

Interestingly, Alloy builds the models by compiling the problem
into SAT, using propositions to represent FO atoms built from
predicate symbols and the finitely many objects in the interpretation
domain. We are not interested here in the compilation step,
but in: \myi how we can cast our problems
into that of finding a model for a FO theory; and \myii how effective
the (SAT-based) solution approach of Alloy is in solving the
problems of interest. In this section,
we discuss the former point, while, in the section
about experiments, we deal with the latter.

\subsection{Generation of Event Logs}
Generation of event logs is possibly the simplest
problem to solve with Alloy. We simply take $\Gamma$ and the
conjunction of the constraints to satisfy, say $\varphi$, and
ask Alloy to search for the models of $\Gamma\cup\set{\varphi}$,
for a given (finite) $\Delta$.
Observe that, if needed, we can add, on top of the input
theory, additional constraints, as long as these are
consistent with $\Gamma\cup\set{\varphi}$.
For instance, we may require that the set of
attribute names be restricted to a certain desired set:
$$\forall i,a,v.AttV(i,a,v)\rightarrow
	a=\cst v\lor
	a=\cst w\lor
	a=\cst y\lor
	a=\cst s,$$
analogously for attribute values:
$$\forall i,a,v.AttV(i,a,v)\rightarrow
	v=\cst a\lor
	v=\cst b\lor
	v=\cst c\lor
	v=\cst d,$$
or even that certain attributes may take only certain values:
$$\forall i,v.AttV(i,\cst v,v)\rightarrow
	v=\cst a\lor
	v=\cst d.$$
	
To obtain many traces, we ask Alloy, once returned a model,
to build one more, until enough traces have been
produced or no more solutions are available.
Notice that we can incrementally increase the size of the
interpretation domain by adding new constants to obtain
additional traces.


% % % % \subsection{Preliminaries on First-order Logic}
% % % % We start by briefly recalling
% % % % standard preliminary notions about first-order logic.
% % % % %%
% % % % A first-order (FO) \emph{vocabulary} $\sigma$ consists of sets of
% % % % \emph{constant
% % % % symbols} $c_1,\ldots,c_n$, \emph{predicate symbols} $P_1,\ldots,P_m$ and
% % % % \emph{function symbols} $f_1,\ldots,f_q$, where each predicate or function
% % % % symbol has a respective (finite) arity.
% % % % %%
% % % % An \emph{interpretation} $\I$ of a FO vocabulary $\sigma$ over a
% % % % (possibly infinite) \emph{universe} $U$ is a pair
% % % % $I=\tup{U,\cdot^I}$, where $\cdot^\I$ is the \emph{interpretation function}
% % % % of $\I$, i.e., a function
% % % % associating:
% % % % 	\begin{itemize}
% % % % 		\item each constant $c$ with an object $c^\I\in U$;
% % % % 		\item each predicate symbol $P$ of arity $a$ with a relation
% % % % $P^\I\subseteq U^a$;
% % % % 		\item each function symbol $f$ of arity $a$ with a mapping
% % % % $f^\I:U^a\rightarrow U$.
% % % % 	\end{itemize}
% % % % %%
% % % % For our purposes, we can assume that constants
% % % % are always interpreted as themselves, i.e., that for any interpretation $\I$,
% % % % it is the case that $c^\I=c$ (this is a common assumption in many situations,
% % % % e.g., in database theory~\cite{vianu-book}). Obviously, this implies that all
% % % % (and only) constant symbols are included in $U$.
% % % %
% % % % Given a FO vocabulary $\sigma$ and an infinite (countable) set $V$ of
% % % % variable symbols, formulas $\varphi$ of FOL over $\sigma$
% % % % respect the following syntax:
% % % % $$\varphi = P(\vec t)\mid \lnot \varphi\mid \varphi\land\varphi\mid \exists
% % % % x.\varphi, \mbox{ where:}$$
% % % % \begin{itemize}
% % % %  \item $P$ is a predicate symbol from $\sigma$;
% % % %  \item $\vec t$ is a tuple
% % % % 	$\tup{t_1,\ldots,t_a}$, with $a$ the arity of $P$,
% % % % 	and each $t_i$ a \emph{term} from $\sigma$, i.e., a variable or a constant
% % % % symbol from $\sigma$, or a \emph{function term}, i.e., an expression of the
% % % % form $f(\vec t')$, with $f$ a function symbol from $\sigma$ and $\vec t'$ a
% % % % tuple of terms
% % % % from $\sigma$, of size equal to the arity of $f$.
% % % % \end{itemize}
% % % % %%
% % % % As standard, we define the following abbreviations:
% % % % $\varphi_1\lor \varphi_2\doteq \lnot(\lnot\varphi_1\land\lnot\varphi_2)$,
% % % % $\forall x.\varphi\doteq\lnot\exists x.\lnot\varphi$,
% % % % $\varphi_1 \rightarrow \varphi_2\doteq \lnot\varphi_1 \lor \varphi_2$,
% % % % and $\varphi_1 \leftrightarrow \varphi_2\doteq (\varphi_1 \rightarrow
% % % % \varphi_2) \land (\varphi_2 \rightarrow \varphi_1)$.
% % % % %%
% % % % The variable $x$ in $\exists x.\varphi$ (resp.~$\forall x.\varphi$) is
% % % % an (existentially, resp.~universally) quantified variable.
% % % % Variables in a FO
% % % % formula do not have to be quantified, in which case they are called
% % % % \emph{free variables}.
% % % % Formulas containing only quantified variables are called \emph{sentences}.
% % % % %%
% % % %
% % % % FO formulas are interpreted over FO interpretations and
% % % % \emph{variable assignments}, i.e., total mappings $\nu:V\mapsto U$,
% % % % for $U$ the universe of the interpretation in question.
% % % % Formula interpretations require interpreting their terms first.
% % % % Given a FO interpretation $\I=\tup{U,\cdot^\I}$ of $\sigma$ and a term $t$
% % % % over $\sigma$, the
% % % % interpretation of $t$ under $\nu$ and $\I$ is the object
% % % % $t^\I_\nu\in U$ such that:
% % % % \begin{itemize}
% % % %  \item $t^\I_\nu = \nu(t)$, if $t$ is a variable symbol;
% % % %  \item $t^\I_\nu = c^\I(=c)$, if $t$ is a constant symbol;
% % % %  \item $t^\I_\nu = f^\I({t_1}^\I_\nu,\ldots,{t_a}^\I_\nu)$, for $a$
% % % % the arity of $f$, if $t=f(t_1,\ldots,t_a)$.
% % % % \end{itemize}
% % % % %%
% % % %
% % % % A FO interpretation
% % % % $\I$ is said to \emph{satisfy}
% % % % a FO formula $\varphi$ under assignment $\nu$,
% % % % written $\I, \nu\models\varphi$, iff either:
% % % % %%
% % % % \begin{itemize}
% % % % 	\item $\varphi=P(t_1,\ldots,t_a)$ and
% % % % 	$\tup{{t_1}^\I_\nu,\ldots,{t_a}^\I_\nu}\in P^\I$;
% % % % 	\item $\varphi=\lnot \phi$ and $\I,\nu\not\models\phi$;
% % % % 	\item $\varphi=\phi_1\land \phi_2$ and $\I,\nu\models \phi_i$, for $i=1,2$;
% % % % 	\item $\varphi=\exists x.\phi$ and there exists $u\in U$ such that
% % % % $\I,\nu'\models \phi$, with $\nu'(x)=u$ and $\nu'(v)=\nu(v)$, for $v\neq x$.
% % % % \end{itemize}
% % % % %%
% % % % % % Notice that, in order to check satisfaction of $\varphi$, all
% % % % %%terms occurring
% % % % % % in it must be replaced by their interpretation.
% % % % % %
% % % % Finally, an interpretation $\I$ \emph{satisfies}
% % % % a formula $\varphi$, written $\I\models \varphi$, iff $\I,\nu\models\varphi$
% % % % for all assignments $\nu$. This notion is used, in particular, when $\varphi$
% % % % is a sentence.
% % % %
% % % % A \emph{FO theory} $\Gamma$ is a possibly infinite set of FO sentences. Here,
% % % % we focus on finite theories.
% % % % An interpretation is said to
% % % % satisfy a theory $\Gamma$ if $\I\models\varphi$ for all $\varphi\in\Gamma$.
% % % % When
% % % % this is the case, $\I$ is said to be a \emph{model} of $\Gamma$.
% % % % %%
% % % % It is well known that checking existence of a model of a theory is
% % % % an undecidable problem, even if one restricts to \emph{finite} models only,
% % % % i.e., models whose universe $U$ is finite (see, e.g., \cite{FO-book}). The
% % % % problem is instead decidable if one considers only bounded models, i.e., with
% % % % $U$ of cardinality no larger than a fixed bound.In this case, which is the one
% % % % of interest in this work, the model can be effectively computed~\cite{}.
% % % %
% % % % One possible approach to construct (bounded) models of (finite)
% % % % theories is based on compilation into SAT. The idea is to compile, depending
% % % % also on the bound $b$, the theory into a boolean formula $\phi$ such that
% % % % $\phi$ is satisfied iff there exists a model of the theory with a universe
% % % % bounded in size by $b$.
% % % % This approach, the one adopted by
% % % % the software analysis tool Alloy~\cite{}, has the great advantage of benefiting
% % % % directly from the continuous improvements obtained by
% % % % SAT solvers, as well as offering to users, at essentially no additional
% % % % cost, the whole set of SAT solvers available to date.
% % % %
% % % % Here, we rely on Alloy for this reason, i.e., to check the
% % % % effectiveness of SAT technology as a way to address problems typical in Process
% % % % Mining. We stress that we are not interested in the compilation process itself,
% % % % but simply as a way to compile our problems into SAT. Thus, we will use Alloy as
% % % % a black-box tool, without entering the details of the compilation process. We
% % % % refer the interested reader to~\cite{JacksonSS00} and to the numerous papers on
% % % % the topic, accessible from the Alloy website[\cite{}, footnote?,
% % % % already cited?].
% % % %
% % % % \subsection{Modeling Log Traces as FO Theories}
% % % %
% % % % In this section, we explain how FO theories can be used to capture log
% % % % traces. For a given integer $b$, we represent a trace of size $b$ as an ordered
% % % % set of $b$ events.
% % % % Let $\{a_1,\ldots,a_p\}$ be the set of possible activities;
% % % % assume that activity $a_i$ has $q_i$ attributes
% % % % $att_{ij}$ ($j=1,\ldots,q_i$), each with possible values from
% % % % a domain $D_{ij}=\{d_{ij}^1,\ldots,d_{ij}^{r_{ij}}\}$.
% % % % %%
% % % % Then, the vocabulary $\sigma$ includes:
% % % % \begin{itemize}
% % % % 	\item $b$ constants $\mathtt{e}_1,\ldots,\mathtt{e}_b$, representing the
% % % % events in the trace;
% % % % 	\item $p$ constants $\mathtt{a}_1,\ldots,\mathtt{a}_p$, representing the
% % % % possible activities;
% % % % 	\item constants
% % % % $\mathtt{att}_{11},\ldots,\mathtt{att}_{1q_1}\ldots,\mathtt{att}_{p1},\ldots,
% % % % \mathtt{att}_ {pq_p}$, modeling the activity attributes;
% % % % 	\item constants
% % % % $\mathtt{d}_{11}^1,\ldots,\mathtt{d}_{11}^{r_{11}},\ldots,\mathtt{d}_{pq_p}^1,
% % % % \ldots,\mathtt{d}_{ pq_p}^{r_{ pq_p}}$
% % % % representing values in the attribute domains;
% % % % 	\item unary predicate symbols $E$(vents), $A$(ctivities), $Att$(tributes),
% % % % and $D_{11},\ldots,D_{1q_1}\ldots,D_{p1},\ldots{D_{p_{q_p}}}$, one per
% % % % attribute;	
% % % % \item binary predicate symbols
% % % % $hasAtt$(ribute), $N$(ext) and $Aft$(er);
% % % % \item a unary function symbol $act$;
% % % % \item a binary function symbol $att$;
% % % % \end{itemize}
% % % % %%
% % % % With a slight abuse of notation we use
% % % % the same names for activities, events, etc., and their corresponding predicate
% % % % symbols. However, we use a different $\mathtt{font}$ for constants,
% % % % to distinguish them from variables.
% % % %
% % % % Predicate symbol $E$ is intended to capture trace events, $A$ activities, $Att$
% % % % attributes, and $D_{ij}$ the set of possible values that can be assigned to
% % % % attribute $att_{ij}$;
% % % % the relation symbol $hasAtt$(attribute) models the relationship between
% % % % activities and attributes;
% % % % the relation symbols $N$ and
% % % % $Aft$ model the ordering of events;
% % % % the unary function symbol $act$ represents the association between events and
% % % % activities; finally, the binary function symbol $att$ is used to model the
% % % % value assignment to attributes.
% % % %
% % % % We can now formally define the theory $\Gamma$. This is
% % % % obtained as the union of the sentences described below.
% % % % %%
% % % % First, to guarantee that $E$ contains all and only the constants
% % % % $\mathtt{e}_i$, we include in $\Gamma$ the following sentence:
% % % % %%
% % % % \begin{equation}\label{eq1}
% % % % \big(\bigwedge_{i=1}^b E(\mathtt{e}_i)\big)
% % % % \land
% % % % \big(\forall
% % % % x.E(x)\rightarrow \bigvee_{i=1}^b x=\mathtt{e}_i\big)
% % % % \end{equation}
% % % % %%
% % % % We do the same with $A$, $Att$ and each $D_{ij}$, thus imposing that $A$
% % % % contains
% % % % exactly the activities $\mathtt{a}_1,\ldots,\mathtt{a}_p$,
% % % % that $Att$ contains exactly the attributes
% % % % $\mathtt{att}_{11},\ldots,\mathtt{att}_{1q_1},\ldots,\mathtt{att}_{p1},\ldots,
% % % % \mathtt{att}_{pq_p}$ , and that each $D_{ij}$ contains exactly
% % % % the values $\mathtt{d}_{ij}^1,\ldots,\mathtt{d}_{ij}^{r_{ij}}$.
% % % %
% % % % We impose also that the interpretation domain contains exactly the constants
% % % % mentioned so far:
% % % % \begin{equation}\label{eq2}
% % % % \forall x. E(x)\lor A(x)\lor Att(x)\lor D_{11}(x)\lor\cdots\lor
% % % % D_{pq_p}(x)
% % % % \end{equation}
% % % %
% % % %
% % % % To express that each activity $\mathtt{a}_i$ has (exactly) attributes
% % % % $\mathtt{att}_{i1},\ldots,\mathtt{att}_{iq_i}$, we write the following:
% % % % \begin{equation}\label{eq3}
% % % % 	\bigwedge_{i=1}^p
% % % % 		\forall y.hasAtt(\mathtt{a}_i,y)\leftrightarrow
% % % % 		\bigvee_{j=1}^{q_i} (y=\mathtt{att}_{ij})	
% % % % \end{equation}
% % % %
% % % % Then, through $N$, we define an arbitrary order on the events. This can be
% % % % arbitrary because events are essentially placeholders whose identity does not
% % % % carry any relevant information; what matters, instead, is set of activities, in
% % % % the trace, together with their respective attribute
% % % % assignments, and their mutual order.
% % % % With the following sentence, we constrain $N$ to represent an ``immediate
% % % % successor'' relation:
% % % % \begin{equation}\label{eq4}
% % % % \forall e,e'.N(e,e')\leftrightarrow
% % % % 	((e=\mathtt{e}_1\land
% % % % e'=\mathtt{e}_2)\lor\cdots\lor(e=\mathtt{e}_{b-1}\land e'=\mathtt{e}_{b}))
% % % % \end{equation}
% % % % That is, we set the successor of event $\mathtt{e}_i$ as $\mathtt{e}_{i+1}$.
% % % %
% % % % Relation $Aft$ models
% % % % that an event $e'$ occurs in the future wrt an event $e$:
% % % % \begin{equation}\label{eq5}
% % % % \forall
% % % % e,e'.Aft(e,e')\leftrightarrow\bigvee_{i=1}^{b-1}(e=\mathtt{e}_i\land\bigvee_{
% % % % j=i+1 } ^ { b-1 } e'=\mathtt{e}_j)
% % % % \end{equation}
% % % % This sentence says that for any two events $e_i$ and $e_j$, it hods that
% % % % $Aft(\mathtt{e}_i,\mathtt{e}_j$) if and only if $j>i$. Notice this is
% % % % consistent with the
% % % % ordering defined by $N$.
% % % %
% % % % Then, we require that every event $e$ has one activity associated through $act$:
% % % % \begin{equation}\label{eq6}
% % % %  \forall e.E(e)\rightarrow A(act(e))
% % % % \end{equation}
% % % %
% % % % Finally, we impose that every activity associated with an event has all of its
% % % % attributes valuated with some value coming from the appropriate
% % % % domain:
% % % % \begin{equation}\label{eq7}
% % % % \begin{array}{l}
% % % % 	\forall e,a,x,d.(E(e)\land a=act(e)\land
% % % % hasAtt(a,x)\land d=att(e,x))\rightarrow\\
% % % % \phantom{xxxxxxxxxxxxxxxxxxxxxxxxxx}\bigvee_{i=1}^p a=\mathtt{a}_i\land
% % % % (\bigvee_{j=1}^{q_p} (x=\mathtt{att}_{ij} \land D_{ij}(d)))
% % % % \end{array}
% % % % \end{equation}
% % % %
% % % % It can be easily shown that any trace of size $b$ (over the above mentioned
% % % % activities, with corresponding attributes and respective domains)
% % % % can be captured by a FO interpretation of the theory $\Gamma$ defined by the
% % % % above sentences. Viceversa, any model of $\Gamma$ captures a trace as above.
% % % %
% % % % \begin{example}\label{ex1}
% % % % Assume the following possible activities:
% % % %  \begin{itemize}
% % % %   \item $A_1$, with attribute $v$, taking values
% % % % from $D_{11}=\{a,b\}$;
% % % %   \item $A_2$, with attributes $w$ and
% % % % $y$, taking values
% % % % from, respectively, $D_{21}=\{b,d\}$
% % % % and
% % % % $D_{22}=\{b,d\}$;
% % % %   \item $A_3$, with attribute $s$, taking values from
% % % % $D_{31}=\{c,d,e\}$.
% % % %  \end{itemize}
% % % % %%
% % % % For $b=4$ the length of a trace, the vocabulary $\sigma$
% % % % includes (remember we use the same names for events, activities, etc. and for
% % % % their counterparts in $\sigma$):
% % % % \begin{itemize}
% % % % 	\item predicate symbols:
% % % % 		$E$, $A$, $Att$, $D_{11}$, $D_{21}$, $D_{22}$, $D_{31}$;
% % % % 	\item binary relation symbols
% % % % 		$hasAtt, N, Aft$;
% % % % 	\item constants:
% % % % 		$\mathtt{e}_1,\mathtt{e}_2,\mathtt{e}_3,\mathtt{e}_4$,
% % % % 		$\mathtt{A}_1,\mathtt{A}_2,\mathtt{A}_3$,
% % % % 		$\mathtt{v},\mathtt{w},\mathtt{y},\mathtt{s}$,
% % % % 		$\mathtt{a},\mathtt{b},\mathtt{c},\mathtt{d},\mathtt{e}$;
% % % % 	\item function symbols:
% % % % 		$act$, $att$.
% % % % \end{itemize}
% % % % %%
% % % % $\Gamma$ is obtained by instantiating sentences~\ref{eq1}
% % % % to~\ref{eq7} according to the assignments above.
% % % % For instance, for $D_{21}$, sentence~(\ref{eq1}) is as follows:
% % % % %%
% % % % $$D_{21}(\mathtt{b})\land D_{21}(\mathtt{d})\land \big(\forall
% % % % x.D_{21}(x)\rightarrow (x=\mathtt{b}\lor
% % % % x=\mathtt{d})\big)$$
% % % % %%
% % % % Sentence~(\ref{eq2}) remains the same, while sentence~(\ref{eq3}) is
% % % % instantiated
% % % % as follows:
% % % % $$
% % % % 	\begin{array}{c}
% % % % 	(\forall y.hasAtt(\mathtt{A}_1,y)\leftrightarrow (y=\mathtt{v}))\land\\
% % % % 	(\forall y.hasAtt(\mathtt{A}_2,y)\leftrightarrow
% % % % (y=\mathtt{w})\lor(y=\mathtt{y}))\land\\	
% % % % 	(\forall y.hasAtt(\mathtt{A}_3,y)\leftrightarrow (y=\mathtt{s}))
% % % % 	\end{array}
% % % % $$
% % % % %%
% % % % As to sentence~(\ref{eq4}), we have:
% % % % $$
% % % % \forall e,e'.N(e,e')\leftrightarrow
% % % % 	(
% % % % 		(e=\mathtt{e}_1\land
% % % % 		e'=\mathtt{e}_2)\lor
% % % % 		(e=\mathtt{e}_2\land
% % % % 		e'=\mathtt{e}_3)\lor
% % % % 		(e=\mathtt{e}_3\land
% % % % 		e'=\mathtt{e}_4)
% % % % )
% % % % $$
% % % % %%
% % % % Sentence~(\ref{eq5}) becomes:
% % % % $$
% % % % \begin{array}{ll}
% % % % \forall e,e'.Aft(e,e')\leftrightarrow(&
% % % % 	(e=\mathtt{e}_1\land (e'=\mathtt{e}_2\lor e'=\mathtt{e}_3 \lor
% % % % e'=\mathtt{e}_4))\lor  \\ &
% % % % 	(e=\mathtt{e}_2\land (e'=\mathtt{e}_3 \lor
% % % % e'=\mathtt{e}_4))\lor  \\ &
% % % % 	(e=\mathtt{e}_3\land e'=\mathtt{e}_4))
% % % % \end{array}
% % % % $$
% % % % %%
% % % % Sentence~(\ref{eq6}) remains unchanged, while sentence~(\ref{eq7}) becomes as
% % % % follows:
% % % % $$
% % % % \begin{array}{l}
% % % % \forall e,a,x,d.(E(e)\land a=act(e)\land
% % % % hasAtt(a,x)\land d=att(e,x))\rightarrow (\\
% % % % \phantom{xxxxxxxxxxxxxxxxxxxxxxxxxxxxxxxxxx}
% % % % (a=\mathtt{A}_1\land (x=\mathtt{v}\land D_{11}(d)))\lor\\
% % % % \phantom{xxxxxxxxxxxxxxxxxx}
% % % % (a=\mathtt{A}_2\land ((x=\mathtt{w}\land
% % % % D_{21}(d))\lor(x=\mathtt{y}\land D_{22}(d))))\lor\\
% % % % \phantom{xxxxxxxxxxxxxxxxxxxxxxxxxxxxxxxxxx}
% % % % (a=\mathtt{A}_3\land (x=\mathtt{s}\land D_{31}(d))))
% % % % \end{array}
% % % % $$
% % % %
% % % % Now, consider the trace:
% % % % $$A_1[v=a]A_2[w=b,y=d]A_3[s=c]A_3[s=d]$$
% % % % %%
% % % % The model of $\Gamma$ corresponding to this trace is the interpretation $\I$
% % % % such that:
% % % % \begin{itemize}
% % % %  \item $E^\I=\{\mathtt{e}_1,\mathtt{e}_2,\mathtt{e}_3,\mathtt{e}_4\}$;
% % % %  \item $A^\I=\{\mathtt{A}_1,\mathtt{A}_2,\mathtt{A}_3\}$;
% % % %  \item $Att^\I=\{\mathtt{v},\mathtt{w},\mathtt{y},\mathtt{s}\}$;
% % % %  \item $D_{11}^\I=\{\mathtt{a},\mathtt{b}\}$,
% % % % 	$D_{21}^\I=D_{22}^\I=\{\mathtt{b},\mathtt{d}\}$,
% % % % 	$D_{31}^\I=\{\mathtt{c},\mathtt{d},\mathtt{e}\}$;
% % % %  \item
% % % % $hasAtt^\I=\{\tup{\mathtt{A}_1,\mathtt{v}},
% % % % \tup{\mathtt{A}_2,\mathtt{w}},\tup{\mathtt{A}_2,\mathtt{y}},\tup{\mathtt{A}_3,
% % % % \mathtt{s}}\}$
% % % % \item $N^\I=\{\tup{\mathtt{e}_1,\mathtt{e}_2},
% % % % \tup{\mathtt{e}_2,\mathtt{e}_3},
% % % % \tup{\mathtt{e}_3,\mathtt{e}_4}\}$;
% % % % \item $Aft^I=\{
% % % % \tup{\mathtt{e}_1,\mathtt{e}_2},
% % % % \tup{\mathtt{e}_1,\mathtt{e}_3},
% % % % \tup{\mathtt{e}_1,\mathtt{e}_4},
% % % % \tup{\mathtt{e}_2,\mathtt{e}_3},
% % % % \tup{\mathtt{e}_2,\mathtt{e}_4},
% % % % \tup{\mathtt{e}_3,\mathtt{e}_4}
% % % % \}$
% % % % \item $act^I=\{
% % % % \mathtt{e}_1\mapsto\mathtt{A_1},
% % % % \mathtt{e}_2\mapsto\mathtt{A_2},
% % % % \mathtt{e}_3\mapsto\mathtt{A_3},
% % % % \mathtt{e}_4\mapsto\mathtt{A_3},
% % % % \cdots
% % % % \}$, where $\cdots$ denotes that the other assignments can be any (as not
% % % % relevant);
% % % % \item $att^I=\{
% % % % \tup{\mathtt{e}_1,\mathtt{v}}\mapsto\mathtt{a},
% % % % \tup{\mathtt{e}_2,\mathtt{w}}\mapsto\mathtt{b},
% % % % \tup{\mathtt{e}_2,\mathtt{y}}\mapsto\mathtt{d},
% % % % \tup{\mathtt{e}_3,\mathtt{s}}\mapsto\mathtt{c},
% % % % \tup{\mathtt{e}_4,\mathtt{s}}\mapsto\mathtt{d},\cdots
% % % % \}$ (again, $\cdots$ denote irrelevant assignments);
% % % % \end{itemize}
% % % % %%
% % % % For any other trace of length $4$, a similar interpretation obviously exists.
% % % % \end{example}
% % % %
% % % % It is immediate to generalize Example~\ref{ex1} to the case of any $b$ and
% % % % trace of length $b$, thus obtaining that any trace of length be can be captured
% % % % by an interpretation of a theory $\Gamma$ as above.
% % % %
% % % % Also the converse can be shown, i.e., that, for given $b$, any interpretation
% % % % of a theory $\Gamma$ as above represents a trace.
% % % % Indeed, for fixed $b$, all the models of $\Gamma$ match except
% % % % for the interpretations $act^\I$ and $att^\I$, i.e., for the activities
% % % % associated with each event and
% % % % the corresponding assignment to each of their attributes.
% % % % This is a direct consequence of sentences~\ref{eq1} to~\ref{eq5}. However,
% % % % sentences~\ref{eq6} and~\ref{eq7} respectively prescribe that: exactly one
% % % % activity is assigned to each event and all the attributes of an activity
% % % % assigned to some event are assigned a value from their respective domain.
% % % % Based on this, it can be easily seen that any interpretation of $\Gamma$ is,
% % % % essentially, a trace.
% % % %
% % % % \subsection{MP Declare Templates}
% % % % On top of $\Gamma$, we can specify additional requirements, so as to constrain
% % % % the traces represented by its models to satisfy certain properties. In
% % % % particular, here, we are interested in capturing Declare templates...
% % % %
% % % % {\bf [FP: Si possono prendere da Burattin et al.? Magri possiamo dire che sono
% % % % essenzialmente analoghi e dare un paio di esempi (tipo init e response con
% % % % dati)]}
% % % %
% % % % \subsection{Introducing Numeric Values}
% % % % In many cases, we need to deal with numeric attribute values. For instance,
% % % % one can be interested in guaranteeing that the value of a certain attribute
% % % % lies below a fixed threshold, or that it equals a certain value.These
% % % % comparisons may occur within the Declare constraints.
% % % %
% % % % While, in
% % % % general, comparisons can be more involved, such as checking whether a value
% % % % is less than another value plus a certain quantity, here we address the
% % % % basic problem of comparing attributes against constant values. Specifically, we
% % % % assume that some attributes are \emph{integer}, i.e., they can be assigned
% % % % integer values and that the only allowed comparisons are of the form
% % % % $attr=num$, $attr\neq num$, $attr\leq num$, $attr<num$, $attr\geq num$, and
% % % % $attr>num$, with $attr$ an attribute and $num$ a (numeric) constant.
% % % %
% % % % Treating numeric values in FOL is in general problematic, as there is no way to
% % % % control the cardinality of infinite sets. In this respect, it is well known,
% % % % e.g., that one cannot write a theory whose models are (isomorphic to)
% % % % structures such as the integers (in fact, one cannot even express that the
% % % % universe be numerable). To overcome this, a series of frameworks
% % % % have been introduced, see, e.g., the framework of \emph{embedded finite
% % % % models}~\cite{libkin}.
% % % % %%
% % % % In
% % % % our case, the problem is even more basic: since we deal with
% % % % bounded sets only (in order to compile the theory into SAT),
% % % % including (no matter how) the integers in our theory is out of question.
% % % %
% % % % Nonetheless, if we allow only the comparisons mentioned above (extending to
% % % % more general conditions is left for future work) we can still reason about
% % % % numeric values.
% % % % The crucial observation is that in a Declare model, being finite,
% % % % there may occur only a finite number of numeric comparisons, thus only a finite
% % % % number of numeric values can be mentioned. Starting from this, we can
% % % % introduce, for each comparison, a new predicate symbol modeling whether
% % % % the comparison is satisfied or not by the involved attribute, and then use
% % % % this to force, through suitable sentences, its satisfaction.
% % % %
% % % % Assume for instance that the Declare model includes the comparison
% % % % $att_{ij}\leq 10$, with attribute $att_{ij}$ and constant $10$. To deal
% % % % with this, we proceed as follows:
% % % % \begin{itemize}
% % % %  \item We include in $\sigma$ a new unary predicate $lt10$.
% % % %  \item As constants associated with $att_{ij}$, we include in $\sigma$
% % % % only $\mathtt{d}_{ij}^1$ and $\mathtt{d}_{ij}^2$ and constrain,
% % % % through sentences~(\ref{eq1}), $D_{ij}$ to contain exactly
% % % % $\mathtt{d}_{ij}^1$ and $\mathtt{d}_{ij}^2$.
% % % %  \item We specify, through one of the sentences~(\ref{eq1}), that $lt10$
% % % % contains exactly $\mathtt{d}_{ij}^1$ (and not $d_{ij}^2$). This is intended to
% % % % model that $\mathtt{d}_{ij}^1$ is a numeric value less or equal than 10, while
% % % % $\mathtt{d}_{ij}^2$ is not.
% % % % \item Finally, whenever we need to state that the assignment of
% % % % $\mathtt{att}_{ij}$ at a given event $e$ must be less or equal than $10$, we
% % % % write $lt10(att(e,\mathtt{att}_{ij}))$.
% % % % \end{itemize}
% % % % Then, once the model is obtained, the desired trace can be obtained by
% % % % simply replacing every occurrence of $\mathtt{d}_{ij}^1$ with any value $v\leq
% % % % 10$ and every occurrence of $\mathtt{d}_{ij}^2$ with any value $v> 10$.
% % % %
% % % % For instance, if the obtained model $\I$ corresponds to trace:
% % % % $$\tau=A_1[v=\mathtt{d}_{11}^1]A_2[w=b,y=d]A_3[s=c]A_3[s=d]$$
% % % % and in the model we have that
% % % % $lt10^\I=\{\mathtt{d}_{11}^1\}$, then any trace obtained from $\tau$ by
% % % % replacing $\mathtt{d}_{11}^1$ with a value less or equal than 10 is a trace
% % % % that satisfies the desired requirement. In this sense, the constant
% % % % $\mathtt{d}_{11}^1$ is an \emph{abstraction} of the interval $(-\infty, 10]$
% % % % and $\mathtt{d}_{11}^2$ is an abstraction for $(10, +\infty]$
% % % %
% % % % The approach can be generalized to the case of multiple conditions.
% % % % To see how, assume there are two conditions involving overlapping intervals,
% % % % e.g., $att_{ij}\leq 10$ and $att_{lk}\leq 8$. In this case, we need to provide
% % % % an abstraction for all of the possible intervals defined by the mentioned
% % % % constants. We proceed as follows:
% % % % \begin{itemize}
% % % % 	\item We identify the  intervals of
% % % % interest, i.e,: $I_1=(-\infty,8]$, $I_2=(8,10]$, and $I_3=(10,+\infty)$.
% % % % 	\item For
% % % % each interval $I_m$, we introduce one new constant $\mathtt{d}^m$
% % % % abstracting the interval.
% % % % 	\item We introduce one predicate per interval,
% % % % namely:
% % % % $lteq8$, $gt8lteq10$, $gt10$.
% % % % 	\item We add sentences~(\ref{eq1}) requiring that both $D_{ij}$ and
% % % % $D_{lk}$ include exactly all of the constants $\mathtt{d}^m$ (thus match)
% % % % abstracting the intervals.
% % % % 	\item We add the following sentence, which expresses the relationship
% % % % between the values in the intervals:
% % % % $$
% % % % \begin{array}{l}
% % % % 	lteq8(\mathtt{d}^1)\land
% % % % 	\lnot gt8lteq10(\mathtt{d}^1)\land
% % % % 	\lnot gt10(\mathtt{d}^1)\land\\
% % % % 	\lnot lteq8(\mathtt{d}^2)\land	
% % % % 	gt8lteq10(\mathtt{d}^2)\land
% % % % 	\lnot gt10(\mathtt{d}^2)\land\\		
% % % % 	\lnot lteq8(\mathtt{d}^3)\land
% % % % 	\lnot gt8lteq10(\mathtt{d}^3)\land
% % % % 	gt10(\mathtt{d}^3)
% % % % \end{array}
% % % % $$
% % % % \item Finally, we use the relevant predicate to express that an attribute falls
% % % % into an interval. E.g., to express that $att_{ij}\leq 10$, we write
% % % % $\leq10(att(\mathtt{att}_{ij}))$.	
% % % % \end{itemize}
% % % % %%
% % % % The case where more conditions occur, or other relational operators, e.g.,
% % % % $=$, $\neq$, $>$, etc., are used is essentially analogous. Notice that,
% % % % differently from the case where no numeric values occur, in order to write the
% % % % theory, one needs to know in advance the Declare model.
% % % %
% % % %
% % % %
% % % % % \section{Solving Process Mining Problems with SAT}
% % % % %
% % % % % We now show how the various problems typical in Process Mining can be solved
% % % % % using the SAT technology. As explained above, we rely on existing tools that,
% % % % % given a FO theory and a bound on the size of its models, finds a model of the
% % % % % theory by compiling the problem into SAT. We consider such tools as black-boxes
% % % % % and do not enter the details of how the compilation is carried out. For this
% % % % % reason, we describe our approach by simply reporting the theory we use for each
% % % % % problem and how we interpret the obtained model.
% % % % %
% % % % %
% % % % % \subsection{Log Generation}
% % % % % Log generation is the problem of generating a set of traces, i.e., a log,
% % % % % compliant with a set of Declare templates. We can cast this problem to that of
% % % % % searching for a set of models of a theory. The idea is that each of the
% % % % % obtained models represents a trace compliant with the Declare model. To this
% % % % % end, assuming that the activities, the attributes and their respective domains
% % % % % are fixed, we proceed as follows:
% % % % %
% % % % % \begin{enumerate}
% % % % % \item we vary the size $b$ according to the desired length of the traces in
% % % % % the log;
% % % % % \item we produce the set $\C$ of sentences over $\sigma$ that correspond to
% % % % % the MP Declare constraints the traces must satisfy;
% % % % % \item we find as many models as desired of the theory
% % % % % $\Gamma\cup\C$.
% % % % % \end{enumerate}
% % % % %
% % % % % \subsection{Query Checking}
% % % % % Consider a trace $\tau=\tau_1,\ldots,\tau_n$ and a Declare template $\phi$. In
% % % % % general, templates have the form $\phi(X)$ or $\phi(X,Y)$, where $X$ and $Y$
% % % % % are activities to be instantiated depending on the requirements one needs to
% % % % % satisfy.
% % % % % %%
% % % % % While typically one deals with fully instantiated activities,
% % % % %
% % % % % Query checking is the problem of
% % % % %
% % % % %
% % % % % \subsection{Conformance Checking}
% % % % % Conformance checking is the problem of checking whether a given log, i.e., a
% % % % % set of traces, satisfies a Declare model. Here, we focus on the problem of
% % % % % checking only one trace, as the whole log can be checked by simply repeating
% % % % % the process for each trace.
% % % % %
% % % % % The problem can be easily expressed as a variant of log generation.
% % % % % The idea is to use the same theory as that for log generation, with the addition
% % % % % of a set of sentences constraining the trace to match the input trace. In this
% % % % % way, if the input trace satisfies the Declare constraints, then the theory has
% % % % % exactly one model, the input trace itself, which is then returned. If, instead,
% % % % % the trace does not satisfy the constraints, then the theory is inconsistent and
% % % % % no model is returned.
% % % % %
% % % % % In details, let $\tau$ be the input trace, with the $i$-th event consisting of
% % % % % an activity $a^i$ (we use superscripts to indicate the position in $\tau$),
% % % % % with associated attributes $att^i_1,\ldots,att^i_{\ell_i}$, with assigned
% % % % % values, respectively, $d^i_1,\ldots,d^i_\ell$. fix $b$ to the length of the
% % % % % input trace. We write the theory $\Gamma$ and the set of sentences
% % % % % $\C$ corresponding to the Declare constraints, as done for the case of log
% % % % % generation and, for every
% % % % % $i=1,\ldots,b$, add the following sentences:
% % % % % \begin{itemize}
% % % % % 	\item $act(\mathtt{e}_i)=\mathtt{a}^i$;
% % % % % 	\item for every $j=1,\ldots, \ell_i$,
% % % % % 			$att(\mathtt{e}_i,\mathtt{att}^i_j)=\mathtt{d}^i_j$.
% % % % % \end{itemize}
% % % % % %%
% % % % % Then, to check whether $\tau$ satisfies the Declare model, it is then enough to
% % % % % search for a model of the above theory. $\tau$ conforms to
% % % % % the Declare model if and only if some model exists.
% % % % %
% % % % %
% % % % %



\subsection{Conformance Checking}\label{subseq:alloy:conf-check}
To solve conformance checking in Alloy, we proceed as follows.
For every trace $\tau$ in the input set $L$, we write a theory
$\Gamma_\tau$ consisting of a set of FO sentences over $\sigma$,
which admits, for a suitable fixed interpretation domain $\Delta$, a
single model $\I_\tau$ which is also a model of $\Gamma$
corresponding to $\tau$. Once done so, we add the following
sentence to $\Gamma_\tau$, where $R$ is a $0$-ary relation
(i.e., a proposition):
$$\phi\doteq R\leftrightarrow \bigwedge_{\varphi\in \Phi} \varphi,$$
where $\Phi$ is the set of input formulas to be checked.
%%
Then, we run Alloy on the resulting theory, i.e., $\Gamma_\tau$
plus $\phi$, and the fixed interpretation domain $\Delta$,
and conclude that the trace $\tau$ satisfy
the input specification $\Phi$ iff the theory admits a model
where $R$ is true (by construction, if such a model exists
is unique).

\begin{example}\label{ex:alloy-c-check}
As an example, consider Ex.~\ref{ex:3} and trace $\tau$ of
Ex.~\ref{ex:1}.
The theory $\Gamma_\tau$ includes the following sentences:
\begin{itemize}
	\item $
		Seq(0,\cst a_1, 2)\land
		Seq(1,\cst a_2, 5)\land
		Seq(2,\cst a_3, 9)\land
		Seq(3,\cst a_3, 20)$
		
	\item $
		AttV(0,\cst v, \cst a)\land
		AttV(1,\cst w, \cst b)\land
		AttV(1,\cst y, \cst d)\land
		AttV(2,\cst s, \cst c)\land
		AttV(3,\cst s, \cst d)$
		
	\item $N(0)\land
			N(1)\land
			N(2)\land
			N(3)\land
			N(5)\land
			N(9)\land
			N(20)$		
			
	\item $\forall i,a,t.Seq(i,a,t)\rightarrow
		(i=0\land a= \cst a_1\land 2)\lor
		(i=1\land a= \cst a_2\land 5)\lor
		(i=2\land a= \cst a_3\land 9)\lor
		(i=3\land a= \cst a_3\land 20)$
		
	\item sentences analogous to the one above for
	$AttV$ and $N$ (omitted for brevity).
\end{itemize}
%%
For fixed $\Delta=\C=
			\set{0,1,2,3,5,9,20,
					\cst{a_1},\cst{a_2},\cst{a_3},
					\cst{v},\cst{w}, \cst{y},\cst{s},
					\cst{a},\cst{b},\cst{c},\cst{d}
				}
		$,
		it can be seen that the (only) model of $\Gamma_\tau$
		is also one for $\Gamma$ and represents trace
$\tau=\langle\cst{a}_1[
	\cst{t}=2,
	\cst{v}=\cst{a}
],
\cst{a}_2[
	\cst{t}=5,
	\cst{w}=\cst{b},
	\cst{y}=\cst{d}
],
\cst{a}_3[
	\cst{t}=9,
	\cst{s}=\cst{c}
],
\cst{a}_3[
	\cst{t}=20,
	\cst{s}=\cst{d}
]\rangle$.

Then, we define
$\phi=\textsc{Existence}(\cst a_2,\varphi)\land
\textsc{Response}(\cst a_1,\varphi_a,\cst a_2,\varphi_c)$,
for $\varphi$, $\varphi_a$, and $\varphi_c$ defined as
in Ex.~\ref{ex:3}.
%%
Finally, with Alloy, we search for a model of
$\Gamma_\tau\cup\set{\phi}$, giving $\Delta$ in input to Alloy.
In the case at hand, the returned model (there exists
exactly one) will have $R$ true, thus we conclude that
$\tau$ satisfies $\Phi$.
\end{example}



\subsection{Query Checking}
For query checking, we proceed in a way similar to
conformance checking (in fact, from a general point of
view, conformance checking can be seen as a special case
of query checking where the query has no free
variables, i.e., is boolean).

Given a trace $\tau\in L$, we produce the associated theory
(admitting one model only) $\Gamma_\tau$, as explained in
Sec.~\ref{subseq:alloy:conf-check}. Then, given the
query $\varphi(\vec x)$ with free variables
$\vec x$, we first add a new predicate symbol $Ans$
with arity $\card{\vec x}$ to $\sigma$, and then
add the following sentence to $\Gamma_\tau$:
$\forall \vec x.Ans(\vec x)\leftrightarrow \varphi(\vec x)$.
As a result, the (unique) model of the resulting theory
$\Gamma_\tau\cup\set{\varphi}$ is essentially the
same as that of $\Gamma$ but includes the additional relation
$Ans$ containing the tuples included in $\varphi^\I$.

Therefore, by feeding Alloy with the input consisting in
the theory $\Gamma_\tau\cup\set{\varphi}$ and the
interpretation domain associated with $\tau$, we can easily
answer the query.

\begin{example}
For an example of query checking, consider the problem
defined in Ex.~\ref{ex:q-check}, where the query
%%
$$\varphi=\forall i,t.(
			\forall\vec x.(
				(Seq(i,?x,t)\land \varphi_a)
					\rightarrow
$$

\vspace{-5mm}
$$
				(\exists i',t'.i'\geq i\land(
					\exists\vec x'.
						Seq(i',?y,t')\land\varphi_c)
				)
			)
		)
		$$
is evaluated against
trace~$\tau$ of Ex.~\ref{ex:1}:
$\tau=\langle\cst{a}_1[
	\cst{t}=2,
	\cst{v}=\cst{a}
],
\cst{a}_2[
	\cst{t}=5,
	\cst{w}=\cst{b},
	\cst{y}=\cst{d}
],
\cst{a}_3[
	\cst{t}=9,
	\cst{s}=\cst{c}
],
\cst{a}_3[
	\cst{t}=20,
	\cst{s}=\cst{d}
]\rangle$.

To answer the query, we simply feed Alloy with the theory
$\Gamma_\tau$ shown in Ex.~\ref{ex:alloy-c-check}
union the sentence
$\forall \vec x.Ans(\vec x)\leftrightarrow \varphi$,
and with the interpretation domain $\Delta$ associated with
$\tau$, and then ask Alloy to find a model of the
input theory.
The content of relation $Ans$ corresponds to the answer to
$\varphi$.
\end{example}


