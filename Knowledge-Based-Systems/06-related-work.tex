\section{Related Work and Conclusions}
In the literature, different extensions of Declare have been proposed. Some of these extensions have been used as a basis to develop Process Mining algorithms.
In \cite{Westergaard.Maggi/CoopIS2012:LookingintoFuture,DBLP:journals/ijcis/MaggiW14}, the authors introduce \emph{Timed Declare}, an extension of Declare whose semantics is based on a metric temporal logic that allows Declare constraints to be enriched with temporal conditions on timestamps. In \cite{DBLP:conf/bir/Maggi14}, such extension is used for the discovery of Timed Declare constraints.
%\cite{DBLP:conf/ruleml/BernardiCFM14,DBLP:journals/is/BernardiCFM16} presents an approach for the discovery of Declare rules characterizing the lifecycle of non-atomic activities in a log.

In the work proposed in \cite{Maggi.etal/BPM2013:DiscoveringDataAware}, another extension of Declare has been proposed whose semantics is defined by using a first-order variant of LTL that allows Declare constraints to be enriched with conditions on data. This semantics has been used in \cite{Maggi.etal/BPM2013:DiscoveringDataAware,DBLP:conf/bpm/BoseMA13} for the development of process discovery algorithms that produce data-aware Declare constraints from logs. In \cite{DBLP:journals/eswa/BurattinMS16}, MP-Declare is introduced and a technique for conformance checking based on MP-Declare is presented. Techniques for the discovery of MP-Declare constraints from event logs are presented in \cite{DBLP:conf/bpm/LenoDM18,Schoenig.etal/ICSOC2016:DiscoveryMultiperspectiveDeclare}.

%The presented technique is based on the translation of \textsc{Declare} templates into SQL queries on a relational database instance, where the event log has previously been stored. The query answer assigns the free variables with those tasks that lead to the satisfaction of the constraint in the event log. The methodology has later been extended towards multi-perspective \textsc{Declare} discovery~\cite{Schoenig.etal/ICSOC2016:DiscoveryMultiperspectiveDeclare}, to include data in the formulation of constraints.

In this paper, we presented a framework based on SAT to solve problems in the context of Declarative Process Mining.
We showed how three specific Declarative Process Mining problems (namely log generation, conformance checking and temporal query checking) can be expressed as suitable FO theories and solved by compilation into SAT using Alloy. The Alloy output, i.e., the SAT instance, can be provided as input to any SAT solver, thus making any performance optimization in SAT directly
available to Declarative Process Mining. While existing approaches for Declarative Process Mining are tailored for Declare, our approach can deal with any property expressed as a FOL constraint.

As future work, we would like to compare the performance of our tool using different SAT solvers. We are currently working on an approach based on SAT for monitoring the compliance of ongoing process executions wrt a set of MP-Declare constraints.  