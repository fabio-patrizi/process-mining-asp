\documentclass[review]{elsarticle}

\usepackage{graphicx}

% If you use the hyperref package, please uncomment the following line
% to display URLs in blue roman font according to Springer's eBook style:
% \renewcommand\UrlFont{\color{blue}\rmfamily}

\usepackage{xspace}
\usepackage{tikz}

\usepackage{amsfonts}
\usepackage[title]{appendix}
\usepackage{pgfplots}
\usepackage{pgfplotstable}
\usepackage{booktabs}
\usepackage[defblank]{paralist}
\usepackage{enumitem}
\usepackage{verbatim}
\usepackage{amssymb}
\usepackage{rotating}
\usepackage{url}
\usepackage{tabularx}
\usepackage{booktabs}
\usepackage{amsmath}
\usepackage{color}
\usepackage{verbatim}
\usepackage{epsfig}
\usepackage{amssymb}
\usepackage{rotating}
\usepackage{enumitem}
%\usepackage{arydshln}
% \usepackage{ngerman}
%\usepackage{hyperref}
\usepackage{url}
\usepackage{tabularx}
\usepackage{booktabs}
\usepackage[clock,electronic]{ifsym}
\usepackage{eurosym}
\usepackage{subfig}
\usepackage{xspace}
\usepackage[defblank]{paralist}
\usepackage{multirow}
\usepackage{booktabs}
\usepackage{amssymb}  % additional symbols (there are more packages)
\usepackage{gensymb}
\usepackage{makecell}
\usepackage[ruled, vlined, linesnumbered]{algorithm2e}
\newcommand{\removelatexerror}{\let\@latex@error\@gobble}
\usepackage{wrapfig}
%\usepackage[pdftex]{graphicx} % graphics packages
%\usepackage[pdftex]{color}    % color packages
\usepackage{times}
%\usepackage{todonotes}		% todonotes
\usepackage{dsfont}

\newtheorem{example}{Example}

%% CUSTOM MACROS AND PACKAGES
\usetikzlibrary{DECLARE}
\usepackage{declare-marco}
\RequirePackage{rotating}

\newcommand{\sidewaysheader}[1]{\begin{sideways}{#1}\end{sideways}}
\newcommand{\sidewaysheaderpar}[2]{\begin{sideways}\parbox{#1}{\centering #2}\end{sideways}}

\RequirePackage{paralist}

\newenvironment{iiilist}%
{\begin{inparaenum}[\itshape(i)\upshape]}%
{\end{inparaenum}}

\RequirePackage{colortbl}

\newcommand{\grayrow}{\rowcolor{gray!20}} %
\newcommand{\lightgrayrow}{\rowcolor{gray!10}} %
\newcommand{\graycell}{\cellcolor{gray!30}} %
\def\paramx {\taskize{A}}
\def\paramy {\taskize{B}}
\def\paramz {\taskize{C}}
\def\paramw {\taskize{D}}
\def\letterx {\ensuremath{a}}
\def\lettery {\ensuremath{b}}
\def\letterz {\ensuremath{c}}
\def\letterw {\ensuremath{d}}
\newcommand{\taskize}[1] {\ensuremath{\scalebox{0.85}{\textsf{#1}}}}
\def\taska {\taskize{a}}
\def\taskb {\taskize{b}}
\def\taskc {\taskize{c}}
\def\taskd {\taskize{d}}

\def\ExiTxt {Existence}
\def\PartTxt {Participation}
\def\AbseTxt {Absence}
\def\UniqTxt {AtMostOne}
\def\InitTxt {Init}
\def\EndTxt {End}
\def\ResExTxt {RespondedExistence}
\def\RespTxt {Response}
\def\AltRespTxt {AlternateResponse}
\def\AltRespTxtShort {Alt.Response}
\def\ChaRespTxt {ChainResponse}
\def\PrecTxt {Precedence}
\def\AltPrecTxt {AlternatePrecedence}
\def\AltPrecTxtShort {Alt.Precedence}
\def\ChaPrecTxt {ChainPrecedence}
\def\CoExiTxt {CoExistence}
\def\SuccTxt {Succession}
\def\AltSuccTxt {AlternateSuccession}
\def\AltSuccTxtShort {Alt.Succession}
\def\ChaSuccTxt {ChainSuccession}
\def\NotCoExiTxt {NotCoExistence}
\def\NotSuccTxt {NotSuccession}
\def\NotChaSuccTxt {NotChainSuccession}
%
\def\ExiTmp {\ensuremath{\textsc{\ExiTxt}}}
\def\PartTmp {\ensuremath{\textsc{\PartTxt}}}
\def\AbseTmp {\ensuremath{\textsc{\AbseTxt}}}
\def\UniqTmp {\ensuremath{\textsc{\UniqTxt}}}
\def\InitTmp {\ensuremath{\textsc{\InitTxt}}}
\def\EndTmp {\ensuremath{\textsc{\EndTxt}}}
\def\ResExTmp {\ensuremath{\textsc{\ResExTxt}}}
\def\RespTmp {\ensuremath{\textsc{\RespTxt}}}
\def\AltRespTmp {\ensuremath{\textsc{\AltRespTxt}}}
\def\ChaRespTmp {\ensuremath{\textsc{\ChaRespTxt}}}
\def\PrecTmp {\ensuremath{\textsc{\PrecTxt}}}
\def\AltPrecTmp {\ensuremath{\textsc{\AltPrecTxt}}}
\def\ChaPrecTmp {\ensuremath{\textsc{\ChaPrecTxt}}}
\def\CoExiTmp {\ensuremath{\textsc{\CoExiTxt}}}
\def\SuccTmp {\ensuremath{\textsc{\SuccTxt}}}
\def\AltSuccTmp {\ensuremath{\textsc{\AltSuccTxt}}}
\def\ChaSuccTmp {\ensuremath{\textsc{\ChaSuccTxt}}}
\def\NotCoExiTmp {\ensuremath{\textsc{\NotCoExiTxt}}}
\def\NotSuccTmp {\ensuremath{\textsc{\NotSuccTxt}}}
\def\NotChaSuccTmp {\ensuremath{\textsc{\NotChaSuccTxt}}}
%
\newcommand{\Exi}[2] {\ensuremath{\textsc{\ExiTxt}(#1,#2)}}
\newcommand{\Part}[1] {\ensuremath{\textsc{\PartTxt}(#1)}}
\newcommand{\Abse}[2] {\ensuremath{\textsc{\AbseTxt}(#1,#2)}}
\newcommand{\Uniq}[1] {\ensuremath{\textsc{\UniqTxt}(#1)}}
\newcommand{\Ini}[1] {\ensuremath{\textsc{\InitTxt}(#1)}}
\newcommand{\End}[1] {\ensuremath{\textsc{\EndTxt}(#1)}}
\newcommand{\ResEx}[2] {\ensuremath{\textsc{\ResExTxt}(#1,#2)}}
\newcommand{\Resp}[2] {\ensuremath{\textsc{\RespTxt}(#1,#2)}}
\newcommand{\AltRes}[2] {\ensuremath{\textsc{\AltRespTxt}(#1,#2)}}
\newcommand{\AltResp}[2] {\ensuremath{\textsc{\AltRespTxt}(#1,#2)}}
\newcommand{\AltRespShort}[2] {\ensuremath{\textsc{\AltRespTxtShort}(#1,#2)}}
\newcommand{\ChaResp}[2] {\ensuremath{\textsc{\ChaRespTxt}(#1,#2)}}
\newcommand{\ChaRes}[2] {\ensuremath{\textsc{\ChaRespTxt}(#1,#2)}}
\newcommand{\Prec}[2] {\ensuremath{{\textsc{\PrecTxt}}(#1,#2)}}
\newcommand{\AltPrec}[2] {\ensuremath{\textsc{\AltPrecTxt}(#1,#2)}}
\newcommand{\AltPrecShort}[2] {\ensuremath{\textsc{\AltPrecTxtShort}(#1,#2)}}
\newcommand{\ChaPrec}[2] {\ensuremath{\textsc{\ChaPrecTxt}(#1,#2)}}
\newcommand{\CoExi}[2] {\ensuremath{\textsc{\CoExiTxt}(#1,#2)}}
\newcommand{\Succ}[2] {\ensuremath{\textsc{\SuccTxt}(#1,#2)}}
\newcommand{\AltSucc}[2] {\ensuremath{\textsc{\AltSuccTxt}(#1,#2)}}
\newcommand{\AltSuccShort}[2] {\ensuremath{\textsc{\AltSuccTxtShort}(#1,#2)}}
\newcommand{\ChaSucc}[2] {\ensuremath{\textsc{\ChaSuccTxt}(#1,#2)}}
\newcommand{\NotCoExi}[2] {\ensuremath{\textsc{\NotCoExiTxt}(#1,#2)}}
\newcommand{\NotSucc}[2] {\ensuremath{\textsc{\NotSuccTxt}(#1,#2)}}
\newcommand{\NotChaSucc}[2] {\ensuremath{\textsc{\NotChaSuccTxt}(#1,#2)}} 
\newcommand{\lnext}{\ensuremath{\mathbf{X}}}
\newcommand{\lwnext}{\ensuremath{\mathbf{\bar{X}}}}
\newcommand{\luntil}{\ensuremath{\mathbf{U}}}
\newcommand{\lsince}{\ensuremath{\mathbf{S}}}
\newcommand{\lrelease}{\ensuremath{\mathbf{R}}}
\newcommand{\lwuntil}{\ensuremath{\mathbf{W}}}
\newcommand{\lglobally}{\ensuremath{\mathbf{G}}}
\newcommand{\lfuture}{\ensuremath{\mathbf{F}}}
\newcommand{\tnext}{\ensuremath{\mathbf{X}_{I}}}
\newcommand{\twnext}{\ensuremath{\mathbf{\bar{X}_I}}}
\newcommand{\tuntil}{\ensuremath{\mathbf{U}_{I}}}
\newcommand{\tsince}{\ensuremath{\mathbf{S}_{I}}}
\newcommand{\trelease}{\ensuremath{\mathbf{R}_{I}}}
\newcommand{\tglobally}{\ensuremath{\mathbf{G}_{I}}}
\newcommand{\lonce}{\ensuremath{\mathbf{O}}}
\newcommand{\tonce}{\ensuremath{\mathbf{O}_{I}}}
\newcommand{\lyesterday}{\ensuremath{\mathbf{Y}}}
\newcommand{\tyesterday}{\ensuremath{\mathbf{Y}_{I}}}
\newcommand{\lhistorically}{\ensuremath{\mathbf{H}}}
\newcommand{\thistorically}{\ensuremath{\mathbf{H}_{I}}}
\newcommand{\tfuture}{\ensuremath{\mathbf{F}_{I}}}
%\newcommand{\true}{\ensuremath{\mbox{true}}}
%\newcommand{\false}{\ensuremath{\mbox{false}}}
%\newcommand{\n}{\ensuremath{\figitem{N}}}}
%\newcommand{\x}{\ensuremath{\figitem{X}}}
%\newcommand{\y}{\ensuremath{\bar{\textsf{X}}}}
%\newcommand{\R}{\ensuremath{\mathbf{R}_+}}

\newcommand{\condec}{Declare\xspace}
%% example environment
% \newtheorem{example}{Example}[section]
%% definition environment
% \newtheorem{definition}{Definition}[section]
%% definition environment
%\newtheorem{theorem}{Theorem}
%% definition environment
%\newtheorem{lemma}{Lemma}
% \newenvironment{proof}                          % Proof environment.
%   {\trivlist\item[\hskip\labelsep\bf Proof:]}%  %   Puts "Proof:" in front of
%   {\unskip\nobreak\hfill~$\Box$\endtrivlist}    %   and box symbol after text.
%
% \newtheorem{corollary}[theorem]{Corollary}

%%%%%%%%%%%%%%%%%%%%%%%%%% Special symbols

\begingroup
\catcode`\~=11
\gdef\urltilde{\lower 0.6ex\hbox{~}}
\endgroup

%%%%%%%%%%%%%%%%%%%%%%%%%% General Math

%\newcommand{\A}{\mathcal{A}} \newcommand{\B}{\mathcal{B}}
%\newcommand{\C}{\mathcal{C}} \newcommand{\D}{\mathcal{D}}
%\newcommand{\E}{\mathcal{E}} \newcommand{\F}{\mathcal{F}}
%\newcommand{\G}{\mathcal{G}} \renewcommand{\H}{\mathcal{H}}
%\newcommand{\I}{\mathcal{I}} \newcommand{\J}{\mathcal{J}}
%\newcommand{\K}{\mathcal{K}} \renewcommand{\L}{\mathcal{L}}
%\newcommand{\M}{\mathcal{M}} \newcommand{\N}{\mathcal{N}}
%\renewcommand{\O}{\mathcal{O}} \renewcommand{\P}{\mathcal{P}}
%\newcommand{\Q}{\mathcal{Q}} \newcommand{\R}{\mathcal{R}}
%\renewcommand{\S}{\mathcal{S}} \newcommand{\T}{\mathcal{T}}
%\newcommand{\U}{\mathcal{U}}
%\newcommand{\V}{\mathcal{V}}
%\newcommand{\W}{\mathcal{W}} \newcommand{\X}{\mathcal{X}}
%\newcommand{\Y}{\mathcal{Y}} \newcommand{\Z}{\mathcal{Z}}
\newcommand{\BE}{{\mathcal{B}, \mathcal{E}}}
\newcommand{\DB}{{\mathcal{DB}}}
\newcommand{\DS}{{\mathcal{DS}}}

%% ul with space between text and line

\newcommand{\ra}{\rightarrow}
\newcommand{\Ra}{\Rightarrow}
\newcommand{\la}{\leftarrow}
\newcommand{\La}{\Leftarrow}
%\newcommand{\lra}{\leftrightarrow}
\newcommand{\Lra}{\Leftrightarrow}
\newcommand{\lora}{\longrightarrow}
\newcommand{\Lora}{\Longrightarrow}
\newcommand{\lola}{\longleftarrow}
\newcommand{\Lola}{\Longleftarrow}
\newcommand{\lolra}{\longleftrightarrow}
\newcommand{\Lolra}{\Longleftrightarrow}
\newcommand{\ua}{\uparrow}
\newcommand{\Ua}{\Uparrow}
\newcommand{\da}{\downarrow}
\newcommand{\Da}{\Downarrow}
\newcommand{\uda}{\updownarrow}
\newcommand{\Uda}{\Updownarrow}

\newcommand{\goto}[1]{\stackrel{#1}{\lora}}

%%%%%%%%%%%%%%%%%%%%%%%%%% Relations

\newcommand{\incl}{\subseteq}
\newcommand{\imp}{\rightarrow}
%%\newcommand{\deq}{\doteq}
\newcommand{\dleq}{\dot{\leq}}                   % dotted less equal

%%%%%%%%%%%%%%%%%%%%%%%%%% Spaces

\newcommand{\per}{\mbox{\bf .}}                  % period

\newcommand{\cld}{,\ldots,}                      % ,...,
\newcommand{\ld}[1]{#1 \ldots #1}                 % #1 ... #1
\newcommand{\cd}[1]{#1 \cdots #1}                 % #1 ... #1
\newcommand{\lds}[1]{\, #1 \; \ldots \; #1 \,}    % _#1_..._#1_
\newcommand{\cds}[1]{\, #1 \; \cdots \; #1 \,}    % _#1_..._#1_

\newcommand{\dd}[2]{#1_1,\ldots,#1_{#2}}             % x1,...,xn (da da)
\newcommand{\ddd}[3]{#1_{#2_1},\ldots,#1_{#2_{#3}}}  % xi1,...,xin (da da down)
\newcommand{\dddd}[3]{#1_{11}\cld #1_{1#3_{1}}\cld #1_{#21}\cld #1_{#2#3_{#2}}}
%%                                x_11,...,x_1n,...,x_m1,...,x_mn_m

\newcommand{\ldop}[3]{#1_1 \ld{#3} #1_{#2}}   % x1 #3...#3 xn
\newcommand{\cdop}[3]{#1_1 \cd{#3} #1_{#2}}   % x1 #3...#3 xn
\newcommand{\ldsop}[3]{#1_1 \lds{#3} #1_{#2}} % x1 _#3_..._#3_ xn
\newcommand{\cdsop}[3]{#1_1 \cds{#3} #1_{#2}} % x1 _#3_..._#3_ xn

%%%%%%%%%%%%%%%%%%%%%%%%%% Delimiters

\newcommand{\quotes}[1]{{\lq\lq #1\rq\rq}}
%\newcommand{\set}[1]{\{#1\}}                      % set
\newcommand{\Set}[1]{\left\{#1\right\}}
%%\newcommand{\bigset}[1]{\Bigl\{#1\Bigr\}}
\newcommand{\bigmid}{\Big|}
\newcommand{\card}[1]{|{#1}|}                     % cardinality of a set
\newcommand{\Card}[1]{\left| #1\right|}
\newcommand{\cards}[1]{\sharp #1}
\newcommand{\sub}[1]{[#1]}
\newcommand{\tup}[1]{\langle #1\rangle}            % tuple
\newcommand{\Tup}[1]{\left\langle #1\right\rangle}

%%%%%%%%%%%%%%%%%%%%%%%%%% Constraints

%%\newcommand{\relc}[3]{#1#2#3}
\newcommand{\inc}[2]{#1\colon #2}

%%%%%%%%%%%%%%%%%%%%%%%%%% DL specific commands

%%%%%%%%%%%%%%%%%%%%%%%%%% Interpretation

\newcommand{\inter}[1][\I]{(\dom[#1],\Int[#1]{\cdot})}   % (dom^I,.^I)

%\newcommand{\dom}[1][\I]{\Delta^{#1}}  % delta^I (domain of interpretation \I)
\newcommand{\Int}[2][\I]{#2^{#1}}      % #2^I    (interpretation function)
\newcommand{\INT}[2][\I]{(#2)^{#1}}    % (#2)^I  (interpretation function)

%%%%%%%%%%%%%%%%%%%%%%%%%% Basic Concept forming operators

%\newcommand{\AND}{\sqcap}
%\newcommand{\OR}{\sqcup}
% \newcommand{\NOT}{\neg}
\newcommand{\ALL}[2]{\forall #1 \per #2}
 \newcommand{\SOME}[2]{\exists #1 \per #2}
 \newcommand{\SOMET}[1]{\exists #1}
\newcommand{\ALLRC}{\ALL{R}{C}}
\newcommand{\SOMERC}{\SOME{R}{C}}
\newcommand{\SOMERT}{\SOMET{R}}
\newcommand{\ATLEAST}[2]{(\geq #1 \, #2)}
\newcommand{\ATMOST}[2]{(\leq #1 \, #2)}
\newcommand{\EXACTLY}[2]{(= #1 \, #2)}
 \newcommand{\INV}[1]{#1^{-}}

%%%%%%%%%%%%%%%%%%%%%%%%%%


%%\newcommand{\BOX}[2]{\lbrack #1\rbrack #2}
%%\newcommand{\DIAM}[2]{\langle #1\rangle #2}
\newcommand{\QATMOST}[3]{(\leq #1\; \DIAM{#2}{#3})}
%%\newcommand{\true}{tt}
%%\newcommand{\false}{ff}
\newcommand{\Ident}[1]{(#1)?}
%\renewcommand{\circ}{;}


%%\newfont{\bigmathxx}{cmsy10 scaled 1440}
%%\newcommand{\bigsqcap}{\mathop{\mathop{\mbox{\bigmathxx\symbol{117}}}}}

\newcommand{\ISA}{\sqsubseteq}
\newcommand{\EQU}{\equiv}

\newcommand{\QATLEAST}[3]{(\geq #1\, #2 \per #3)}
%%%\newcommand{\QATMOST}[3]{(\leq #1\, #2 \per #3)}
%%%\newcommand{\Ident}[1]{\mathit{id}(#1)}

\newcommand{\ALCQreg}{\mathcal{ALCQ}_\mathit{reg}}
\newcommand{\PDLgm}{\mathit{PDL}_\mathit{gm}}
\newcommand{\ALC}{\mathcal{ALC}}

\newcommand{\Nat}{{\rm I\kern-.23em N}}

\newcommand{\FaCT}{\textsc{FaCT}\xspace}
\newcommand{\Racer}{\textsc{Racer}\xspace}
\newcommand{\SHIQ}{\mathcal{SHIQ}\xspace}

\newcommand{\served}{\mathsf{served}}
\newcommand{\initiated}{\mathsf{initiated}}
\newcommand{\trans}{\mathit{trans}}
%%\newcommand{\Init}{\mathsf{Init}}

\newcommand{\ecard}{\emph{e}-card\xspace}

%%%%%%%%%%%%%%%%%%%%%



%%%%%%%%%%%%%%%%%%%%%%%%%%%%%%%%%%%%%%%%%%%%%%%%%%%%%%%%
% Macro and personal command
%%%%%%%%%%%%%%%%%%%%%%%%%%%%%%%%%%%%%%%%%%%%%%%%%%%%%%%%

\newcommand{\eService}{\emph{e-}Service\xspace}
\newcommand{\eServices}{\emph{e-}Services\xspace}
\newcommand{\eservice}{\emph{e-}Service\xspace}
\newcommand{\eservices}{\emph{e-}Services\xspace}
\newcommand{\eGov}{\emph{e-}Government\xspace}
\newcommand{\e}{\emph{e-}}


\newcommand{\Init}{\mathit{Init}}
%%\newcommand{\init}{_{\mathit{init}}}
\newcommand{\KB}{\mathit{KB}}
\newcommand{\VA}{\mathit{VA}}
\newcommand{\VAb}{\mathbf{VA}}

\newcommand{\exec}{\mathit{exec}}
\newcommand{\change}{\mathit{Change}}
\newcommand{\eS}{\mathit{e}\S}
\newcommand{\TeS}{\T^{\eS}}
\newcommand{\Final}{\mathit{Final}}
\newcommand{\Step}{\mathit{Step}}
\newcommand{\Poss}{\mathit{Poss}}
%\newcommand{\undef}{\mathit{undef}}
\newcommand{\fin}{\mathit{fin}}
\newcommand{\comp}{\mathit{comp}}
%%\newcommand{\true}{\mathtt{true}}
%%\newcommand{\false}{\mathtt{false}}

% \newcommand{\ttrue}{\mathit{true}}
\newcommand{\ffalse}{\mathit{false}}

\newcommand{\limp}{\rightarrow}
\newcommand{\lequiv}{\leftrightarrow}
\newcommand{\conc}{\mathord{\cdot}}
\newcommand{\eword}{\varepsilon}
\newcommand{\Diam}[2]{\langle #1 \rangle #2}
\newcommand{\Boxx}[2]{[#1]#2}

\newcommand{\myi}{\emph{(i)}\xspace}
\newcommand{\myii}{\emph{(ii)}\xspace}
\newcommand{\myiii}{\emph{(iii)}\xspace}
\newcommand{\myiv}{\emph{(iv)}\xspace}
\newcommand{\myv}{\emph{(v)}\xspace}
\newcommand{\myvi}{\emph{(vi)}\xspace}

%\newcommand{\my}[1]{\emph{(#1)}\xspace}

\newcommand{\moved}{\mathit{moved}}
\newcommand{\itext}{{\mathit{ext}}}
\newcommand{\itint}{{\mathit{int}}}

\newcommand{\ES}[1]{{#1}^\itext}
\newcommand{\IS}[1]{{#1}^\itint}
\newcommand{\EA}[1]{A^\itext_{#1}}
\newcommand{\IA}[1]{A^\itint_{#1}}
\newcommand{\CG}{CG}
\newcommand{\NCG}{NCG}
\newcommand{\MNCG}{MNCG}
\newcommand{\CHOOSE}{choose(\cdot)}
\newcommand{\NDSP}{NDSP}
\newcommand{\NDS}{NDS}
\newcommand{\MNDS}{MNDS}
\newcommand{\PLAN}{\mbox{\emph{SP}}}

%% Draft delle figure
\newcommand{\figureDraft}{
\begin{tabular}{|p{7cm}|}
\hline
\vspace{4cm}\\
\hline
\end{tabular}
}


\newcommand{\ttt}{{\mathit{tt}}}
\newcommand{\fff}{{\mathit{ff}}}
\newcommand{\uuu}{{\mathit{uu}}}
\newcommand{\transA}[3]{\{#1\}#2\{#3\}}
\newcommand{\transT}[2]{\{#1\}#2}

\newcommand{\reminder}[1]{\marginpar{\mbox{$<NEW>$}} [[{\it\small #1}]]}


\newcommand{\bsy}{\boldsymbol}
\newcommand{\certans}{\rhd}

\newcommand{\Goto}[1]{\stackrel{#1}{\Longrightarrow}}
\newcommand{\mf}{\mathfrak}
\newcommand{\msf}{\mathsf}
\newcommand{\homeq}{\stackrel{h}{=}}
\newcommand{\eqmnr}{\stackrel{mnr}{=}}
\newcommand{\sseqmnr}{\stackrel{mnr}{\subseteq}}
\newcommand{\ssmnr}{\stackrel{mnr}{\subset}}


%%%%%%%%%%%
% \newcommand\BOX[1]{\lbrack #1 \rbrack}
% \newcommand\DIAM[1]{\langle #1 \rangle}
% \newcommand{\Var}{\mbox{\em Var}}
%%%%%%%%%%% Mu-Calculus Interpretation: \mf{A}, \V

\newcommand{\Moda}[1]{#1_{\V}^\mf{A}}     % argument_{\V}^\mf{A} % Giuseppe
                                        % interp. and assign. fun.
\newcommand{\Modax}[2]{#1_{\V #2}^\mf{A}} % argument_{\V}^\mf{A} % Giuseppe
                                        % interp. and modif. assign. fun.
\newcommand{\MODA}[1]{(#1)_{\V}^\mf{A}}   % (argument)_{\V}^\mf{A} % Giuseppe
                                        % interp. and assign. fun.
\newcommand{\MODAX}[2]{(#1)_{\V #2}^\mf{A}} % (argument)_{\V.}^\mf{A} % Giuseppe
                                        % interp. and modif. assign. fun.

\newcommand{\Modat}[2]{#1_{\V}^{#2}}     % argument_{\V}^\mf{A} % Giuseppe
                                        % interp. and assign. fun.
\newcommand{\MODAT}[2]{(#1)_{\V}^{#2}}   % (argument)_{\V}^\mf{A} % Giuseppe
                                        % interp. and assign. fun.

\renewcommand{\varsigma}{\mathit{R}}
%\renewcommand{\Ra}{\mathit{Tr}}

\newcommand{\Amax}{\A_{\mathit{max}}}

% \newcommand{\ex}[1]{\mathsf{#1}}
\newcommand{\actex}[1]{\mathsf{#1}}

\newcommand{\const}[1]{\C_{#1}}

%\newcommand{\bsy}[1]{\boldsymbol{#1}}
\newcommand{\brho}{\bsy{\rho}}
\newcommand{\bpi}{\bsy{\pi}}

\newcommand{\ans}[2][]{\mathit{ans}_{#1}(#2)}
\newcommand{\rew}[1]{\mathit{rew}(#1)}
\newcommand{\DO}[1]{\mathit{do}(#1)}
\newcommand{\GOTO}[1][\mf{A}]{\mathrel{\Ra_{#1}}}

%\newcommand{\conj}{\mathit{conj}}

\newcommand{\homom}[4][\C]{#3 \mathrel{\ra^{#1}_{#2}} #4}
\newcommand{\map}[2]{#1 \rightsquigarrow #2}
\newcommand{\carule}[2]{#1 \mapsto #2}

\newcommand{\dllite}{\emph{DL-Lite}\xspace}
\newcommand{\dllitef}{\emph{DL-Lite}\ensuremath{_{\mathcal{F}}}\xspace}
\newcommand{\dlliter}{\emph{DL-Lite}\ensuremath{_{\mathcal{R}}}\xspace}
\newcommand{\dllitea}{\emph{DL-Lite}\ensuremath{_{\mathcal{A}}}\xspace}



%%%%%%%%%%%%%%%%%%%%%%%%%%%%%%%%%%%%%%%%%%%%%%%%%%%%%%
%%%%%%%%%%%%%%%%%%%%%%%%%%%%%%%%%%%%%%%%%%%%%%%%%%%%%%


\newcommand{\LTLthree}{{\sc ltl}$_3$}
%\newcommand{\bs}{\boldsymbol}
\newcommand{\univproj}[1]{#1_{\downarrow \forall}}
\newcommand{\existproj}[1]{#1_{\downarrow \exists}}
\newcommand{\bsl}{\backslash}
\newcommand{\prop}{\P rop}
\newcommand{\pfunct}{\mathsf{p}}
\newcommand{\ffunct}{\mathsf{f}}
\newcommand{\rfunct}{\mathsf{r}}

\newcommand{\old}{\mathsf{old}}
\newcommand{\upsphi}{\upsilon^{\Phi}}
\newcommand{\upsnotphi}{\upsilon^{\neg \Phi}}
\newcommand{\epsphi}{\epsilon^{\Phi}}
\newcommand{\epsnotphi}{\epsilon^{\neg \Phi}}

%\newcommand{\id}{\textsc{id}\xspace}
\newcommand{\ids}{\textsc{id}s\xspace}

\newcommand{\pid}{p-\textsc{id}\xspace}
\newcommand{\pids}{p-\textsc{id}s\xspace}

\newcommand{\uid}{$\forall$-\textsc{id}\xspace}
\newcommand{\uids}{$\forall$-\textsc{id}s\xspace}

\newcommand{\eid}{$\exists$-\textsc{id}\xspace}
\newcommand{\eids}{$\exists$-\textsc{id}s\xspace}

\newcommand{\cid}{c-\textsc{id}\xspace}
\newcommand{\cids}{c-\textsc{id}s\xspace}

\newcommand{\coddid}{Codd-\textsc{id}\xspace}
\newcommand{\coddids}{Codd-\textsc{id}s\xspace}

\newcommand{\offid}{offline-\textsc{id}\xspace}
\newcommand{\offids}{offline-\textsc{id}s\xspace}


\newcommand{\prun}{p-run\xspace}

\newcommand{\urun}{$\forall$-run\xspace}

\newcommand{\erun}{$\exists$-run\xspace}

\newcommand{\crun}{c-run\xspace}
\newcommand{\cruns}{c-runs\xspace}

\newcommand{\coddrun}{Codd-run\xspace}

\newcommand{\offrun}{offline-run\xspace}

\newcommand{\Buchi}{B\"{u}chi\xspace}
\renewcommand{\=}{\! = \!}

\newcommand{\nv}{\star}

\newcommand{\coddify}{\mathsf{coddify}}
\newcommand{\rep}{\mathsf{rep}}
\newcommand{\poss}{\mathsf{poss}}

\newcommand{\cin}{\! \in \!}

%% Logics
\newcommand{\FO}{{\sc fo}\xspace}
\newcommand{\LT}{{\sc lt}$_f$\xspace}
\newcommand{\LTL}{{\sc ltl}\xspace}
\newcommand{\XES}{{\sc xes}\xspace}
\newcommand{\EC}{{\sc ec}\xspace}
\newcommand{\FOLTL}{{\sc fo-ltl}\xspace}
\newcommand{\RVLTL}{{\sc rv-ltl}\xspace}
\newcommand{\LTLf}{{\sc ltl}$_f$\xspace}
\newcommand{\LTLi}{{\sc ltl}\xspace}
\newcommand{\LDL}{{\sc ldl}\xspace}
\newcommand{\add}{\mathit{ADD}}
\newcommand{\del}{\mathit{DEL}}
\newcommand{\LDLf}{{\sc ldl}$_f$\xspace}
\newcommand{\RE}{{\sc re}$_f$\xspace}
\newcommand{\PDL}{{\sc pdl}\xspace}
\newcommand{\FOLf}{{\sc fol}$_f$\xspace}
\newcommand{\MSOf}{{\sc mso}$_f$\xspace}
\newcommand{\FOL}{{\sc fol}\xspace}
\newcommand{\MSO}{{\sc mso}\xspace}
%\newcommand{\ATA}{{\sc ata}\xspace}
\newcommand{\AFW}{{\sc afw}\xspace}
\newcommand{\NFA}{{\sc nfa}\xspace}
\newcommand{\DFA}{{\sc dfa}\xspace}
\newcommand{\CTL}{{\sc ctl}\xspace}
\newcommand{\QLTL}{{\sc qltl}\xspace}
\newcommand{\muLTL}{$\mu${\sc ltl}\xspace}
\newcommand{\declare}{{\sc Declare}}
\newcommand{\fol}{\mathit{fol}}
\newcommand{\f}{\mathit{f}}
\newcommand{\g}{\mathit{g}}
\newcommand{\re}{\mathit{re}}

%% LTL
\newcommand{\Next}{\raisebox{-0.27ex}{\LARGE$\circ$}}
\newcommand{\Wnext}{\raisebox{-0.27ex}{\LARGE$\bullet$}}
%%\newcommand{\Next}{\raisebox{0.4ex}{\tiny$\bigcirc$}}
\newcommand{\Until}{\mathop{\mathcal{U}}}
\newcommand{\Release}{\mathop{\R}}
\newcommand{\Wuntil}{\mathop{\W}}
\newcommand{\true}{\mathit{true}}
%\newcommand{\false}{\mathit{false}}
\newcommand{\temptrue}{\mathit{temp\_true}}
\newcommand{\tempfalse}{\mathit{temp\_false}}
%%\newcommand{\ttrue}{\mathtt{true}}
%%\newcommand{\ffalse}{\mathtt{false}}
\newcommand{\Last}{\mathit{Last}}
\newcommand{\Ended}{\mathit{Ended}}
\newcommand{\length}{\mathit{length}}
\newcommand{\last}{\mathit{last}}
\newcommand{\nnf}{\mathit{nnf}}
\newcommand{\CL}{\mathit{CL}}
\newcommand{\MU}[2]{\mu #1.#2}
\newcommand{\NU}[2]{\nu #1.#2}
\newcommand{\BOX}[1]{ [#1]}
\newcommand{\DIAM}[1]{\langle #1 \rangle}

\endinput
%%% Local Variables:
%%% mode: latex
%%% TeX-master: "main"
%%% save-place: t
%%% End: 

\usepackage{lineno,hyperref}
\modulolinenumbers[5]

\journal{Journal of \LaTeX\ Templates}

%%%%%%%%%%%%%%%%%%%%%%%
%% Elsevier bibliography styles
%%%%%%%%%%%%%%%%%%%%%%%
%% To change the style, put a % in front of the second line of the current style and
%% remove the % from the second line of the style you would like to use.
%%%%%%%%%%%%%%%%%%%%%%%

%% Numbered
%\bibliographystyle{model1-num-names}

%% Numbered without titles
%\bibliographystyle{model1a-num-names}

%% Harvard
%\bibliographystyle{model2-names.bst}\biboptions{authoryear}

%% Vancouver numbered
%\usepackage{numcompress}\bibliographystyle{model3-num-names}

%% Vancouver name/year
%\usepackage{numcompress}\bibliographystyle{model4-names}\biboptions{authoryear}

%% APA style
%\bibliographystyle{model5-names}\biboptions{authoryear}

%% AMA style
%\usepackage{numcompress}\bibliographystyle{model6-num-names}

%% `Elsevier LaTeX' style
\bibliographystyle{elsarticle-num}
%%%%%%%%%%%%%%%%%%%%%%%

\begin{document}

\begin{frontmatter}

\title{Declarative Process Mining with SAT}
%\tnotetext[mytitlenote]{Fully documented templates are available in the elsarticle package on \href{http://www.ctan.org/tex-archive/macros/latex/contrib/elsarticle}{CTAN}.}

%% Group authors per affiliation:
\author[1]{Fabrizio Maria Maggi}
%\address{Free University of Bozen-Bolzano}
\ead{maggi@inf.unibz.it}
\address[1]{Free University of Bozen-Bolzano}

\author[2]{Andrea Marrella}
%\address{Sapienza University of Rome}
\ead{marrella@diag.uniroma1.it}

\author[2]{Fabio Patrizi}
%\address{Sapienza University of Rome}
\ead{patrizi@diag.uniroma1.it}

\address[2]{Sapienza University of Rome}


%\author[3]{Vasyl Skydanienko}
%\address{University of Tartu}
%\ead{skydanienko@ut.ee}
%\address[3]{University of Tartu}

%\fntext[myfootnote]{Since 1880.}
%
%% or include affiliations in footnotes:
%\author[mymainaddress,mysecondaryaddress]{Elsevier Inc}
%\ead[url]{www.elsevier.com}
%
%\author[mysecondaryaddress]{Global Customer Service\corref{mycorrespondingauthor}}
%\cortext[mycorrespondingauthor]{Corresponding author}

%


%\address[mysecondaryaddress]{360 Park Avenue South, New York}

\begin{abstract}
Process Mining is a family of techniques for analyzing business process execution data recorded in event logs. Process models can be obtained as output of automated process discovery techniques or can be used as input of techniques for conformance checking or model enhancement. In Declarative Process Mining, process models are represented as sets of temporal constraints (instead of procedural descriptions where all control-flow details are explicitly modeled). An open research direction in Declarative Process Mining is whether multi-perspective specifications can be supported, i.e., specifications that not only describe the process behavior from the control-flow point of view, but also from other perspectives like data or time. In this paper, we address this question by considering SAT as a solving technology for a number of classical problems in Declarative Process Mining, namely log generation, conformance checking and temporal query checking. To do so, we first express each problem as a suitable FO theory whose bounded models represent solutions to the problem, and then find a bounded model of such theory by compilation into SAT.
%This is actually done by resorting to Alloy, a tool for finding bounded models of FO theories via compilation into SAT.
%Notably, the Alloy output, i.e., the SAT instance, can be provided as input to any SAT solver, thus making any performance optimization in SAT directly available to Declarative Process Mining.
%By using Alloy we obtain a full-fledged tool that can solve all the existing problems arising in multi-perspective Declarative Process Mining. We report an experimental evaluation to test the feasibility of the approach.


%\keywords{Process Mining  \and SAT \and Alloy \and Multi-Perspective Models \and Declarative Models.}
\end{abstract}

\begin{keyword}
Process Mining  \sep SAT \sep Alloy \sep Multi-Perspective Models \sep Declarative Models
\end{keyword}

\end{frontmatter}

\linenumbers

%\begin{abstract}
Process Mining is a family of techniques for analyzing business process execution data recorded in event logs. Process models can be obtained as output of automated process discovery techniques or can be used as input of techniques for conformance checking or model enhancement. In Declarative Process Mining, process models are represented as sets of temporal constraints (instead of procedural descriptions where all control-flow details are explicitly modeled). An open research direction in Declarative Process Mining is whether multi-perspective specifications can be supported, i.e., specifications that not only describe the process behavior from the control-flow point of view, but also from other perspectives like data or time. In this paper, we address this question by considering SAT as a solving technology for a number of classical problems in Declarative Process Mining, namely log generation, conformance checking and temporal query checking.
%We show that these problems can be effectively solved via reduction to SAT.
To do so, we first express each problem as a suitable FO theory whose bounded models represent solutions to the problem, and then find a bounded model of such theory by compilation into SAT.
%This is actually done by resorting to Alloy, a tool for finding bounded models of FO theories via compilation into SAT.
%Notably, the Alloy output, i.e., the SAT instance, can be provided as input to any SAT solver, thus making any performance optimization in SAT directly available to Declarative Process Mining.
%By using Alloy we obtain a full-fledged tool that can solve all the existing problems arising in multi-perspective Declarative Process Mining. We report an experimental evaluation to test the feasibility of the approach.


%\keywords{Process Mining  \and SAT \and Alloy \and Multi-Perspective Models \and Declarative Models.}
\end{abstract}

%\newcommand{\condec}{Declare\xspace}

\section{Introduction}
%{\bf {Process Mining, trace logs, events, Problems in PM, Declare, What we do:
%we compile all problems into Alloy to solve them using SAT, splittare anche in
%preliminiaries}}

%We use Alloy for multi-perspective DECLARE models:
%\begin{itemize}
%	\item Event-log generation
%	\item Conformance checking
%	\item Query checking
%\end{itemize}


%Why?

%\begin{itemize}
%	\item 	 Because there are no other tools available for 1 and 3 (remember
%we are in the presence of data). For 2, we compare against ProM (DECLARE
%analyzer)
%	\item because we want to investigate the effectiveness of SAT solvers for
%the tasks above
%\end{itemize}

%- We can use Alloy because we are in a bounded setting (because the prefix is
%finite and we can perform an abstraction step).

%- We consider only $=$ and $\neq$ (but let see if we can also have total
%order and/or timestamps, which require sum only, otherwise leave for
%future work).

%- We have to describe the abstraction step

%- (Vasyl): we can also check infinite traces, i.e., if a model is satisfiable.

%- MAX 15 pages.

%In recent years, declarative languages have been proposed to model business rules and loosely-structured business processes, mediating between support and flexibility. A notable example are constraint-based approaches. These approaches enjoy declarativeness by supporting the modeler in the elicitation of the (minimal) set of behavioral \emph{constraints} that must be respected to correctly execute the process, without stating how the involved stakeholders
%must behave to satisfy them. Acceptable courses of execution are not explicitly enumerated in a pre-defined way; rather, they are implicitly derived as the execution traces that comply with all the modeled constraints, thus enjoying \emph{flexibility by design}. In this work, we focus on the
%\condec\ constraint-based framework \cite{Pesic2007:DECLARE}. \condec provides a graphical, declarative language for the
%specification of activities and constraints: activities model atomic units of work, while constraints impose expectations about the (non) execution of %activities. Constraints range from classical sequence patterns to loose relations, prohibitions and cardinality constraints.


In the last years, declarative process modeling languages have been proposed to represent knowledge-intensive business processes \cite{DBLP:journals/jodsn/CiccioM015}. These languages are based on constraint-based modeling approaches able to balance flexibility and support. The modeler can elicit the (minimal) set of \emph{constraints} that must be satisfied to execute the process in the correct way, without explicitly stating how the involved process participants must execute the process to satisfy them. 
%In particular, the acceptable sequences of activities are not explicitly enumerated; instead, they are implicitly derived as the execution traces that comply with all the modeled constraints. 
In this work, we focus on the \condec\ constraint-based modeling language \cite{Pesic2007:DECLARE}. \condec provides a graphical language for modeling business processes in terms of activities (atomic units of work) and constraints over them, imposing expectations about their occurrence. Constraints range from standard sequential patterns to loose relations, prohibitions and cardinality constraints.


%One of the key advantages of \condec\ is that its semantics can be characterized in different logic-based approaches, enabling a wide
%range of reasoning and verification capabilities. In its original shape, \condec\ is mainly exploited to constrain control-flow aspects, disregarding other important aspects such as event data and corresponding conditions, as well as temporal constraints, which are all key requirements when dealing with real-world case studies \cite{DBLP:journals/internet/0002SSA12,MMvdA,Ly11}. Previous works have tackled these limitations by incorporating these additional perspectives into the constraint language \emph{MP-Declare} (\emph{Multi-Perspective Declare}) \cite{DBLP:journals/eswa/BurattinMS16}. Based on this semantics, some techniques have been proposed for the analysis of event logs using MP-Declare constraints. Nevertheless, very few techniques have been proposed in the wide spectrum of problems that arise in this context and a theoretically founded, fully integrated tool for Multi-Perspective Declarative Process Mining is still missing.

The semantics of \condec\ has been defined using different logic-based approaches that enable a wide range of reasoning and verification capabilities. Originally, \condec\ was meant to constrain control-flow aspects, without considering other crucial aspects that come into play when modeling a business process such as event data and corresponding conditions, as well as temporal constraints \cite{DBLP:journals/internet/0002SSA12,MMvdA,Ly11}. Previous works have tackled these limitations by incorporating these additional perspectives into the declarative process modeling language \emph{MP-Declare} (\emph{Multi-Perspective Declare}) \cite{DBLP:journals/eswa/BurattinMS16}. Based on this semantics, some techniques have been proposed for the analysis of event logs using MP-Declare constraints. Nevertheless, very few techniques have been proposed in the wide spectrum of problems that arise in this context and a theoretically founded, fully integrated tool for Multi-Perspective Declarative Process Mining is still missing.


In this paper, we aim at closing this gap by considering SAT as a solving technology for a number of problems in Multi-Perspective Declarative Process Mining thus showing the potential of this technology when solving Declarative Process Mining problems involving multi-perspective process specifications.
%We show that these problems can be effectively solved via reduction to SAT.
To do so, we first express each problem as a suitable FO theory whose bounded models represent solutions to the problem, and then find a bounded model of such theory by compilation into SAT. This is actually done by resorting to the Alloy tool.
%for finding bounded models of FO theories by compilation into SAT.
Notably, the Alloy output, i.e., the SAT instance, can be provided as input to any SAT solver, thus making any performance optimization in SAT directly available to Declarative Process Mining.
%By using Alloy we obtain a full-fledged tool that can solve all the existing problems arising in multi-perspective Declarative Process Mining.
At the end of the paper, we report an experimental evaluation to test the feasibility of the approach.
%we aim at closing this gap by proposing a multi-perspective approach based on Declare that allows for defining multi-perspective constraints jointly considering data, temporal, and control-flow perspectives. To this aim,
%we formally define Multi-Perspective Declare (MP-Declare), an augmented version of Declare that, being based on Metric First-Order Linear Temporal Logic semantics, allows for the definition of activation, correlation, and time conditions to build constraints over traces.

%A nice feature of MP-Declare is that, by construction, it allows the user to efficiently perform conformance checking over event logs.
%In fact, we show that it is possible to define a conformance checking algorithmic framework for MP-Declare that is linear in the number of traces, constraints, and in the number of events of each trace.
%Conformance checking for a specific MP-Declare template is obtained via definition of  template-dependent procedures within the framework, whose time complexity depends on the actual template. Overall, however, the time complexity is upper bounded in the worst case by a quadratic function.

%We assess the validity of the proposed approach both on artificial and real-life event logs. Controlled artificial data, involving logs containing up to 5 million events, are used to prove the scalability of the proposed approach, while real-life event logs generated by three real business processes are used to demonstrate the expressivity and flexibility of constraints defined via MP-Declare.


%The rest of the paper is structured as follows. In Section \ref{sec:related}, we discuss the related work. Section \ref{sec:preliminaries} provides some background notions needed to understand the main contribution of the paper. In Section \ref{sec:semantics}, we introduce the semantics of Multi-Perspective Declare based on Metric First-Order Linear Temporal Logic. In Section \ref{sec:algorithms}, we discuss the proposed conformance checking algorithms. Section \ref{sec:benchmarks} describes the implementation of the approach and provides a benchmark analysis. Section \ref{sec:casestudy} discusses three case studies on real-life datasets. Section~\ref{sec:conclusion} concludes the paper and spells out directions for future work.

\section{Preliminaries}
\label{sec:semantics}

In this section, we first provide some basic notions about Declare, then we give some background about Process Mining.

\subsection{The Declare Modeling Language}
\label{sec:declare}
The comparison between the use of procedural and declarative process modeling languages to model a business process has been largely investigated in the last years \cite{DBLP:conf/caise/ZugalPW11,DBLP:conf/bpm/PichlerWZPMR11,DBLP:conf/bpm/ReijersSS13}. The results of these studies have shown that procedural models are more suitable to support the execution of business processes in stable and predictable environments characterized by predefined procedures. In contrast, declarative process modeling languages like Declare work under an ``open world'' assumption and provide process participants with a set of rules that should not be violated. This approach is suitable to model unstable and unpredictable processes since process participants have the flexibility to follow any path that does not violate the modeled rules.
%Compared to the procedural models, a declarative process model defines a set of constraints that should be followed during the process execution. In this way, a declarative model implicitly defines the control-flow as all the possible paths that do not violate any of the given constraints. In this way declarative models, differently from the procedural ones, enjoy flexibility.

Declare has been first presented in~\cite{Pesic2007:DECLARE}.
%The three main components it consists of are
%\begin{enumerate}
%\item	Designer (modeling tool), that is used for system settings and process model design
%\item	Framework (process enactment tool), which is also used for communication with other programs and changing models at run-time
%\item	Worklist (process execution tool), which is meant for users to execute traces and see recommendations
%\end{enumerate}
%Different application domains may require a different set of relation types (constraint templates). Therefore, Declare facilitates the definition of sets of constraint templates.
A Declare model consists of a set of constraints applied to (atomic) activities.
%Constraints, in turn, are based on templates. Templates are abstract parameterized patterns, and constraints are their concrete instantiations on real activities.
%Templates have a user-friendly graphical representation understandable to the user.
The semantics of Declare constraints can be can be expressed using LTL for finite traces \cite{DBLP:conf/ijcai/GiacomoV13}. %, making them verifiable and executable.
%Each constraint inherits the graphical representation and semantics from its template.
%The major benefit of using templates is that analysts do not have to be aware of the underlying logic-based formalization to understand the models. They work with the graphical representation of templates, while the underlying formulas remain hidden.
%Table~\ref{tab:dec} reports the main Declare templates, their graphical representation and a textual description.
%The reader can refer to \cite{Pesic2007:DECLARE} for a full description of the language.



%Each constraint template has three attributes:
%\begin{enumerate}
%\item	A unique name
%\item	Semantics specified in LTL
%\item	Graphical representation (for visual representation)
%\end{enumerate}

%Here, we indicate template parameters with capital letters (see Table~\ref{tab:dec})
%
%
%
%
%
%and real activities in their
%instantiations with lower case letters (e.g., constraint \Resp{\taska}{\taskb}). A trace is a sequence of events like $\langle \taska, \taska, \taskb, \taskc \rangle$.
%There is a total of 19 different templates in the Declare modeling language: 5 of these are existence templates which involve only one event; 11 are relation templates which describe a dependency between two events and 3 are negative relation templates~\cite{maggietal2011}.
%Declare templates can be grouped in three main categories: \emph{existence} templates (first 4 rows of the table), which involve only one event; \emph{(mutual) relation} templates (rows from 5 to 15), which describe a dependency between two events; and \emph{negative relation} templates (last 3 rows), which describe a negative dependency between two events.

%\input{tables/declare-constraints}

%\begin{table}[t!]
%  \centering
%  %!TEX root = ../main.tex
\renewcommand{\arraystretch}{1.2}
\begin{scriptsize}
\begin{tabular}{ l p{5cm} c}
\toprule
\textbf{Constraint} & \textbf{Explanation} & \textbf{Notation} \\  % bcaaccbbbaba
\midrule
\multicolumn{3}{l}{Existence constraints}\\
\midrule
$\Exi{n}{\paramx}$ & % [^$2]*($2[^$2]*){$1,}[^$2]*
Activity $\paramx$ occurs at least $n$ times &
% & {\paramx} & --<w
\begin{tikzpicture}[baseline=(current bounding box.center)]\node[DECLARE.task,DECLARE.existcon=$n..\ast$]{\paramx};\end{tikzpicture}
\\
$\Abse{m+1}{\paramx}$ & % [^$2]*($2[^$2]*){0,$max}[^$2]*
$\paramx$ occurs at most $m$ times &
% & {\paramx} & --
\begin{tikzpicture}[baseline=(current bounding box.center)]\node[DECLARE.task,DECLARE.existcon=$0..m$]{\paramx};\end{tikzpicture}
\\
$\Ini{\paramx}$ &
{\paramx} is the \emph{first} to occur &
%\taskize{\textbf{\uline{a}}ccbbbaba}
% & {\paramx} & --
\begin{tikzpicture}[baseline=(current bounding box.center)]\node[DECLARE.task,DECLARE.existcon=Init]{\paramx};\end{tikzpicture}
\\
$\End{\paramx}$ &
{\paramx} is the \emph{last} to occur &
%\taskize{bcaaccbbbab\textbf{\uline{a}}}
% & {\paramx} & --
\begin{tikzpicture}[baseline=(current bounding box.center)]\node[DECLARE.task,DECLARE.existcon=End]{\paramx};\end{tikzpicture}
\\
\midrule
\multicolumn{3}{l}{Relation constraints}\\
\midrule
$\ResEx{\paramx}{\paramy}$ &
If {\paramx} occurs, then {\paramy} occurs as well &
%\taskize{\textbf{b}c\textbf{\uline{a}}\uline{a}ccbbb\uline{a}b\uline{a}}
% & {\paramx} & \paramy
\begin{tikzpicture}[baseline=(current bounding box.center),node distance=8*\DTZU]
 \node[DECLARE.task,] (1) {$\paramx$};
 \node[DECLARE.task,right of=1] (2) {$\paramy$};
   
 \path (1) edge [DECLARE.resex] node {} (2);
\end{tikzpicture}
\\
$\Resp{\paramx}{\paramy}$ &
If {\paramx} occurs, then {\paramy} occurs after {\paramx} &
%\taskize{bc\textbf{\uline{a}}\uline{a}cc\textbf{b}bb\textbf{\uline{a}}\textbf{b}}
% & {\paramx} & \paramy
\begin{tikzpicture}[baseline=(current bounding box.center),node distance=8*\DTZU]
 \node[DECLARE.task,] (1) {$\paramx$};
 \node[DECLARE.task,right of=1] (2) {$\paramy$};
   
 \path (1) edge [DECLARE.resp] node {} (2);
\end{tikzpicture}
\\
$\AltResp{\paramx}{\paramy}$ &
Each time {\paramx} occurs, then {\paramy} occurs afterwards, before {\paramx} recurs &
%\taskize{bc\textbf{\uline{a}}cc\textbf{b}bb\textbf{\uline{a}}\textbf{b}}
% & {\paramx} & \paramy
\begin{tikzpicture}[baseline=(current bounding box.center),node distance=8*\DTZU]
 \node[DECLARE.task,] (1) {$\paramx$};
 \node[DECLARE.task,right of=1] (2) {$\paramy$};
   
 \path (1) edge [DECLARE.alt.resp] node {} (2);
\end{tikzpicture}
\\
$\ChaResp{\paramx}{\paramy}$ &
Each time {\paramx} occurs, then {\paramy} occurs immediately after &
%\taskize{bc\textbf{\uline{a}}\textbf{b}bb\textbf{\uline{a}}\textbf{b}}
% & {\paramx} & \paramy
\begin{tikzpicture}[baseline=(current bounding box.center),node distance=8*\DTZU]
 \node[DECLARE.task,] (1) {$\paramx$};
 \node[DECLARE.task,right of=1] (2) {$\paramy$};
   
 \path (1) edge [DECLARE.chn.resp] node {} (2);
\end{tikzpicture}
\\
$\Prec{\paramx}{\paramy}$ &
{\paramy} occurs only if preceded by {\paramx} &
%\taskize{c\textbf{a}acc\textbf{\uline{b}}\uline{b}\uline{b}a\uline{b}a}
% & {\paramy} & \paramx
\begin{tikzpicture}[baseline=(current bounding box.center),node distance=8*\DTZU]
 \node[DECLARE.task,] (1) {$\paramx$};
 \node[DECLARE.task,right of=1] (2) {$\paramy$};
   
 \path (1) edge [DECLARE.prec] node {} (2);
\end{tikzpicture}
\\
$\AltPrec{\paramx}{\paramy}$ &
Each time {\paramy} occurs, it is preceded by {\paramx} and no other {\paramy} can recur in between &
%\taskize{ca\textbf{a}cc\textbf{\uline{b}}\textbf{a}\textbf{\uline{b}}a}
% & {\paramy} & \paramx
\begin{tikzpicture}[baseline=(current bounding box.center),node distance=8*\DTZU]
 \node[DECLARE.task,] (1) {$\paramx$};
 \node[DECLARE.task,right of=1] (2) {$\paramy$};
   
 \path (1) edge [DECLARE.alt.prec] node {} (2);
\end{tikzpicture}
\\
$\ChaPrec{\paramx}{\paramy}$ &
Each time {\paramy} occurs, then {\paramx} occurs immediately before &
%\taskize{c\textbf{a}\textbf{\uline{b}}\textbf{a}\textbf{\uline{b}}a}
% & {\paramy} & \paramx
\begin{tikzpicture}[baseline=(current bounding box.center),node distance=8*\DTZU]
 \node[DECLARE.task,] (1) {$\paramx$};
 \node[DECLARE.task,right of=1] (2) {$\paramy$};
   
 \path (1) edge [DECLARE.chn.prec] node {} (2);
\end{tikzpicture}
\\
\midrule
\multicolumn{3}{l}{Mutual relation constraints}\\
\midrule
$\CoExi{\paramx}{\paramy}$ &
If {\paramy} occurs, then {\paramx} occurs, and viceversa &
%\taskize{\textbf{\uline{b}}c\textbf{\uline{a}}cc\uline{bbbaba}}
% & \paramx, {\paramy} & \paramy, \paramx
\begin{tikzpicture}[baseline=(current bounding box.center),node distance=8*\DTZU]
 \node[DECLARE.task,] (1) {$\paramx$};
 \node[DECLARE.task,right of=1] (2) {$\paramy$};
   
 \path (1) edge [DECLARE.coex] node {} (2);
\end{tikzpicture}
\\
$\Succ{\paramx}{\paramy}$ &
{\paramx} occurs if and only if it is followed by {\paramy} &
%\taskize{c\textbf{\uline{a}}\uline{a}cc\textbf{\uline{b}}\uline{bbab}}
% & \paramx, {\paramy} & \paramy, \paramx
\begin{tikzpicture}[baseline=(current bounding box.center),node distance=8*\DTZU]
 \node[DECLARE.task,] (1) {$\paramx$};
 \node[DECLARE.task,right of=1] (2) {$\paramy$};
   
 \path (1) edge [DECLARE.succ] node {} (2);
\end{tikzpicture}
\\
$\AltSucc{\paramx}{\paramy}$ &
{\paramx} and {\paramy} occur if and only if the latter follows the former, and they alternate each other &
%\taskize{c\textbf{\uline{a}}cc\textbf{\uline{b}}\textbf{\uline{a}}\textbf{\uline{b}}}
% & \paramx, {\paramy} & \paramy, \paramx
\begin{tikzpicture}[baseline=(current bounding box.center),node distance=8*\DTZU]
 \node[DECLARE.task,] (1) {$\paramx$};
 \node[DECLARE.task,right of=1] (2) {$\paramy$};
   
 \path (1) edge [DECLARE.alt.succ] node {} (2);
\end{tikzpicture}
\\
$\ChaSucc{\paramx}{\paramy}$ &
{\paramx} and {\paramy} occur if and only if the latter immediately follows the former &
%\taskize{c\textbf{\uline{a}}\textbf{\uline{b}}\textbf{\uline{a}}\textbf{\uline{b}}}
% & \paramx, {\paramy} & \paramy, \paramx
\begin{tikzpicture}[baseline=(current bounding box.center),node distance=8*\DTZU]
 \node[DECLARE.task,] (1) {$\paramx$};
 \node[DECLARE.task,right of=1] (2) {$\paramy$};
   
 \path (1) edge [DECLARE.chn.succ] node {} (2);
\end{tikzpicture}
\\
\midrule
\multicolumn{3}{l}{Negative relation constraints}\\
\midrule
$\NotCoExi{\paramx}{\paramy}$ &
{\paramx} and {\paramy} never occur together &
%\taskize{c\uline{aa}cc\uline{a}}
% & \paramx, {\paramy} & \paramy, \paramx
\begin{tikzpicture}[baseline=(current bounding box.center),node distance=8*\DTZU]
 \node[DECLARE.task,] (1) {$\paramx$};
 \node[DECLARE.task,right of=1] (2) {$\paramy$};
   
 \path (1) edge [DECLARE.coex,DECLARE.neg] node {} (2);
\end{tikzpicture}
\\
$\NotSucc{\paramx}{\paramy}$ &
{\paramx} can never occur before {\paramy} &
%\taskize{bc\textbf{\uline{a}}\uline{a}cc\uline{a}}
% & \paramx, {\paramy} & \paramy, \paramx
\begin{tikzpicture}[baseline=(current bounding box.center),node distance=8*\DTZU]
 \node[DECLARE.task,] (1) {$\paramx$};
 \node[DECLARE.task,right of=1] (2) {$\paramy$};
   
 \path (1) edge [DECLARE.succ,DECLARE.neg] node {} (2);
\end{tikzpicture}
\\
$\NotChaSucc{\paramx}{\paramy}$ &
{\paramx} and {\paramy} occur if and only if the latter does not immediately follows the former &
%\taskize{bc\uline{a}\textbf{\uline{a}}cc\textbf{b}bbb\uline{a}}
% & \paramx, {\paramy} & \paramy, \paramx
\begin{tikzpicture}[baseline=(current bounding box.center),node distance=8*\DTZU]
 \node[DECLARE.task,] (1) {$\paramx$};
 \node[DECLARE.task,right of=1] (2) {$\paramy$};
   
 \path (1) edge [DECLARE.chn.succ,DECLARE.neg] node {} (2);
\end{tikzpicture}
\\
\bottomrule
\end{tabular}
\end{scriptsize} 
%  \caption{Declare templates.}
%  \vspace{-5mm}
%  \label{tab:dec}
%\end{table}

%The first four rows of the table report the \textbf{existence templates}:
%\begin{itemize}
%\item $existence(n, A)$, which specifies that $A$ should occur at least $n$ times in a trace
%\item	$absence(n +1, A)$, which specifies that $A$ should occur at most $n$ times
%\item	$exactly(n, A)$ which specifies that $A$ should occur exactly $n$ times
%\item	$init(A)$ which specifies that each trace should start with event $A$
%\end{itemize}

%The next rows report the \emph{relational templates}:
%\begin{itemize}
%\item	$responded$ $existence(A, B)$ which specifies that if event $A$ occurs, event $B$ should also occur
%\item	$co-existence(A, B)$ which specifies that if one of the events $A$ or $B$ occurs, the other one should also occur
%\item	$response(A, B)$ which specifies if event A occurs, event B should eventually occur after A
%\item	$precedence(A, B)$ which specifies that event B should occur only if event A has occurred before
%\item	$succession(A, B)$ which requires both precedence and response algorithms to hold between events A and B
%\item	$alternate$ $response(A, B)$ which is the same as response but allows no repetitions of these events in between
%\item	$alternate$ $precedence(A, B)$ which is the same as precedence but allows no repetitions of these events in between
%\item	$alternate$ $succession(A, B)$ which is the same as succession but allows no repetitions of these events in between
%\item	$chain$ $response(A, B)$ which is the same as response but the events must happen one after another
%\item	$chain$ $precedence(A, B)$ which is the same as precedence but the events must happen one after another
%\item	$chain$ $succession(A, B)$ which is the same as succession but the events must happen one after another
%\end{itemize}

%Finally, the last three rows report the \emph{negative relation templates}:
%\begin{itemize}
%\item	$not$ $co-existence(A, B)$ which specifies that A and B cannot occur together in the same process
%\item	$not$ $succession(A, B)$ which specifies that that any occurrence of A cannot be eventually followed by B
%\item	$not$ $chain$ $succession(A, B)$ which specifies that A cannot be directly followed by B
%\end{itemize}

One example of a Declare constraint is the \emph{response} constraint \Resp{\taska}{\taskb}. This constraint indicates that if $\taska$ {\it occurs}, $\taskb$ must eventually {\it follow}.
Constraints of type \emph{alternate} express stronger types of relations specifying that activities must alternate without repetitions in between.
%
Even stronger ordering relations are specified by constraints of type \emph{chain}, which state that activities must occur one immediately after the other.
Other constraints refer to the cardinality of an activity, e.g., $\taska$ has to occur at least once or at most once in a trace. There are also constraints that force activities to be mutually exclusive or not to occur one after the other in a trace.
%The reader can refer to \cite{Pesic2007:DECLARE} for a full description of the language.

Considering its semantics, the response constraint \Resp{\taska}{\taskb} is satisfied for traces
$\langle \taskc, \taska, \taskb \rangle$, $\langle \taskb, \taskc, \taskb \rangle$ and $\langle \taska, \taskb, \taskb \rangle$.
It is not satisfied for $\langle
\taska, \taskc, \taskb, \taska \rangle$, because the second occurrence of $\taska$ is not followed by $\taskb$.
%Additionally, in trace $\langle \taskb, \taskb, \taskc, \taskd \rangle$, the
%considered response rule is satisfied in a trivial way because $a$ never occurs.
%In this case, the rule is considered as \emph{vacuously
%satisfied}~\cite{kupf:vacu03}.
An \emph{activation} of a constraint in a trace is an event that ``triggers'' the constraint by imposing an obligation on the occurrence of another event (the \emph{target}). For example, for \Resp{\taska}{\taskb}, $\taska$ is an activation, because its occurrence forces $\taskb$ (the target) to be executed eventually.


%
%Constraint \emph{co-existence(A,B)} requires that if one of the actions \emph{A} or \emph{B} occur, the other one must also occur.
%
%
%Declare also includes some negative constraints to explicitly forbid the execution of actions.
%
%The \emph{not co-existence(A,B)} constraint indicates that \emph{A} and \emph{B} can not occur together in the same interaction.
%
%According to the  constraint, any occurrence of $\taska$ can not be eventually followed by $\taskb$. Finally, the \emph{not chain succession(A,B)} constraint states that \emph{A} and \emph{B} can not occur one immediately after the other.


%\begin{figure*}[t!]
%	\centering
%		\includegraphics[width=1.0\textwidth]{figures/Fracturetreatmentprocess-DeclareModel.pdf}
%	\caption{The Declare model for a fracture treatment process.}
%	\label{fig:fracture}
	%\vspace{-.5cm}
%\end{figure*}



%An activation of a constraint can be a \emph{fulfillment} or a \emph{violation}
%for that constraint. When a trace is perfectly compliant with respect to a
%constraint, every activation of the constraint in the trace leads to a
%fulfilment.
%Consider, again, the response constraint \Resp{\taska}{\taskb}.
%In trace $\langle \taska, \taska, \taskb, \taskc \rangle$, the constraint is activated and fulfilled twice,
%whereas, in trace $\langle \taska, \taskb, \taskc, \taskb \rangle$, the same constraint is activated and fulfilled
%only once. On the other hand, when a trace is not compliant with respect to a
%constraint, %an activation of the constraint in the trace can lead to a
%fulfilment but also to a violation (
%at least one activation leads to a violation.
%).
%In trace $\langle
%\taska, \taskb, \taska, \taskc \rangle$, for example, the response constraint
%\Resp{\taska}{\taskb} is activated twice, but the first
%activation leads to a fulfilment (eventually $\taskb$ occurs), whereas the second
%activation leads to a violation ($\taskb$ does not occur subsequently).
%Finally, there exist cases in which the constraint is not activated at all.
%Consider, for instance, trace $\left\langle \taskb, \taskb, \taskc, \taskd\right\rangle$. The considered response constraint is satisfied in a trivial way because $\taska$ never occurs. In this case, we say that the constraint
%is \emph{vacuously satisfied}~\cite{kupf:vacu03}. In~\cite{DBLP:conf/bpm/MaggiMCM16,DBLP:journals/is/CiccioMMM18}, the authors introduce the notion of semantical vacuity detection according to which a constraint is non-vacuously satisfied in a trace when it is activated at least once in that trace.

%A Declare model containing multiple constraints is defined as a conjunction of the constraints. This means that the actions of users during execution must fulfill all constraints. Declare constraints can be either mandatory or optional.

%The system forces its users to follow all mandatory constraints in the model. In case of optional constraints users have the ability to decide whether to follow the corresponding rule or to violate it. Optional constraints are not enforced by the Declare system during execution. When a user is about to perform an action that violates an optional constraint, a warning about the violation is presented and the user can decide whether to continue with the action and violate the constraint or to cancel the action and follow the constraint. The text of the warning can be specified in the definition of the constraint.
%A model in Declare is mapped onto a set of LTL formulas. Based on these LTL formulas, automata are automatically generated to support enactment. Declare uses an algorithm that creates finite-words automata from LTL formulas of the constraints that are used. These automata are used both to drive the execution and to monitor the state of each constraint.

%Some compositions of constraints in process models may cause errors that lead to problems at run-time. Thus, Declare verifies process models against different types of errors and finds a minimal set of constraints that causes a specific error. All models can be verified against dead activities and conflicting constraints. A dead activity is an activity that can never be executed in the model. A set of constraints is conflicting if there exists no execution that would fulfill all constraints~\cite{pesicetal2007}.

%\subsection{\emph{Declare Miner} plugin in the ProM framework}

%\paragraph{Prom Framework.}
%ProM is a Process Mining framework that integrates the functionality of several existing Process Mining tools and provides many additional Process Mining plug-ins. It supports multiple formats and multiple languages such as Petri nets, EPCs, Social Networks and so on and so forth. The plug-ins can be used in several ways and combined to be applied in real-life situations~\cite{dongenetal2005}.

%\paragraph{\emph{Declare Miner} plugin.}
%The \emph{Declare Miner} plug-in for ProM allows users to discover a Declare model from a log by specifying a number of settings. There are two versions of the plug-in. The first one, the Declare Miner, requires a user-specified Declare language as input. The second one, the Declare Miner Default, uses a predefined Declare language and does not require any language as input. More information on the usage and set-up can be found at \url{http://www.win.tue.nl/declare/declare-miner/} .

%In order to remove irrelevant constraints from the output set, the authors apply vacuity detection techniques \cite{Kupferman.Vardi/STTT2003:Vacuitydetection}. Constraints are considered as vacuously satisfied when no trace in the log violates them, yet no trace shows the effect of their application either. A vacuously satisfied constraint is, e.g., that every request is eventually acknowledged, in a trace that does not contain requests.


  %There exist two variants of the plug-in: the classical one and the default version. While the former demands that users know a specific language for the definition of the Declare templates, the latter does not require knowl.
%The first one, the Declare Miner, requires a user-specified Declare language as input. The second one, the Declare Miner Default, uses a predefined Declare language and does not require any language as input. More information on the usage and set-up can be found at \url{http://www.win.tue.nl/declare/declare-miner/} .


Recently, an extension of the Declare language, Multi-Perspective Declare (MP-Declare), has been presented in \cite{DBLP:journals/eswa/BurattinMS16}. This extension not only captures control-flow constraints (like Declare), but takes also into consideration the data perspective. In particular, in MP-Declare, two types of data conditions can be defined: \emph{activation condition} and \emph{correlation condition}. In these conditions, data related to the activation of a constraint is expressed in the form $\cst{A.data}$, whereas data which relates to the target is expressed in the form $\cst{T.data}$.

When the activation of an MP-Declare constraint occurs, the constraint is activated only if the corresponding activation condition is satisfied. For example, constraint $\Abse{1}{\cst{SendInvoice}}$ without conditions indicates that an invoice should never be sent. However, if we add an activation condition to it as follows:\footnote{Activation condition and correlation condition are  specified in a constraint with the format [Activation] [Correlation]}
\begin{center}
	$\Abse{1}{\cst{SendInvoice}}$ $[\cst{A.amount} < 100]$
\end{center}
this means that an invoice cannot be sent only if the paid amount is lower than 100.


A correlation condition must be valid when the target of a constraint occurs and can involve data related to both activation and target. Therefore, this type of conditions can be used to specify correlations between two events. For example, the constraint:
\begin{center}
	$\Resp{\cst{ReceivePayment}}{\cst{SendInvoice}}$ \\
	$[\cst{A.amount} > 100][\cst{T.orderID} = \cst{A.orderID}]$
\end{center}
will only be activated if the paid amount is greater than 100. If this happens, then the payment should be followed by sending an invoice for the same order.




\subsection{Process Mining}
\label{sec:procmining}

%Process Mining~\cite{DBLP:books/sp/Aalst16} is still a rather young research discipline which lies between data mining and computational intelligence, and between process modeling and analysis.
Process Mining~\cite{DBLP:books/sp/Aalst16} is a field studying techniques for analyzing business processes by using event logs recorded by the systems supporting their execution. Some of these techniques are developed to build a process model representing the behavior of the process as recorded in an event log (automated process discovery). Other Process Mining techniques allow users to compare the real behavior of a process derived from an event log and the expected behavior of the process represented as a process model (conformance checking), or to extend/enhance a process model using the information retrieved from a log.
%Different Process Mining algorithms have been implemented in academic and commercial systems.
The increasing interest from industry in Process Mining is witnessed by the growing number of software vendors providing tools for Process Mining.\footnote{http://fluxicon.com/disco/}\footnote{https://www.celonis.com/intelligent-business-cloud}\footnote{https://www.signavio.com/products/process-intelligence/}\footnote{https://www.my-invenio.com/}\footnote{https://www.minit.io/}
%

As already mentioned, the main input of any Process Mining technique is an \emph{event log}. XES (eXtensible Event Stream) \cite{XES-standard-2013,Verbeek10} is the standard for representing event logs in XML format. This standard represents a log as a set of traces (i.e., process executions) and traces as sequences of events. Each event in a trace represents the execution of an activity in the process. An event must always record a timestamp (when the corresponding activity was executed) and can (optionally) record additional information such as the resource executing the activity, or other data elements related to the event.

%\begin{figure}[t]
%\centering
%\includegraphics[width=12cm]{figrefoverview}
%\caption{Overview of the Process Mining spectrum.}
%\label{figrefoverview}
%\end{figure}

%Fig.~\ref{figrefoverview} (based on

%The main guiding principles and upcoming challenges of %such a recent research field
%Process Mining have been reported in~\cite{IEEE2011:Manifesto}.
%Purpose of the principles is supporting in
%A list of guiding principles that aid in
%The former serve as a means for process miners to orient their investigations in real-world environments.
%avoiding mistakes that can be made when applying Process Mining in actual, real-life settings.
%The latter shed light on relevant open issues that are worth being tackled in the future.
% is presented in~\cite{aalstetal2012}. The guide consists of the following six principles:
%For instance:
%\begin{itemize}
%\item	Event data should be treated as first-class citizens, which means that the event logs are classified under different maturity levels ranging from excellent to poor or 5 stars to 1 start respectively. The higher the maturity level, the more reliable are the results when Process Mining is applied to the log;
%\item	Log extraction should be driven by questions because without concrete questions it is very difficult to extract reasonable information from event logs;
%\item	Concurrency, choice and other basic control-flow constructs should be supported to make sure the generated models are fitting and easy to understand;
%\item	Events should be related to model elements in order to support conformance checking and enhancement;
%\item	Models should be treated as purposeful abstractions of reality due to the fact that the results may be used by various; stakeholders in different situations. It also helps with producing understandable maps;
%\item	Process Mining should be a continuous process to cope with process changes.
%\end{itemize}

%Along-side key points and guide principles, there are challenges that need to be addressed due to the fact that Process Mining is, as mentioned before, a young discipline. These challenges are considered to be incomplete as, over time, new challenges may appear or existing challenges may disappear due to advances is Process Mining. Nevertheless, the challenges listed below are still relevant~\cite{aalstetal2012}.
%Although the challenges are considered to be incomplete as, over time, new challenges may appear or existing challenges may disappear due to advances is Process Mining, most of them are still relevant:
%\begin{itemize}
%\item	Finding, merging, and cleaning event data;
%\item	Dealing with complex event logs having diverse characteristics;
%\item	Creating representative benchmarks;
%\item	Dealing with concept drift;
%\item	Improving the representational bias used for process discovery;
%\item	Balancing between quality criteria such as fitness, simplicity, precision and generalization;
%\item	Cross-organizational mining;
%\item	Providing operational support;
%\item	Combining Process Mining with other types of analysis;
%\item	Improving usability for non-experts;
%\end{itemize}


%\subsection{Event Log}
%Process Mining techniques rely on the existence of \emph{event logs}, sets of process case executions in the form of sequences of events.
%In an event log:
%\begin{itemize}
%\item	Each event refers to an activity (a well-defined step in the process);
%\item	Each event refers to a trace (a trace);
%\item	Each event can have a performer also referred to as originator (the actor executing or initiating the activity);
%\item	Events have a timestamp and are totally ordered.
%\end{itemize}

%\begin{figure}
%	\centering
%	\includegraphics[width=\textwidth]{figures/XES_structure.png}
%	\caption{Process log XML format and transactional model}
%	\label{fig:xml_format}
%\end{figure}

%Since each information system has its own format for storing log files a generic XML format to store in a log information about process executions called MXML has been developed. The MXML format is presented in Fig.~\cref{fig:xml_format}. A log file typically contains information about events that took place in a system (\emph{AuditTrailEntry} in XML). Such events typically refer to a trace (\emph{ProcessInstance} in the XML) and a specific activity (\emph{WorkflowModelElement} in XML) within that trace. The originator and the timestamp are connected to the \emph{AuditTrailEntry} so they are always related to the event itself.
%Fig.~\cite{fig:} shows the transactional model for activity lifecycles. The transactional model is adopted by several commercial systems~\cite{dongenetal2005}.

%Even if MXML has been used as a standard for storing event logs for several years, based on practical experiences with applying MXML in about one hundred organizations, several problems and limitations related to MXML format have been discovered. One of the main problems is the semantics of additional attributes stored in the event log. In MXML, these are all treated as string values with a key and have no generally understood meaning. Another problem is the nomenclature used for different concepts. This is caused by the MXML's assumption that strictly structured process would be stored in this format.

%Since each information system has its own format for storing log files a generic, a standard called \emph{eXtensible Event Stream} (\emph{XES}), has been developed  for representing and storing event logs. The XES meta-model is shown in Fig.\cref{fig:xes}.
%To solve the problems encountered with MXML, and to create a standard that could also be used to store event logs from many different information systems directly, a new format has been developed called eXtensible Event Stream or XES. It enhances the MXML format in many ways as shown in~\cite{verbeeketal2010}.
%A log file (\emph{Log}) typically contains information about events that took place in a system. Such events typically refer to a trace (\emph{Trace}), representing a trace execution, and a specific activity (\emph{Event}) within that trace. The \emph{Log}, \emph{Trace} and \emph{Event} elements only define the structure of the document: they do not contain any information themselves. To store any data in the XES format, attributes are used. Every attribute has a string based key, a known type, and a value of that type. Possible types are string, date, integer, float and boolean. Note that attributes can have attributes themselves which can be used to provide more specific information~\cite{verbeeketal2010}.

%\begin{figure}
%	\centering
%		\includegraphics[width=.7\textwidth]{figures/XES_metamodel.png}
%	\caption{XES Metamodel}
%	\label{fig:xes}
%\end{figure}


%The advantages of XES are simplicity, flexibility, extensibility and expressivity. From these points of view it improves MXML.

One of the main open issues in Process Mining is to develop multi-perspective analysis techniques that fully leverage the heterogeneous information available in an event log as stated in~\cite{IEEE2011:Manifesto}. This is especially true in the context of Multi-Perspective Declarative Process Mining.
In this paper, we close this gap by using SAT as a solving technology for a number of classical problems in Multi-Perspective Declarative Process Mining, namely log generation, conformance checking and temporal query checking. In the following, we introduce these problems in detail.


\paragraph{Generation of Event Logs} As already mentioned, one branch of Process Mining is automated process discovery. Although there are several real life logs publicly available that can be used to test and evaluate process discovery techniques (see \url{https://data.4tu.nl/repository/collection:event_logs_real}), these logs can be incomplete and/or contain noise. Therefore. in order to evaluate process discovery techniques in a controlled environment and allow the researchers to fine tune the developed algorithms, tools for the generation of synthetic event logs with given predefined characteristics are needed.
%. However, these logs usually contain
%imperfections and have some missing information that can alter the evaluation of
%the discovery algorithms (e.g., they can be incomplete and/or contain noise).
%For this reason, a common approach adopted for testing process discovery
%algorithms is based on the use of synthetic logs created via simulation.
%Synthetic logs with different predefined characteristics and


%Starting from these needs, several model simulators and log generators have been
%developed and are available in the literature based on procedural \cite{andrea11,Jensen:2007,Hee,Ehrig}
%and declarative~\cite{Ackermann_et_al:2017,DiCiccio_et_al:2015,Schonig_et_al:2015} models.
%However, data support in the currently existing tools is not present or it does not cover the entire semantics of MP-Declare.
%This makes them not suitable for the evaluation of process discovery
%techniques based on multi-perspective declarative process models. Such techniques have
%recently attracted the attention of the Process Mining community and are useful
%to mine processes working in dynamic environments
%\cite{Pesic2006,Halpin,Pichler,diciccio.mecella/acmtmis2015:discoverydeclarativecontrol}.


\paragraph{Conformance Checking}  Conformance checking techniques allow users to compare the behavior of a process observed in an event log with a model of the same process representing its expected behavior \cite{wires-replay,costBasedReplayEDOC,anne_confcheck_is}. There are several contexts in which the analysis of the deviations of the execution of a business process from a prescriptive specification is critical, e.g., in process auditing~\cite{auditing} or risk analysis~\cite{riskAnalysis}.


\paragraph{Query Checking}
Query checking \cite{DBLP:conf/otm/RaimCMMM14} aims at discovering temporal properties of a trace that are not known a priori, but that have a predefined structure. In particular, the inputs of query checking are a trace and a \textit{query}, i.e., a temporal logic formula containing one or more \emph{placeholders}. The output is a set of temporal logic formulas, which derive from the input query by replacing all the placeholders with propositional formulas, which make the overall formula satisfied in the trace.
%The seminal work of \cite{Chan/CAV2000:TemporallogicQueries} considered Computation Tree Logic (CTL) \cite{Clarke.etal/ACMTPLS1986:CTL} as the language for expressing queries.
%Unlike {LTL}, adopting a linear time assumption, {CTL} is a branching-time logic: the evaluation of the formula is based on a tree-like structure, where different evolutions of the system over time are simultaneously considered in different branches of the computation model.
%Following the example of \cite{DBLP:conf/otm/RaimCMMM14} applying {LTL} query checking to discover (standard) declarative constraints, here we mine multi-perspective declarative processes out of event logs stored in databases through query checking.

\section{The Framework}

In this section, we introduce the technical framework used in the paper.
Essentially, we provide an axiomatization of traces in first-order logic
(FOL) interpreted over the naturals (and additional objects). We then show how MP-Declare constraints can be captured as FOL formulas and formalize the problems of interest for Declarative Process Mining.

\subsection{Log Traces as FO interpretations}
We model traces as suitable first-order (FO) interpretations.
A (FO)~\emph{signature} is a tuple $\sigma=\tup{\C,\P,\F}$, where:
$\C$ is a countable set of \emph{constant} symbols, $\P$ is a finite set of
\emph{predicate} symbols, and $\F$ is a finite set of \emph{function} symbols.
Predicate and function symbols have finite arity. When needed, we write $P/a$
(resp.~$f/a$) to indicate that a predicate (function) symbol has arity $a$.

An \emph{interpretation} $\iota$ of a signature $\sigma$ over a
(possibly infinite) \emph{universe} $\Delta$ is a pair
$\iota=\tup{\Delta,\cdot^\iota}$, where $\cdot^\iota$ is the
\emph{interpretation function}, i.e., a function associating:
	\begin{itemize}
		\item each constant symbol $c\in\C$ with an object $c^\iota\in\Delta$;
		\item each predicate symbol $P\in\P$ of arity $a$ with a relation
$P^\iota\subseteq \Delta^a$;
		\item each function symbol $f\in F$ of arity $a$ with a mapping
$f^\iota:\Delta^a\rightarrow\Delta$.
	\end{itemize}
%%

Given a FO vocabulary $\sigma$ and a numerable set $V$ of
variable symbols, formulas $\varphi$ of FOL over $\sigma$
respect the following syntax:
$$\varphi = P(\vec t)\mid \lnot \varphi\mid \varphi\land\varphi\mid \exists
x.\varphi, \mbox{ where:}$$
\begin{itemize}
 \item $P\in \P$ is a predicate symbol;
 \item $\vec t$ is a tuple
	$\tup{t_1,\ldots,t_a}$, with $a$ the arity of $P$,
	and each $t_i$ a \emph{term} from $\sigma$, i.e., a variable or a constant
symbol from $\sigma$, or a \emph{function term}, i.e., an expression of the
form $f(\vec t')$, with $f$ a function symbol from $\sigma$ and $\vec t'$ a
tuple of terms
from $\sigma$, of size equal to the arity of $f$.
\end{itemize}
%%
As standard, we define the following abbreviations:
$\varphi_1\lor \varphi_2\doteq \lnot(\lnot\varphi_1\land\lnot\varphi_2)$,
$\forall x.\varphi\doteq\lnot\exists x.\lnot\varphi$,
$\varphi_1 \rightarrow \varphi_2\doteq \lnot\varphi_1 \lor \varphi_2$,
and $\varphi_1 \leftrightarrow \varphi_2\doteq (\varphi_1 \rightarrow
\varphi_2) \land (\varphi_2 \rightarrow \varphi_1)$.
%%
The variable $x$ in $\exists x.\varphi$ (resp.~$\forall x.\varphi$) is
an (existentially, resp.~universally) quantified variable.
Variables in a FO
formula do not have to be quantified, in which case they are called
\emph{free variables}.
Formulas containing only quantified variables are called \emph{sentences}.
%%

FO formulas are interpreted over FO interpretations and
\emph{variable assignments}, i.e., total mappings $\nu:V\mapsto \Delta$,
for $\Delta$ the universe of the interpretation in question.
Formula interpretations require interpreting terms first.
Given an interpretation $\I=\tup{\Delta,\cdot^\I}$ of $\sigma$ and a term $t$
over $\sigma$, the
interpretation of $t$ under $\nu$ and $\I$ is the object
$t^\I_\nu\in \Delta$ such that:
\begin{itemize}
 \item $t^\I_\nu = \nu(t)$, if $t$ is a variable symbol;
 \item $t^\I_\nu = c^\I$, if $t$ is a constant symbol;
 \item $t^\I_\nu = f^\I({t_1}^\I_\nu,\ldots,{t_a}^\I_\nu)$, for $a$
the arity of $f$, if $t=f(t_1,\ldots,t_a)$.
\end{itemize}
%%
In this paper, we assume that every interpretation $\iota$ of $\sigma$ is
s.t.~$\Delta=\C$ and
for every $c\in\C$, $\iota(c)=c$.
In a word, we have \emph{nominals}, i.e., we can always refer to an
element of the interpretation domain through a constant (notice that, again,
this cannot be expressed in FOL, thus needs to be assumed).


A FO interpretation
$\I$ is said to \emph{satisfy}
a FO formula $\varphi$ under assignment $\nu$,
written $\I, \nu\models\varphi$, iff one of the following holds:
%%
\begin{itemize}
	\item $\varphi=P(t_1,\ldots,t_a)$ and
	$\tup{{t_1}^\I_\nu,\ldots,{t_a}^\I_\nu}\in P^\I$;
	\item $\varphi=\lnot \phi$ and $\I,\nu\not\models\phi$;
	\item $\varphi=\phi_1\land \phi_2$ and $\I,\nu\models \phi_i$, for $i=1,2$;
	\item $\varphi=\exists x.\phi$ and there exists $o\in\Delta$ such that
$\I,\nu'\models \phi$, with $\nu'(x)=o$ and $\nu'(v)=\nu(v)$, for $v\neq x$.
\end{itemize}
%%
% % Notice that, in order to check satisfaction of $\varphi$, all terms
%%occurring
% % in it must be replaced by their interpretation.
% %
An interpretation $\I$ \emph{satisfies}
a formula $\varphi$, written $\I\models \varphi$, iff $\I,\nu\models\varphi$
for all assignments $\nu$.
%%
A \emph{FO theory} $\Gamma$ is a possibly infinite set of FO sentences. Here,
we focus on finite theories.
An interpretation is said to
satisfy a theory $\Gamma$ if $\I\models\varphi$ for all $\varphi\in\Gamma$.
When
this is the case, $\I$ is said to be a \emph{model} of $\Gamma$.
%%


We model traces as models of a particular theory.
%%
A signature
$\sigma=\tup{\C,\P,\F}$ is said to be a \emph{trace signature} if (we use, as
standard, the infix notation for $<$, $\leq$, etc.):
\begin{itemize}
	\item	
		$\C$ includes a finite set of integers, including 0,
		plus a \emph{finite number}
		of additional constants;		
	\item $\P=\set{N/1,Seq/3,AttV/3,<,\leq,=,\geq,>}$;		
	\item $\F=\set{+,-}$.
	
\end{itemize}
%%
We consider only those interpretations where $N$ is interpreted as a
\emph{finite subset} of $\mathbb{N}_0$ and where all relational operators
and functions on integers are interpreted as the restriction on $N$ of
the standard corresponding operators on $\mathbb{N}_0$. Notice that this cannot
be expressed in FOL as, in particular, there is no axiomatization for
integers (thus the requirements need to be assumed). As a consequence of these
requirements, we deal only with \emph{finite models}.

Given a trace signature, we next define a \emph{trace theory} $\Gamma$.
Essentially, the intuition is that the models of $\Gamma$ are s.t.: $Seq$
models a finite sequence, indexed from $0$ to some $n$, of activities with
non-decreasing (integer) timestamps --we call the indices of the
sequence \emph{events};
$AttV$ is a (possibly partial) functional relation mapping
events and attribute names into objects.
%%
Formally, $\Gamma$ is the theory including the following sentences:
\begin{itemize}
	\item $Seq$ is a non-empty indexed sequence of activities and
non-decreasing timestamps:	
		\begin{itemize}
			\item
			indexes and timestamps are integers: $\forall
i,a,t.Seq(i,a,t)\rightarrow N(i)\land N(t)$
			\item $Seq$ is non-empty and starts with index 0: $\exists
a,t.Seq(0,a,t)$
			\item $Seq$ is indexed from $0$ to some final index $i$:\\			
			$\exists i,a,t.Seq(i,a,t)\land $\\
		\phantom{x}$
			(\lnot\exists i',a',t'.i'>i\land Seq(i',a',t'))\land $
($i$ is the last index of $Seq$)\\
		\phantom{x}$(\forall i'.0<i'<i\rightarrow \exists
a,t.Seq(i',a,t))$ ($Seq$ is indexed from 0 to $i$)
			\item timestamps are non-decreasing:\\
				$\forall i,a,t,a',t'.(Seq(i,a,t)\land
Seq(i+1,a',t'))\rightarrow t'\geq t$
		\end{itemize}
		
	\item $AttV$ is a functional relation mapping events and attribute names
(which cannot be integers) into objects:
	\begin{itemize}	
		\item $\forall i,n,v.AttV(i,n,v)\rightarrow$\\
			$\phantom{xx}\lnot N(n)\land$ (attribute names cannot be
integers)\\
			$\phantom{xx}\lnot\exists v'.v\neq v'\land AttV(i,n,v')$ ($AttV$ is
functional, i.e., given an event $i$ and an attribute name $n$ assigns at
most one value $v$ to $n$)
	\end{itemize}
\end{itemize}
%%
In this paper, a \emph{trace} is a model of the theory $\Gamma$ defined
above. As already mentioned, by the definition of
$\Gamma$, traces are always finite.
%%
It is immediate to see that any trace corresponds to one such model and,
viceversa, any model of
$\Gamma$ represents a trace.
In particular, $Seq$ represents the sequence of events, each represented by
its position in the sequence and its associated activity and timestamp, and
$AttV$ is the assignment of values to the relevant attributes at each event.

\begin{example}\label{ex:1}
Consider trace
$\tau=\langle\cst{a}_1[
	\cst{t}=2,
	\cst{v}=\cst{a}
],
\cst{a}_2[
	\cst{t}=5,
	\cst{w}=\cst{b},
	\cst{y}=\cst{d}
],
\cst{a}_3[
	\cst{t}=9,
	\cst{s}=\cst{c}
],
\cst{a}_3[
	\cst{t}=20,
	\cst{s}=\cst{d}
]\rangle$, which represents the sequence of events including
activity $\cst{a}_1$ with timestamp $2$ and attribute $\cst{v}$ assigned to
value $\cst{a}$,
then activity $\cst{a}_2$ with timestamp $5$ and
attributes $\cst w$ and $\cst y$ respectively assigned to value $\cst b$ and
$\cst d$, and so on.
%%
Trace $\tau$ is captured by the following interpretation $\I$ of the
trace signature $\sigma$ (we omit the interpretation of the relational
operations, as well as functions $+$ and $-$, for brevity, but these can be
easily obtained as their
restriction to $N^\I$):
%%
\begin{itemize}
	\item
		$\Delta=\C=
			\set{0,1,2,3,5,9,20}\cup
				\set{
					\cst{a_1},\cst{a_2},\cst{a_3},
					\cst{v},\cst{w}, \cst{y},\cst{s},
					\cst{a},\cst{b},\cst{c},\cst{d}
				}
		$
	\item $N^{\I}=\set{0,1,2,3,5,9,20}$
	\item $Seq^{\I}=\set{\tup{0,\cst a_1, 2},\tup{1,\cst a_2,5},\tup{2,\cst a_3,9},\tup{3,\cst a_3,20}}$
	\item $AttV^{\I}=\set{
							\tup{0,\cst v,\cst a},
							\tup{1,\cst w,\cst b},
							\tup{1,\cst y,\cst d},
							\tup{2,\cst s,\cst c},
							\tup{3, \cst s,\cst d}}$
\end{itemize}
\end{example}


\subsection{MP-Declare}
We are interested in capturing formal properties of traces to express and
solve problems typical in Declarative Process Mining.
%%
In general, our approach can deal with any property expressed
as a FO formula over the (possibly extended) signature
$\sigma$.
%%
However, in Process Mining, there are
a number of patterns that typically occur. In fact,
these are so common that the Process Mining community has
devised a specific language, namely MP-Declare~\cite{DBLP:journals/eswa/BurattinMS16},
for this purpose.
%%
We can formally prove that MP-Declare can be easily captured
by FO in our framework (this is a rather simple translation
exercise), so
by being able to solve problems state in FO, we can also
solve problems for MP-Declare. In this section,
we show some of these typical constraints, together with
their translation in FOL. Their formal semantics can be
easily obtained directly from this.

For a FOL formula $\varphi$, we write
$free(\varphi)$ to denote the set of $\varphi$'s free variables.
By a slight abuse of notation, we consider a vector $\vec x$
as a set, i.e., the set including all the components
of $\vec x$. Some typical MP-Declare constraints follow:

\begin{itemize}
	\item $Existence(A,\varphi_a)=
		\exists i,t,\vec x.Seq(i,A,t)\land\varphi_a$,
		where:
		$free(\varphi_a)\subseteq\set{i,t}\cup\vec x$
		and
		$\varphi_a$ does not mention $Seq$
		but
		can mention $AttV$ only in atoms of the form
		$AttV(i,\cdot,\cdot)$;
	\item $Absence(A,\phi)=\lnot Existence(A,\phi)$;
	\item $Choice(A,B,\phi)=Existence(A,\phi)\lor Existence(B,\phi)$;
	\item $ExclusiveChoice(A,B,\phi)=
		(Existence(A,\phi)\land Absence(B,\phi))\lor
		(Absence(A,\phi)\land Existence(B,\phi))$;
		
	\item $RespExistence(A,\varphi_a,B,\varphi_c)=
		\forall i,t.(
			\forall\vec x.(
				(Seq(i,A,t)\land \varphi_a)
					\rightarrow
				(\exists i',t'.(
					\exists \vec x'.
						Seq(i',B,t')\land\varphi_c)
				)
			)
		)
		$,
		where:
		$free(\varphi_a)\subseteq\set{i,t}\cup\vec x$
		$free(\varphi_c)\subseteq\set{i,t,i',t'}\cup\vec x\cup\vec x'$
		and
		$\varphi_a$ and $\varphi_c$ do not mention
		$Seq$ but can mention $AttV$ only in atoms
		of the forms, respectively,
		$Attv(i,\cdot, \cdot)$ and
		$Attv(i,\cdot, \cdot)$ or $Attv(i',\cdot, \cdot)$.
	\item $Response(A,\varphi_a,B,\varphi_c)=
		\forall i,t.(
			\forall\vec x.(
				(Seq(i,A,t)\land \varphi_a)
					\rightarrow
				(\exists i',t'.i'>i\land(
					\exists\vec x'.
						Seq(i',B,t')\land\varphi_c)
				)
			)
		)
		$, where $\varphi_a$ and $\varphi_c$ are as above.	
\end{itemize}
%%
This list is non-exhaustive and does not
include many other MP-Declare constraints.
%%
We observe, however, that all the constraints share a similar
structure and that none of them contains more than 2
free variables (parameters). As a result, modulo the
specific formulas $\varphi_a$ and $\varphi_c$ (representing the activation and the correlation condition of each constraint),
the examples above one can easily be adapted to the
remaining cases.

\section{Problems}
We next describe the problems that we deal with in this paper.
We stress that our framework is capable of dealing with any
FOL constraint, however, for typical problems in Declarative
Process Mining, formulas from MP-Declare are sufficient. So, while
here we remain as general as possible, in the experiments,
we will on MP-Declare constraints only.



\subsection{Generation of Event Logs}
The problem of \emph{log generation} consists in finding a set of traces,
of a desired cardinality, that satisfy a sentence $\varphi$ over $\sigma$.
In other words, this problem consists in finding a set of models of the formula
$\varphi\land\Phi_\Gamma$, where $\Phi_\Gamma=\bigwedge_{\phi\in\Gamma}\phi$
is the conjunction of all the sentences occurring in $\Gamma$.

Since $\varphi$ is an arbitrary FO formula and the problem of checking whether
one such formula admits any (even only finite) model is undecidable~\cite{libkin-book},
the desired set of traces is in general not computable. However, if one fixes
the maximum length of the traces and the interpretation domain of $\I$,
then it is immediate to see that the set is effectively computable, as the
logic used essentially reduces to propositional logic.
This is the typical scenario in Process Mining.

\begin{example}\label{ex:3}
Consider the following set of MP-Declare constraints:
\begin{itemize}
 \item $\textsc{Existence}(\cst a_2,\varphi)$, with:
	\begin{itemize}
		\item $\varphi=t>3\land \forall a_1,v_1.
			AttV(i,a_1,v_1)\rightarrow
				v_1\neq \cst d$
	\end{itemize}
		\item $\textsc{Response}(\cst a_1,\varphi_a,\cst a_3,\varphi_c)$, with:
	\begin{itemize}
		\item $\varphi_a=t\geq 1\land AttV(i,\cst v,\cst a)$, and
		\item $\varphi_c=
		\exists v_1,v_2.
			AttV(i,\cst v,v_1)\land AttV(i',\cst s,v_2)\land
				v_1\neq v_2$
	\end{itemize}
\end{itemize}
%%
Assume
that one needs to generate a set of traces of length at most, say, 10,
satisfying such constraints. Moreover, assume that:
\begin{itemize}
 \item the set of available activities is
	$A=\set{\cst a_1, \ldots,\cst a_6}$;
 \item the set of attribute names of interest is
	$AN=\set{\cst{v},\cst{w}, \cst{y},\cst{s}}$;
 \item the set of possible attribute values is:
	$AV=\set{\cst{a},\cst{b},\cst{c},\cst{d}}$
 \item we are interested in the time interval $TI=[0,100]$, i.e., the timestamps
 can take values only in this range.
\end{itemize}
%%
Notice that (for now), we do not care about the fact that certain attribute names
can be associated with certain activities only or that some values
can be used only as attribute names or values.
Of course, this can be expressed (and then enforced) by introducing
additional constraints (in the form of FOL formulas) on top of
$\Gamma$.

Considering the correspondence between models of $\Gamma$ and traces, the log generation problem consists in finding a number of distinct
models with interpretation domain $\Delta=TI\cup A\cup AN\cup AV$.
Obviously, $\Delta$ being finite, all such models can be effectively
computed (they are finite and finitely many).

For instance, the following traces (i.e., models of $\Gamma$)
satisfy the constraints of Ex.~\ref{ex:3}:

\begin{itemize}
 \item trace $\tau$ of Ex.~\ref{ex:1};
 \item trace $\tau_1=\langle\cst{a}_2[
	\cst{t}=5,
	\cst{v}=\cst{c},
	\cst{w}=\cst{v}]\rangle$ (notice that, in $\tau_1$, $\cst v$ is used as both
	an attribute name and an attribute value).
\end{itemize}



\end{example}

\subsection{Conformance Checking}
\emph{Conformance checking} is the problem of checking whether a set $L$ of
traces, a.k.a.~\emph{log}, is s.t.~each of
its elements satisfies a set of FOL (MP-Declare) constraints
$\Phi=\set{\varphi_1,\ldots,\varphi_n}$.
Observe that, since traces are always finite, the problem is decidable.
Indeed, the problem essentially
reduces to standard (boolean) query evaluation in
databases~\cite{vianu-book}, by seeing
each trace in $L$, which is a FO interpretation $\I$,
as a database over the
schema $\sigma$, and each formula $\phi\in\Phi$ as a
formula (query) over it.


\begin{example}\label{ex:2}
Consider again the set of constraints of Ex.~\ref{ex:3}.
The first constraint is satisfied by a trace if it contains
an event with timestamp $t$ greater than 3,
with activity $\cst a_2$, and s.t.~the value $\cst d$ is
not assigned to any of its attributes.
%%
The second constraint requires that whenever an event $e$ occurs
with
timestamp greater or equal than 1, with activity $\cst a_1$, and
s.t.~the value $\cst a$ is assigned to attribute $\cst v$, then
an event $e'$ must occur (strictly) later with activity
$\cst a_3$ and
s.t.~attribute $\cst v$ at $e$ and attribute
$\cst s$ at $e'$ are assigned distinct values.

Considering trace~$\tau$ of Ex.~\ref{ex:1}:
$\tau=\langle\cst{a}_1[
	\cst{t}=2,
	\cst{v}=\cst{a}
],
\cst{a}_2[
	\cst{t}=5,
	\cst{w}=\cst{b},
	\cst{y}=\cst{d}
],
\cst{a}_3[
	\cst{t}=9,
	\cst{s}=\cst{c}
],
\cst{a}_3[
	\cst{t}=20,
	\cst{s}=\cst{d}
]\rangle$, it can be easily seen that both constraints are satisfied by the
interpretation $\I$ of $\sigma$ associated with $\tau$ and described
in Ex.\ref{ex:1}.
%%
\end{example}


% %
% % \subsection{Compliance Monitoring}
% %
% %
% %
% % \begin{table*}[t!]
% % 	\centering
% %     \tiny
% % 	\begin{tabular}{ l | l  }
% % 		\hline
% % 		\textbf{Template} & \textbf{Condition}  \\
% % 		\hline
% % 		\multirow{2}{*}{response}  & when $ p > 0 $, for each pending activation, an ILP problem is instantiated using the correlation condition. \\
% %                                        & When the activation becomes fulfilled, the corresponding ILP problem is deleted. \\ \hline
% % 		  not response  & when it is activated, the negation of the correlation condition is added to all active ILP problems. \\ \hline
% % 		  precedence   & can never be in conflict. \\ \hline
% % 		  not precedence   & can never be in conflict. \\ \hline
% % 		  init   & can never be in conflict. \\ \hline
% % 		  absence   & the negation of the activation condition is always added to all active ILP problems. \\ \hline
% % 		 \multirow{2}{*}{existence}  & when $ f < 1 $, an ILP problem is instantiated using the activation condition. When the constraint becomes  \\
% %                           &   fulfilled, the corresponding ILP problem is deleted.\\ \hline
% % 		 \multirow{2}{*}{responded existence}  & when $ p > 0 $, for each pending activation, an ILP problem is instantiated using the correlation condition. \\
% %                            &        When the activation becomes fulfilled, the corresponding ILP problem is deleted.  \\ \hline
% % 		  not responded existence  & when it is activated, the negation of the correlation condition is added to all active ILP problems. \\ \hline
% % 		 \multirow{2}{*}{chain response}  & when there is a pending activation, an ILP problem is instantiated using the correlation condition. When \\
% % &  the pending activation becomes fulfilled, the corresponding ILP problem is deleted. \\ \hline
% % 		  not chain response  & can only be in conflict with a chain response with the same parameters in case both are activated. \\ \hline
% % 		  chain precedence   & can never be in conflict. \\ \hline
% % 		  not chain precedence  & can never be in conflict. \\ \hline	
% % 		 \multirow{2}{*}{alternate response}  &  when $ p > 0 $, for each pending activation, an ILP problem is instantiated using the correlation condition.\\
% % &  When the activation becomes fulfilled, the corresponding ILP problem is deleted. \\ \hline
% % 		  alternate precedence   & can never be in conflict. \\ \hline
% % 	\end{tabular}
% % 	\caption{ Criteria for ILP problem instantiation and update}
% % 	\label{tab:advConfCtr}
% % \end{table*}
% %
% %
% %
% % Our framework used for compliance monitoring is composed of the operations:
% % \begin{itemize}
% % 	\item opening: this method is called once per trace, before starting the analysis of the first event of the trace;
% % 	\item fulfillments: this method is called for each event of the trace and is supposed to return the set of fulfillments that have been observed so far;
% % 	\item violations: this method is called for each event of the trace and is supposed to return the set of violations that have been observed so far;
% % 	\item activations:  this method is called for each event of the trace and is supposed to update the set of activations that have been observed so far;
% % 	\item closing: this procedure is called once per trace, after all the events have been analyzed.
% % \end{itemize}
% % This set of operations can be instantiated for different Declare templates.
% % For example, for the response template the operations are instantiated as in Table~\ref{tab:responsealgo}:
% % \begin{itemize}
% % 	\item opening: not used;
% % 	\item fulfillments: this procedure checks whether the input event refers to a target. If this is the case, then all pending activations that can be correlated to this target (in case the time and the correlation conditions are satisfied) become fulfillments;
% % 	\item violations: not used;
% % 	\item activations: the activation procedure checks whether the input event refers to an activation of the constraint and the activation condition is satisfied (in this case the event has to be added to the set of pending activations);
% % 	\item closing: all pending activations that do not have a corresponding target when the entire trace has been processed become violations.
% % \end{itemize}
% %
% %
% %
% % The current state of a trace with respect to a constraint is described using a four valued semantics. These values are \textit{possibly satisfied}, \textit{possibly violated}, \textit{permanently satisfied} or \textit{permanently violated}. These semantic values can be interpreted as follows:
% %
% % \begin{itemize}
% % 	\item \textbf{Possibly satisfied}: means that the constraint is currently satisfied, but might be violated in the future.
% % 	\item \textbf{Possibly violated}: means that constraint is currently violated, but might be satisfied in the future.
% % 	\item \textbf{Permanently satisfied}: means that the constraint is permanently satisfied and can no longer become violated in the future.
% % 	\item \textbf{Permanently violated}: means that the constraint is permanently violated and can no longer become satisfied in the future.
% % \end{itemize}
% %
% % The semantic value for the current state of a trace depends on the type of constraint. Table~\ref{tab:semanticCriterion} shows the criteria to determine the state of a trace for each constraint type.
% %
% %
% %
% %
% % Consider constraints $absence(B)$ and $ response(A,B) $ without data. The absence constraint says that activity $ B $ should never occur in a trace. The response constraint says that if $ A $ occurs then $ B $ must eventually occur in the trace. As soon as $ A $ occurs, these constraints can still be individually satisfied, but they are \emph{conflicting} meaning that only one of them can be fulfilled at the end of the process execution. If $ B $ occurs in the trace after $A$, then the absence constraint will be violated and, if $ B $ does not occur, then the response constraint will be violated.





\subsection{Query Checking}
\emph{Query checking} can be described as the problem of \emph{query
answering}~\cite{vianu-book} over traces.
To formally define this, we need some preliminary notions.
A \emph{query} is a FO formula. If the formula is a sentence, it is called a
\emph{boolean} query. Intuitively, given an interpretation
$\I=\tup{\Delta,\cdot^\I}$ and a query $\varphi$ over the same signature
$\sigma$,
the answer to $\varphi$ over the interpretation $\I$, denoted  $\varphi^I$, is
the set of assignments to the free variables (the placeholders) of $\varphi$ that make $\I$
satisfy $\varphi$. Formally, $\varphi^\I$ is defined by induction as follows.

Let $v_1,\ldots,v_n$ be the free variables of $\varphi$ (arbitrarily sorted).
Then, $\varphi^I$ is the relation $\varphi^I\subseteq \Delta^n$
s.t.~$\tup{a_1,\ldots,a_n}\in\varphi^\I$ iff for any assignment $\nu$
s.t.~$\nu(v_i)=a_i$, it is the case that $\I,\nu\models\varphi$.
The problem of query checking consists in determining $\varphi^\I$. Notice that
if the interpretation has an infinite interpretation domain $\Delta$, then
$\varphi^\I$ cannot be computed, in general. If, instead, $\Delta$
is finite, this is not the case anymore. Again, this
is the setting of interest in Process Mining.

\begin{example}\label{ex:q-check}
	Consider once again trace~$\tau$ of Ex.~\ref{ex:1}:
$\tau=\langle\cst{a}_1[
	\cst{t}=2,
	\cst{v}=\cst{a}
],
\cst{a}_2[
	\cst{t}=5,
	\cst{w}=\cst{b},
	\cst{y}=\cst{d}
],
\cst{a}_3[
	\cst{t}=9,
	\cst{s}=\cst{c}
],
\cst{a}_3[
	\cst{t}=20,
	\cst{s}=\cst{d}
]\rangle$.
%%
Assume we are interested in all the pairs of events involved in the
satisfaction of a response constraint.
These are the activities
$?x$ and $?y$ (notice these are variables) s.t.~the following formula
holds, with $\varphi_a$ and $\varphi_c$ specific formulas
(that meet the requirements discussed before):
%%
$$\varphi=\forall i,t.(
			\forall\vec x.(
				(Seq(i,?x,t)\land \varphi_a)
					\rightarrow 
$$

\vspace{-5mm}
$$
				(\exists i',t'.i'\geq i\land(
					\exists\vec x'.
						Seq(i',?y,t')\land\varphi_c)
				)
			)
		)
		$$
Notice that the formula $\varphi$ defined above has two
free variables, $?x$ and $?y$. Thus, the answer $\varphi^\I$, where $\I$ is
the model of $\Gamma$ capturing $\tau$ (see Ex.~\ref{ex:1}) is
a relation $\varphi^\I\subseteq \Delta^2$. Specifically, the
relation contains all those pairs of activities that, once
replacing $?x$ and $?y$, make $\varphi$ true.
%%
For instance, if $\varphi_a$ and $\varphi_c$ are both $true$, then
$\varphi^I=\set{
			\tup{\cst a_1, \cst a_2},
			\tup{\cst a_1, \cst a_3},
			\tup{\cst a_2, \cst a_3},
			\tup{\cst a_3, \cst a_3}}$

If $\varphi_a$ and $\varphi_c$ are instead as in Ex.~\ref{ex:3},
we have:
$\varphi^I=\set{
			\tup{\cst a_1, \cst a_3},\allowbreak
			\tup{\cst a_2, \cst a_1},\allowbreak
			\tup{\cst a_2, \cst a_2},\allowbreak
			\tup{\cst a_2, \cst a_3},
			\tup{\cst a_3, \cst a_1},
			\tup{\cst a_3, \cst a_2},
			\tup{\cst a_3, \cst a_3}}
			$.
The counter-intuitive answer to this query, i.e., the fact
that some pairs contain activities in reversed order wrt that in
which they appear in the trace, is due to the fact
that whenever $\varphi_a$ does not hold, as a consequence of the
implication, $\varphi$ trivially evaluates to true, no
matter what activity is assigned to $?y$. It is demanded
to the designer of the query to take care of this aspect.
\end{example}
% % %
% % %
% % %
% % %
% % %
% % %
% % %
% % %
% % %
% % %
% % %
% % %
% % %
% % %
% % %
% % %
% % %
% % % Temporal logic query checking \cite{Chan/CAV2000:TemporallogicQueries} aims at discovering properties of the model that are not known \textit{a priori}.
% % % Therefore, the extra input element as compared to model checking is a \textit{query}, i.e., a temporal logic formula containing so-called \emph{placeholders}, typically denoted as $?x$. The output of the query checking technique is a set of temporal logic formulas, which derive from the input query. Every output formula stems from the replacement of placeholders with propositional formulas, which make the overall temporal logic formula satisfied in a given structure.
% % % The seminal work of \cite{Chan/CAV2000:TemporallogicQueries} considered Computation Tree Logic (CTL) \cite{Clarke.etal/ACMTPLS1986:CTL} as the language for expressing queries.
% % % Unlike {LTL}, adopting a linear time assumption, {CTL} is a branching-time logic: the evaluation of the formula is based on a tree-like structure, where different evolutions of the system over time are simultaneously considered in different branches of the computation model.
% % % %\paragraph{Constraint queries}
% % % Following the example of \cite{DBLP:conf/otm/RaimCMMM14} applying {LTL} query checking to discover (standard) declarative constraints, here we mine multi-perspective declarative processes out of event logs stored in databases through query checking.
% % % % Unlike \cite{DBLP:conf/otm/RaimCMMM14} though, additional attributes than task labels play a pivotal role in our research.
% % % % In other words, we look for assignments in {\LTLFOf} formulae expressing {\Declare} constraints such that the resulting constraint is fulfilled by the event log.
% % % %The remainder of this section is organized as follows: \cref{sec:queries:compliance} introduces the notions of event- and trace-compliance queries, namely {\LTLFOf} formulae whose answers are events and traces satisfying them in the event log; \cref{sec:metrics} builds upon the queries introduced before to define measures to be used in the context of constraints discovery, based on the frequency with which constraints are activated and fulfilled in the event log; \cref{sec:constraints:discovery:via:queries} introduces the concept of constraint queries, namely partially assigned pre-constraints bearing parametric placeholders that are meant to be valuated by the query answer; \cref{sec:sql:mapping:pre} discusses the foundations at the basis of the translation of constraint queries into SQL statements, successively detailed in \cref{sec:sql:mapping} for standard {\Declare} and \cref{sec:sql:mapping:additionalperspectives} for Multi-Perspective {\Declare}.
% % %
% % %
% % % %To discover declarative process models out of event logs stored in a relational database, we first show how to use formulae that can express declarative constraints as queries to be answered on such logs.
% % % %This motivates the following definitions.
% % % %
% % % %\begin{definition}[Event- and trace-compliance queries]\label{def:declare:mp:compliance:queries}
% % % %	Let $\psi$ be a formula of {\LTLFOfrag}, the fragment of {\LTLFOf} from which MP-{\Declare} constraints are derived as per \cref{def:ltlfo:declare:mp:fragment}. % and by extension the {\LTLFOf} formula that expresses it, as per \cref{def:declare:mp:constraint}, and $\Actv$ its activation formula, as per \cref{def:declare:mp:activation:target:formulae}.
% % % %%	Let $L \doteq (E, \LogAlph, C, \ell, \iota, \preccurlyeq,
% % % %	\widehat{\Delta}, \widehat{\alpha})$ be an event log as per \cref{def:eventlog},
% % % %	where $E \ni \Evt$ is the set of events,
% % % %	$C \ni \rho$ is the set of cases, and
% % % %	$\iota : E \to C$ is the instance function.
% % % %	As per \cref{def:trace}, $\iota$ binds events to cases defining equivalence classes $ {[\Case] \doteq \{ \Evt \,|\, \iota(\Evt) = \Case \}} $.
% % % %	Let $T \ni \Trc_{\Case}$ be the set of traces induced by $\iota$ on $L$, where trace $\Trc_{\Case}$ is a restriction of $L$ on events of $[\Case]$.
% % % %	We define the following queries, on the basis of the $\models$ relation discussed in \cref{def:ltlf:semantics,def:declare:mp:pre:constraint}:
% % % %	\begin{description}
% % % %		\item[Event-compliance query]  $\DeQuery{\Evt}[\psi]$: Its \emph{answer} on $L$ is
% % % %		${ \lbrace \Evt \mid \exists \Case.\ \Evt \in [\Case]  } \land { \Trc_{\Case}, \Evt \models \psi \rbrace }$;
% % % %		\item[Activated event compliance query]  $\DeQuery{\Evt}[\Cns]_\bullet$: Its \emph{answer} on $L$ is
% % % %		${ \lbrace \Evt \mid \exists \Case.\ \Evt \in [\Case] } \land { \Trc_{\Case}, \Evt \models \Actv \land \Cns \rbrace }$;
% % % %		\item[Trace-compliance query]  $\DeQuery{\Case}[\psi]$: Its \emph{answer} on $L$ is
% % % %		${ \lbrace \Case \mid \Trc_{\Case}, \FirstEl{\Case} \models \psi \rbrace }$.
% % % %		\item[Activated trace compliance query]  $\DeQuery{\Case}[\Cns]_\bullet$: Its \emph{answer} on $L$ is
% % % %		${ \lbrace \Case \mid \forall \Evt \exists \Evt.\ \Evt \in [\Case] \to \Trc_{\Case}, \Evt \models \Cns } \land { \Evt \in [\Case] \land \Evt \models \Actv \land \Cns \rbrace }$.
% % % %	\end{description}
% % % %\end{definition}
% % % %\noindent
% % % %A constraint per se can thus be seen as a query on the event log. An answer to an event- or trace-compliance query is the set of events, or resp.\ cases, %that satisfy it, thus being compliant with the constraint. %, i.e., the set of events that comply with the constraint. %
% % % %Let us consider as an example the formula corresponding to
% % % %$ \Cns \doteq \Resp{\taskb}{\taskd} \equiv \lglobally ( \taskb \rightarrow \taskd ) $.
% % % %The answer to $\DeQuery{\Evt}[\Cns]$ on the example event log excerpt of \cref{table:eventlog} is
% % % %$\lbrace \underline{{\Evt}_{\Task{4}}}, \underline{{\Evt}_{\Task{5}}}, \underline{{\Evt}_{\Task{6}}} \rbrace$.
% % % %The answer to $\DeQuery{\Case}[\Cns]$ on the same event log is
% % % %$\lbrace \underline{{\Case}_{\Task{1}}} \rbrace$.
% % % %Given another formula
% % % %$ \Cns' \doteq \Resp{\taskb \land \Attr{Score} > \text{50} }{\taskd} \equiv \lglobally ( ( \taskb \land \Attr{Score} > \text{50} ) \rightarrow \taskd ) $
% % % %we have that
% % % %the answer to $\DeQuery{\Evt}[\Cns]$ is
% % % %$\lbrace \underline{{\Evt}_{\Task{1}}}, \underline{{\Evt}_{\Task{2}}}, \ldots, \underline{{\Evt}_{\Task{6}}} \rbrace$,
% % % %as expected due to the anti-monotonicity \cref{thm:check-expansion:activation:anti-monotonicity} -- here the activation is check-expanded with $\Attr{Score} %> \text{50}$ with respect to $\Cns$.
% % % %Likewise,
% % % %the answer to $\DeQuery{\Case}[\Cns]$ is
% % % %$\lbrace \underline{{\Case}_{\Task{1}}}, \underline{{\Case}_{\Task{2}}} \rbrace$, accordingly.
% % % %
% % % % We remark here that the aforementioned queries can be extended to any {\LTLf} formula and, a fortiori, to SP-{\Declare}.
% % %
% % % %Also queries non strictly related to the templates of (MP-){\Declare} can be formulated.
% % % %In particular, we shall make use of the following queries in the remainder of this paper.
% % % %Given a constraint with activation formula $\Actv$, the answer to $\DeQuery{\Evt}[\Actv]$ consists of all the events activating a constraint;
% % % %the answer to $\DeQuery{\Case}[\lfuture \Actv]$ consists of all the \emph{cases} in which a constraint was activated at least once.
% % % %The answer to $\DeQuery{\Case}[\ltrue]$ is the set of all the cases of a log, namely $C$.

\section{Alloy for Declarative Process Mining}

The problems presented in the previous section can be cast into
various problems
typical of databases. This can be done because in all such problems, we deal
with a finite interpretation domain.
As described above, solving these
problems requires in practice either constructing a set of models
or building
a relation that satisfies certain requirements.
%%
In this work, we explore the use of Alloy~\cite{alloy} as a tool for
these tasks.

Alloy is a tool for the construction of models of FO theories over
finite interpretation domains. In its essence, the tool takes
as input a FO theory and an interpretation domain (or a bound
on the size of the interpretation domain) and returns,
if any, a model of the input theory.

Interestingly, Alloy builds the models by compiling the problem
into SAT, using propositions to represent FO atoms built from
predicate symbols and the finitely many objects in the interpretation
domain. We are not interested here in the compilation step,
but in: \myi how we can cast our problems
into that of finding a model for a FO theory; and \myii how effective
the (SAT-based) solution approach of Alloy is in solving the
problems of interest. In this section,
we discuss the former point, while, in the section
about experiments, we deal with the latter.

\subsection{Generation of Event Logs}
Generation of event logs is possibly the simplest
problem to solve with Alloy. We simply take $\Gamma$ and the
conjunction of the constraints to satisfy, say $\varphi$, and
ask Alloy to search for the models of $\Gamma\cup\set{\varphi}$,
for a given (finite) $\Delta$.
Observe that, if needed, we can add, on top of the input
theory, additional constraints, as long as these are
consistent with $\Gamma\cup\set{\varphi}$.
For instance, we may require that the set of
attribute names be restricted to a certain desired set:
$$\forall i,a,v.AttV(i,a,v)\rightarrow
	a=\cst v\lor
	a=\cst w\lor
	a=\cst y\lor
	a=\cst s,$$
analogously for attribute values:
$$\forall i,a,v.AttV(i,a,v)\rightarrow
	v=\cst a\lor
	v=\cst b\lor
	v=\cst c\lor
	v=\cst d,$$
or even that certain attributes may take only certain values:
$$\forall i,v.AttV(i,\cst v,v)\rightarrow
	v=\cst a\lor
	v=\cst d.$$
	
To obtain many traces, we ask Alloy, once returned a model,
to build one more, until enough traces have been
produced or no more solutions are available.
Notice that we can incrementally increase the size of the
interpretation domain by adding new constants to obtain
additional traces.


% % % % \subsection{Preliminaries on First-order Logic}
% % % % We start by briefly recalling
% % % % standard preliminary notions about first-order logic.
% % % % %%
% % % % A first-order (FO) \emph{vocabulary} $\sigma$ consists of sets of
% % % % \emph{constant
% % % % symbols} $c_1,\ldots,c_n$, \emph{predicate symbols} $P_1,\ldots,P_m$ and
% % % % \emph{function symbols} $f_1,\ldots,f_q$, where each predicate or function
% % % % symbol has a respective (finite) arity.
% % % % %%
% % % % An \emph{interpretation} $\I$ of a FO vocabulary $\sigma$ over a
% % % % (possibly infinite) \emph{universe} $U$ is a pair
% % % % $I=\tup{U,\cdot^I}$, where $\cdot^\I$ is the \emph{interpretation function}
% % % % of $\I$, i.e., a function
% % % % associating:
% % % % 	\begin{itemize}
% % % % 		\item each constant $c$ with an object $c^\I\in U$;
% % % % 		\item each predicate symbol $P$ of arity $a$ with a relation
% % % % $P^\I\subseteq U^a$;
% % % % 		\item each function symbol $f$ of arity $a$ with a mapping
% % % % $f^\I:U^a\rightarrow U$.
% % % % 	\end{itemize}
% % % % %%
% % % % For our purposes, we can assume that constants
% % % % are always interpreted as themselves, i.e., that for any interpretation $\I$,
% % % % it is the case that $c^\I=c$ (this is a common assumption in many situations,
% % % % e.g., in database theory~\cite{vianu-book}). Obviously, this implies that all
% % % % (and only) constant symbols are included in $U$.
% % % %
% % % % Given a FO vocabulary $\sigma$ and an infinite (countable) set $V$ of
% % % % variable symbols, formulas $\varphi$ of FOL over $\sigma$
% % % % respect the following syntax:
% % % % $$\varphi = P(\vec t)\mid \lnot \varphi\mid \varphi\land\varphi\mid \exists
% % % % x.\varphi, \mbox{ where:}$$
% % % % \begin{itemize}
% % % %  \item $P$ is a predicate symbol from $\sigma$;
% % % %  \item $\vec t$ is a tuple
% % % % 	$\tup{t_1,\ldots,t_a}$, with $a$ the arity of $P$,
% % % % 	and each $t_i$ a \emph{term} from $\sigma$, i.e., a variable or a constant
% % % % symbol from $\sigma$, or a \emph{function term}, i.e., an expression of the
% % % % form $f(\vec t')$, with $f$ a function symbol from $\sigma$ and $\vec t'$ a
% % % % tuple of terms
% % % % from $\sigma$, of size equal to the arity of $f$.
% % % % \end{itemize}
% % % % %%
% % % % As standard, we define the following abbreviations:
% % % % $\varphi_1\lor \varphi_2\doteq \lnot(\lnot\varphi_1\land\lnot\varphi_2)$,
% % % % $\forall x.\varphi\doteq\lnot\exists x.\lnot\varphi$,
% % % % $\varphi_1 \rightarrow \varphi_2\doteq \lnot\varphi_1 \lor \varphi_2$,
% % % % and $\varphi_1 \leftrightarrow \varphi_2\doteq (\varphi_1 \rightarrow
% % % % \varphi_2) \land (\varphi_2 \rightarrow \varphi_1)$.
% % % % %%
% % % % The variable $x$ in $\exists x.\varphi$ (resp.~$\forall x.\varphi$) is
% % % % an (existentially, resp.~universally) quantified variable.
% % % % Variables in a FO
% % % % formula do not have to be quantified, in which case they are called
% % % % \emph{free variables}.
% % % % Formulas containing only quantified variables are called \emph{sentences}.
% % % % %%
% % % %
% % % % FO formulas are interpreted over FO interpretations and
% % % % \emph{variable assignments}, i.e., total mappings $\nu:V\mapsto U$,
% % % % for $U$ the universe of the interpretation in question.
% % % % Formula interpretations require interpreting their terms first.
% % % % Given a FO interpretation $\I=\tup{U,\cdot^\I}$ of $\sigma$ and a term $t$
% % % % over $\sigma$, the
% % % % interpretation of $t$ under $\nu$ and $\I$ is the object
% % % % $t^\I_\nu\in U$ such that:
% % % % \begin{itemize}
% % % %  \item $t^\I_\nu = \nu(t)$, if $t$ is a variable symbol;
% % % %  \item $t^\I_\nu = c^\I(=c)$, if $t$ is a constant symbol;
% % % %  \item $t^\I_\nu = f^\I({t_1}^\I_\nu,\ldots,{t_a}^\I_\nu)$, for $a$
% % % % the arity of $f$, if $t=f(t_1,\ldots,t_a)$.
% % % % \end{itemize}
% % % % %%
% % % %
% % % % A FO interpretation
% % % % $\I$ is said to \emph{satisfy}
% % % % a FO formula $\varphi$ under assignment $\nu$,
% % % % written $\I, \nu\models\varphi$, iff either:
% % % % %%
% % % % \begin{itemize}
% % % % 	\item $\varphi=P(t_1,\ldots,t_a)$ and
% % % % 	$\tup{{t_1}^\I_\nu,\ldots,{t_a}^\I_\nu}\in P^\I$;
% % % % 	\item $\varphi=\lnot \phi$ and $\I,\nu\not\models\phi$;
% % % % 	\item $\varphi=\phi_1\land \phi_2$ and $\I,\nu\models \phi_i$, for $i=1,2$;
% % % % 	\item $\varphi=\exists x.\phi$ and there exists $u\in U$ such that
% % % % $\I,\nu'\models \phi$, with $\nu'(x)=u$ and $\nu'(v)=\nu(v)$, for $v\neq x$.
% % % % \end{itemize}
% % % % %%
% % % % % % Notice that, in order to check satisfaction of $\varphi$, all
% % % % %%terms occurring
% % % % % % in it must be replaced by their interpretation.
% % % % % %
% % % % Finally, an interpretation $\I$ \emph{satisfies}
% % % % a formula $\varphi$, written $\I\models \varphi$, iff $\I,\nu\models\varphi$
% % % % for all assignments $\nu$. This notion is used, in particular, when $\varphi$
% % % % is a sentence.
% % % %
% % % % A \emph{FO theory} $\Gamma$ is a possibly infinite set of FO sentences. Here,
% % % % we focus on finite theories.
% % % % An interpretation is said to
% % % % satisfy a theory $\Gamma$ if $\I\models\varphi$ for all $\varphi\in\Gamma$.
% % % % When
% % % % this is the case, $\I$ is said to be a \emph{model} of $\Gamma$.
% % % % %%
% % % % It is well known that checking existence of a model of a theory is
% % % % an undecidable problem, even if one restricts to \emph{finite} models only,
% % % % i.e., models whose universe $U$ is finite (see, e.g., \cite{FO-book}). The
% % % % problem is instead decidable if one considers only bounded models, i.e., with
% % % % $U$ of cardinality no larger than a fixed bound.In this case, which is the one
% % % % of interest in this work, the model can be effectively computed~\cite{}.
% % % %
% % % % One possible approach to construct (bounded) models of (finite)
% % % % theories is based on compilation into SAT. The idea is to compile, depending
% % % % also on the bound $b$, the theory into a boolean formula $\phi$ such that
% % % % $\phi$ is satisfied iff there exists a model of the theory with a universe
% % % % bounded in size by $b$.
% % % % This approach, the one adopted by
% % % % the software analysis tool Alloy~\cite{}, has the great advantage of benefiting
% % % % directly from the continuous improvements obtained by
% % % % SAT solvers, as well as offering to users, at essentially no additional
% % % % cost, the whole set of SAT solvers available to date.
% % % %
% % % % Here, we rely on Alloy for this reason, i.e., to check the
% % % % effectiveness of SAT technology as a way to address problems typical in Process
% % % % Mining. We stress that we are not interested in the compilation process itself,
% % % % but simply as a way to compile our problems into SAT. Thus, we will use Alloy as
% % % % a black-box tool, without entering the details of the compilation process. We
% % % % refer the interested reader to~\cite{JacksonSS00} and to the numerous papers on
% % % % the topic, accessible from the Alloy website[\cite{}, footnote?,
% % % % already cited?].
% % % %
% % % % \subsection{Modeling Log Traces as FO Theories}
% % % %
% % % % In this section, we explain how FO theories can be used to capture log
% % % % traces. For a given integer $b$, we represent a trace of size $b$ as an ordered
% % % % set of $b$ events.
% % % % Let $\{a_1,\ldots,a_p\}$ be the set of possible activities;
% % % % assume that activity $a_i$ has $q_i$ attributes
% % % % $att_{ij}$ ($j=1,\ldots,q_i$), each with possible values from
% % % % a domain $D_{ij}=\{d_{ij}^1,\ldots,d_{ij}^{r_{ij}}\}$.
% % % % %%
% % % % Then, the vocabulary $\sigma$ includes:
% % % % \begin{itemize}
% % % % 	\item $b$ constants $\mathtt{e}_1,\ldots,\mathtt{e}_b$, representing the
% % % % events in the trace;
% % % % 	\item $p$ constants $\mathtt{a}_1,\ldots,\mathtt{a}_p$, representing the
% % % % possible activities;
% % % % 	\item constants
% % % % $\mathtt{att}_{11},\ldots,\mathtt{att}_{1q_1}\ldots,\mathtt{att}_{p1},\ldots,
% % % % \mathtt{att}_ {pq_p}$, modeling the activity attributes;
% % % % 	\item constants
% % % % $\mathtt{d}_{11}^1,\ldots,\mathtt{d}_{11}^{r_{11}},\ldots,\mathtt{d}_{pq_p}^1,
% % % % \ldots,\mathtt{d}_{ pq_p}^{r_{ pq_p}}$
% % % % representing values in the attribute domains;
% % % % 	\item unary predicate symbols $E$(vents), $A$(ctivities), $Att$(tributes),
% % % % and $D_{11},\ldots,D_{1q_1}\ldots,D_{p1},\ldots{D_{p_{q_p}}}$, one per
% % % % attribute;	
% % % % \item binary predicate symbols
% % % % $hasAtt$(ribute), $N$(ext) and $Aft$(er);
% % % % \item a unary function symbol $act$;
% % % % \item a binary function symbol $att$;
% % % % \end{itemize}
% % % % %%
% % % % With a slight abuse of notation we use
% % % % the same names for activities, events, etc., and their corresponding predicate
% % % % symbols. However, we use a different $\mathtt{font}$ for constants,
% % % % to distinguish them from variables.
% % % %
% % % % Predicate symbol $E$ is intended to capture trace events, $A$ activities, $Att$
% % % % attributes, and $D_{ij}$ the set of possible values that can be assigned to
% % % % attribute $att_{ij}$;
% % % % the relation symbol $hasAtt$(attribute) models the relationship between
% % % % activities and attributes;
% % % % the relation symbols $N$ and
% % % % $Aft$ model the ordering of events;
% % % % the unary function symbol $act$ represents the association between events and
% % % % activities; finally, the binary function symbol $att$ is used to model the
% % % % value assignment to attributes.
% % % %
% % % % We can now formally define the theory $\Gamma$. This is
% % % % obtained as the union of the sentences described below.
% % % % %%
% % % % First, to guarantee that $E$ contains all and only the constants
% % % % $\mathtt{e}_i$, we include in $\Gamma$ the following sentence:
% % % % %%
% % % % \begin{equation}\label{eq1}
% % % % \big(\bigwedge_{i=1}^b E(\mathtt{e}_i)\big)
% % % % \land
% % % % \big(\forall
% % % % x.E(x)\rightarrow \bigvee_{i=1}^b x=\mathtt{e}_i\big)
% % % % \end{equation}
% % % % %%
% % % % We do the same with $A$, $Att$ and each $D_{ij}$, thus imposing that $A$
% % % % contains
% % % % exactly the activities $\mathtt{a}_1,\ldots,\mathtt{a}_p$,
% % % % that $Att$ contains exactly the attributes
% % % % $\mathtt{att}_{11},\ldots,\mathtt{att}_{1q_1},\ldots,\mathtt{att}_{p1},\ldots,
% % % % \mathtt{att}_{pq_p}$ , and that each $D_{ij}$ contains exactly
% % % % the values $\mathtt{d}_{ij}^1,\ldots,\mathtt{d}_{ij}^{r_{ij}}$.
% % % %
% % % % We impose also that the interpretation domain contains exactly the constants
% % % % mentioned so far:
% % % % \begin{equation}\label{eq2}
% % % % \forall x. E(x)\lor A(x)\lor Att(x)\lor D_{11}(x)\lor\cdots\lor
% % % % D_{pq_p}(x)
% % % % \end{equation}
% % % %
% % % %
% % % % To express that each activity $\mathtt{a}_i$ has (exactly) attributes
% % % % $\mathtt{att}_{i1},\ldots,\mathtt{att}_{iq_i}$, we write the following:
% % % % \begin{equation}\label{eq3}
% % % % 	\bigwedge_{i=1}^p
% % % % 		\forall y.hasAtt(\mathtt{a}_i,y)\leftrightarrow
% % % % 		\bigvee_{j=1}^{q_i} (y=\mathtt{att}_{ij})	
% % % % \end{equation}
% % % %
% % % % Then, through $N$, we define an arbitrary order on the events. This can be
% % % % arbitrary because events are essentially placeholders whose identity does not
% % % % carry any relevant information; what matters, instead, is set of activities, in
% % % % the trace, together with their respective attribute
% % % % assignments, and their mutual order.
% % % % With the following sentence, we constrain $N$ to represent an ``immediate
% % % % successor'' relation:
% % % % \begin{equation}\label{eq4}
% % % % \forall e,e'.N(e,e')\leftrightarrow
% % % % 	((e=\mathtt{e}_1\land
% % % % e'=\mathtt{e}_2)\lor\cdots\lor(e=\mathtt{e}_{b-1}\land e'=\mathtt{e}_{b}))
% % % % \end{equation}
% % % % That is, we set the successor of event $\mathtt{e}_i$ as $\mathtt{e}_{i+1}$.
% % % %
% % % % Relation $Aft$ models
% % % % that an event $e'$ occurs in the future wrt an event $e$:
% % % % \begin{equation}\label{eq5}
% % % % \forall
% % % % e,e'.Aft(e,e')\leftrightarrow\bigvee_{i=1}^{b-1}(e=\mathtt{e}_i\land\bigvee_{
% % % % j=i+1 } ^ { b-1 } e'=\mathtt{e}_j)
% % % % \end{equation}
% % % % This sentence says that for any two events $e_i$ and $e_j$, it hods that
% % % % $Aft(\mathtt{e}_i,\mathtt{e}_j$) if and only if $j>i$. Notice this is
% % % % consistent with the
% % % % ordering defined by $N$.
% % % %
% % % % Then, we require that every event $e$ has one activity associated through $act$:
% % % % \begin{equation}\label{eq6}
% % % %  \forall e.E(e)\rightarrow A(act(e))
% % % % \end{equation}
% % % %
% % % % Finally, we impose that every activity associated with an event has all of its
% % % % attributes valuated with some value coming from the appropriate
% % % % domain:
% % % % \begin{equation}\label{eq7}
% % % % \begin{array}{l}
% % % % 	\forall e,a,x,d.(E(e)\land a=act(e)\land
% % % % hasAtt(a,x)\land d=att(e,x))\rightarrow\\
% % % % \phantom{xxxxxxxxxxxxxxxxxxxxxxxxxx}\bigvee_{i=1}^p a=\mathtt{a}_i\land
% % % % (\bigvee_{j=1}^{q_p} (x=\mathtt{att}_{ij} \land D_{ij}(d)))
% % % % \end{array}
% % % % \end{equation}
% % % %
% % % % It can be easily shown that any trace of size $b$ (over the above mentioned
% % % % activities, with corresponding attributes and respective domains)
% % % % can be captured by a FO interpretation of the theory $\Gamma$ defined by the
% % % % above sentences. Viceversa, any model of $\Gamma$ captures a trace as above.
% % % %
% % % % \begin{example}\label{ex1}
% % % % Assume the following possible activities:
% % % %  \begin{itemize}
% % % %   \item $A_1$, with attribute $v$, taking values
% % % % from $D_{11}=\{a,b\}$;
% % % %   \item $A_2$, with attributes $w$ and
% % % % $y$, taking values
% % % % from, respectively, $D_{21}=\{b,d\}$
% % % % and
% % % % $D_{22}=\{b,d\}$;
% % % %   \item $A_3$, with attribute $s$, taking values from
% % % % $D_{31}=\{c,d,e\}$.
% % % %  \end{itemize}
% % % % %%
% % % % For $b=4$ the length of a trace, the vocabulary $\sigma$
% % % % includes (remember we use the same names for events, activities, etc. and for
% % % % their counterparts in $\sigma$):
% % % % \begin{itemize}
% % % % 	\item predicate symbols:
% % % % 		$E$, $A$, $Att$, $D_{11}$, $D_{21}$, $D_{22}$, $D_{31}$;
% % % % 	\item binary relation symbols
% % % % 		$hasAtt, N, Aft$;
% % % % 	\item constants:
% % % % 		$\mathtt{e}_1,\mathtt{e}_2,\mathtt{e}_3,\mathtt{e}_4$,
% % % % 		$\mathtt{A}_1,\mathtt{A}_2,\mathtt{A}_3$,
% % % % 		$\mathtt{v},\mathtt{w},\mathtt{y},\mathtt{s}$,
% % % % 		$\mathtt{a},\mathtt{b},\mathtt{c},\mathtt{d},\mathtt{e}$;
% % % % 	\item function symbols:
% % % % 		$act$, $att$.
% % % % \end{itemize}
% % % % %%
% % % % $\Gamma$ is obtained by instantiating sentences~\ref{eq1}
% % % % to~\ref{eq7} according to the assignments above.
% % % % For instance, for $D_{21}$, sentence~(\ref{eq1}) is as follows:
% % % % %%
% % % % $$D_{21}(\mathtt{b})\land D_{21}(\mathtt{d})\land \big(\forall
% % % % x.D_{21}(x)\rightarrow (x=\mathtt{b}\lor
% % % % x=\mathtt{d})\big)$$
% % % % %%
% % % % Sentence~(\ref{eq2}) remains the same, while sentence~(\ref{eq3}) is
% % % % instantiated
% % % % as follows:
% % % % $$
% % % % 	\begin{array}{c}
% % % % 	(\forall y.hasAtt(\mathtt{A}_1,y)\leftrightarrow (y=\mathtt{v}))\land\\
% % % % 	(\forall y.hasAtt(\mathtt{A}_2,y)\leftrightarrow
% % % % (y=\mathtt{w})\lor(y=\mathtt{y}))\land\\	
% % % % 	(\forall y.hasAtt(\mathtt{A}_3,y)\leftrightarrow (y=\mathtt{s}))
% % % % 	\end{array}
% % % % $$
% % % % %%
% % % % As to sentence~(\ref{eq4}), we have:
% % % % $$
% % % % \forall e,e'.N(e,e')\leftrightarrow
% % % % 	(
% % % % 		(e=\mathtt{e}_1\land
% % % % 		e'=\mathtt{e}_2)\lor
% % % % 		(e=\mathtt{e}_2\land
% % % % 		e'=\mathtt{e}_3)\lor
% % % % 		(e=\mathtt{e}_3\land
% % % % 		e'=\mathtt{e}_4)
% % % % )
% % % % $$
% % % % %%
% % % % Sentence~(\ref{eq5}) becomes:
% % % % $$
% % % % \begin{array}{ll}
% % % % \forall e,e'.Aft(e,e')\leftrightarrow(&
% % % % 	(e=\mathtt{e}_1\land (e'=\mathtt{e}_2\lor e'=\mathtt{e}_3 \lor
% % % % e'=\mathtt{e}_4))\lor  \\ &
% % % % 	(e=\mathtt{e}_2\land (e'=\mathtt{e}_3 \lor
% % % % e'=\mathtt{e}_4))\lor  \\ &
% % % % 	(e=\mathtt{e}_3\land e'=\mathtt{e}_4))
% % % % \end{array}
% % % % $$
% % % % %%
% % % % Sentence~(\ref{eq6}) remains unchanged, while sentence~(\ref{eq7}) becomes as
% % % % follows:
% % % % $$
% % % % \begin{array}{l}
% % % % \forall e,a,x,d.(E(e)\land a=act(e)\land
% % % % hasAtt(a,x)\land d=att(e,x))\rightarrow (\\
% % % % \phantom{xxxxxxxxxxxxxxxxxxxxxxxxxxxxxxxxxx}
% % % % (a=\mathtt{A}_1\land (x=\mathtt{v}\land D_{11}(d)))\lor\\
% % % % \phantom{xxxxxxxxxxxxxxxxxx}
% % % % (a=\mathtt{A}_2\land ((x=\mathtt{w}\land
% % % % D_{21}(d))\lor(x=\mathtt{y}\land D_{22}(d))))\lor\\
% % % % \phantom{xxxxxxxxxxxxxxxxxxxxxxxxxxxxxxxxxx}
% % % % (a=\mathtt{A}_3\land (x=\mathtt{s}\land D_{31}(d))))
% % % % \end{array}
% % % % $$
% % % %
% % % % Now, consider the trace:
% % % % $$A_1[v=a]A_2[w=b,y=d]A_3[s=c]A_3[s=d]$$
% % % % %%
% % % % The model of $\Gamma$ corresponding to this trace is the interpretation $\I$
% % % % such that:
% % % % \begin{itemize}
% % % %  \item $E^\I=\{\mathtt{e}_1,\mathtt{e}_2,\mathtt{e}_3,\mathtt{e}_4\}$;
% % % %  \item $A^\I=\{\mathtt{A}_1,\mathtt{A}_2,\mathtt{A}_3\}$;
% % % %  \item $Att^\I=\{\mathtt{v},\mathtt{w},\mathtt{y},\mathtt{s}\}$;
% % % %  \item $D_{11}^\I=\{\mathtt{a},\mathtt{b}\}$,
% % % % 	$D_{21}^\I=D_{22}^\I=\{\mathtt{b},\mathtt{d}\}$,
% % % % 	$D_{31}^\I=\{\mathtt{c},\mathtt{d},\mathtt{e}\}$;
% % % %  \item
% % % % $hasAtt^\I=\{\tup{\mathtt{A}_1,\mathtt{v}},
% % % % \tup{\mathtt{A}_2,\mathtt{w}},\tup{\mathtt{A}_2,\mathtt{y}},\tup{\mathtt{A}_3,
% % % % \mathtt{s}}\}$
% % % % \item $N^\I=\{\tup{\mathtt{e}_1,\mathtt{e}_2},
% % % % \tup{\mathtt{e}_2,\mathtt{e}_3},
% % % % \tup{\mathtt{e}_3,\mathtt{e}_4}\}$;
% % % % \item $Aft^I=\{
% % % % \tup{\mathtt{e}_1,\mathtt{e}_2},
% % % % \tup{\mathtt{e}_1,\mathtt{e}_3},
% % % % \tup{\mathtt{e}_1,\mathtt{e}_4},
% % % % \tup{\mathtt{e}_2,\mathtt{e}_3},
% % % % \tup{\mathtt{e}_2,\mathtt{e}_4},
% % % % \tup{\mathtt{e}_3,\mathtt{e}_4}
% % % % \}$
% % % % \item $act^I=\{
% % % % \mathtt{e}_1\mapsto\mathtt{A_1},
% % % % \mathtt{e}_2\mapsto\mathtt{A_2},
% % % % \mathtt{e}_3\mapsto\mathtt{A_3},
% % % % \mathtt{e}_4\mapsto\mathtt{A_3},
% % % % \cdots
% % % % \}$, where $\cdots$ denotes that the other assignments can be any (as not
% % % % relevant);
% % % % \item $att^I=\{
% % % % \tup{\mathtt{e}_1,\mathtt{v}}\mapsto\mathtt{a},
% % % % \tup{\mathtt{e}_2,\mathtt{w}}\mapsto\mathtt{b},
% % % % \tup{\mathtt{e}_2,\mathtt{y}}\mapsto\mathtt{d},
% % % % \tup{\mathtt{e}_3,\mathtt{s}}\mapsto\mathtt{c},
% % % % \tup{\mathtt{e}_4,\mathtt{s}}\mapsto\mathtt{d},\cdots
% % % % \}$ (again, $\cdots$ denote irrelevant assignments);
% % % % \end{itemize}
% % % % %%
% % % % For any other trace of length $4$, a similar interpretation obviously exists.
% % % % \end{example}
% % % %
% % % % It is immediate to generalize Example~\ref{ex1} to the case of any $b$ and
% % % % trace of length $b$, thus obtaining that any trace of length be can be captured
% % % % by an interpretation of a theory $\Gamma$ as above.
% % % %
% % % % Also the converse can be shown, i.e., that, for given $b$, any interpretation
% % % % of a theory $\Gamma$ as above represents a trace.
% % % % Indeed, for fixed $b$, all the models of $\Gamma$ match except
% % % % for the interpretations $act^\I$ and $att^\I$, i.e., for the activities
% % % % associated with each event and
% % % % the corresponding assignment to each of their attributes.
% % % % This is a direct consequence of sentences~\ref{eq1} to~\ref{eq5}. However,
% % % % sentences~\ref{eq6} and~\ref{eq7} respectively prescribe that: exactly one
% % % % activity is assigned to each event and all the attributes of an activity
% % % % assigned to some event are assigned a value from their respective domain.
% % % % Based on this, it can be easily seen that any interpretation of $\Gamma$ is,
% % % % essentially, a trace.
% % % %
% % % % \subsection{MP Declare Templates}
% % % % On top of $\Gamma$, we can specify additional requirements, so as to constrain
% % % % the traces represented by its models to satisfy certain properties. In
% % % % particular, here, we are interested in capturing Declare templates...
% % % %
% % % % {\bf [FP: Si possono prendere da Burattin et al.? Magri possiamo dire che sono
% % % % essenzialmente analoghi e dare un paio di esempi (tipo init e response con
% % % % dati)]}
% % % %
% % % % \subsection{Introducing Numeric Values}
% % % % In many cases, we need to deal with numeric attribute values. For instance,
% % % % one can be interested in guaranteeing that the value of a certain attribute
% % % % lies below a fixed threshold, or that it equals a certain value.These
% % % % comparisons may occur within the Declare constraints.
% % % %
% % % % While, in
% % % % general, comparisons can be more involved, such as checking whether a value
% % % % is less than another value plus a certain quantity, here we address the
% % % % basic problem of comparing attributes against constant values. Specifically, we
% % % % assume that some attributes are \emph{integer}, i.e., they can be assigned
% % % % integer values and that the only allowed comparisons are of the form
% % % % $attr=num$, $attr\neq num$, $attr\leq num$, $attr<num$, $attr\geq num$, and
% % % % $attr>num$, with $attr$ an attribute and $num$ a (numeric) constant.
% % % %
% % % % Treating numeric values in FOL is in general problematic, as there is no way to
% % % % control the cardinality of infinite sets. In this respect, it is well known,
% % % % e.g., that one cannot write a theory whose models are (isomorphic to)
% % % % structures such as the integers (in fact, one cannot even express that the
% % % % universe be numerable). To overcome this, a series of frameworks
% % % % have been introduced, see, e.g., the framework of \emph{embedded finite
% % % % models}~\cite{libkin}.
% % % % %%
% % % % In
% % % % our case, the problem is even more basic: since we deal with
% % % % bounded sets only (in order to compile the theory into SAT),
% % % % including (no matter how) the integers in our theory is out of question.
% % % %
% % % % Nonetheless, if we allow only the comparisons mentioned above (extending to
% % % % more general conditions is left for future work) we can still reason about
% % % % numeric values.
% % % % The crucial observation is that in a Declare model, being finite,
% % % % there may occur only a finite number of numeric comparisons, thus only a finite
% % % % number of numeric values can be mentioned. Starting from this, we can
% % % % introduce, for each comparison, a new predicate symbol modeling whether
% % % % the comparison is satisfied or not by the involved attribute, and then use
% % % % this to force, through suitable sentences, its satisfaction.
% % % %
% % % % Assume for instance that the Declare model includes the comparison
% % % % $att_{ij}\leq 10$, with attribute $att_{ij}$ and constant $10$. To deal
% % % % with this, we proceed as follows:
% % % % \begin{itemize}
% % % %  \item We include in $\sigma$ a new unary predicate $lt10$.
% % % %  \item As constants associated with $att_{ij}$, we include in $\sigma$
% % % % only $\mathtt{d}_{ij}^1$ and $\mathtt{d}_{ij}^2$ and constrain,
% % % % through sentences~(\ref{eq1}), $D_{ij}$ to contain exactly
% % % % $\mathtt{d}_{ij}^1$ and $\mathtt{d}_{ij}^2$.
% % % %  \item We specify, through one of the sentences~(\ref{eq1}), that $lt10$
% % % % contains exactly $\mathtt{d}_{ij}^1$ (and not $d_{ij}^2$). This is intended to
% % % % model that $\mathtt{d}_{ij}^1$ is a numeric value less or equal than 10, while
% % % % $\mathtt{d}_{ij}^2$ is not.
% % % % \item Finally, whenever we need to state that the assignment of
% % % % $\mathtt{att}_{ij}$ at a given event $e$ must be less or equal than $10$, we
% % % % write $lt10(att(e,\mathtt{att}_{ij}))$.
% % % % \end{itemize}
% % % % Then, once the model is obtained, the desired trace can be obtained by
% % % % simply replacing every occurrence of $\mathtt{d}_{ij}^1$ with any value $v\leq
% % % % 10$ and every occurrence of $\mathtt{d}_{ij}^2$ with any value $v> 10$.
% % % %
% % % % For instance, if the obtained model $\I$ corresponds to trace:
% % % % $$\tau=A_1[v=\mathtt{d}_{11}^1]A_2[w=b,y=d]A_3[s=c]A_3[s=d]$$
% % % % and in the model we have that
% % % % $lt10^\I=\{\mathtt{d}_{11}^1\}$, then any trace obtained from $\tau$ by
% % % % replacing $\mathtt{d}_{11}^1$ with a value less or equal than 10 is a trace
% % % % that satisfies the desired requirement. In this sense, the constant
% % % % $\mathtt{d}_{11}^1$ is an \emph{abstraction} of the interval $(-\infty, 10]$
% % % % and $\mathtt{d}_{11}^2$ is an abstraction for $(10, +\infty]$
% % % %
% % % % The approach can be generalized to the case of multiple conditions.
% % % % To see how, assume there are two conditions involving overlapping intervals,
% % % % e.g., $att_{ij}\leq 10$ and $att_{lk}\leq 8$. In this case, we need to provide
% % % % an abstraction for all of the possible intervals defined by the mentioned
% % % % constants. We proceed as follows:
% % % % \begin{itemize}
% % % % 	\item We identify the  intervals of
% % % % interest, i.e,: $I_1=(-\infty,8]$, $I_2=(8,10]$, and $I_3=(10,+\infty)$.
% % % % 	\item For
% % % % each interval $I_m$, we introduce one new constant $\mathtt{d}^m$
% % % % abstracting the interval.
% % % % 	\item We introduce one predicate per interval,
% % % % namely:
% % % % $lteq8$, $gt8lteq10$, $gt10$.
% % % % 	\item We add sentences~(\ref{eq1}) requiring that both $D_{ij}$ and
% % % % $D_{lk}$ include exactly all of the constants $\mathtt{d}^m$ (thus match)
% % % % abstracting the intervals.
% % % % 	\item We add the following sentence, which expresses the relationship
% % % % between the values in the intervals:
% % % % $$
% % % % \begin{array}{l}
% % % % 	lteq8(\mathtt{d}^1)\land
% % % % 	\lnot gt8lteq10(\mathtt{d}^1)\land
% % % % 	\lnot gt10(\mathtt{d}^1)\land\\
% % % % 	\lnot lteq8(\mathtt{d}^2)\land	
% % % % 	gt8lteq10(\mathtt{d}^2)\land
% % % % 	\lnot gt10(\mathtt{d}^2)\land\\		
% % % % 	\lnot lteq8(\mathtt{d}^3)\land
% % % % 	\lnot gt8lteq10(\mathtt{d}^3)\land
% % % % 	gt10(\mathtt{d}^3)
% % % % \end{array}
% % % % $$
% % % % \item Finally, we use the relevant predicate to express that an attribute falls
% % % % into an interval. E.g., to express that $att_{ij}\leq 10$, we write
% % % % $\leq10(att(\mathtt{att}_{ij}))$.	
% % % % \end{itemize}
% % % % %%
% % % % The case where more conditions occur, or other relational operators, e.g.,
% % % % $=$, $\neq$, $>$, etc., are used is essentially analogous. Notice that,
% % % % differently from the case where no numeric values occur, in order to write the
% % % % theory, one needs to know in advance the Declare model.
% % % %
% % % %
% % % %
% % % % % \section{Solving Process Mining Problems with SAT}
% % % % %
% % % % % We now show how the various problems typical in Process Mining can be solved
% % % % % using the SAT technology. As explained above, we rely on existing tools that,
% % % % % given a FO theory and a bound on the size of its models, finds a model of the
% % % % % theory by compiling the problem into SAT. We consider such tools as black-boxes
% % % % % and do not enter the details of how the compilation is carried out. For this
% % % % % reason, we describe our approach by simply reporting the theory we use for each
% % % % % problem and how we interpret the obtained model.
% % % % %
% % % % %
% % % % % \subsection{Log Generation}
% % % % % Log generation is the problem of generating a set of traces, i.e., a log,
% % % % % compliant with a set of Declare templates. We can cast this problem to that of
% % % % % searching for a set of models of a theory. The idea is that each of the
% % % % % obtained models represents a trace compliant with the Declare model. To this
% % % % % end, assuming that the activities, the attributes and their respective domains
% % % % % are fixed, we proceed as follows:
% % % % %
% % % % % \begin{enumerate}
% % % % % \item we vary the size $b$ according to the desired length of the traces in
% % % % % the log;
% % % % % \item we produce the set $\C$ of sentences over $\sigma$ that correspond to
% % % % % the MP Declare constraints the traces must satisfy;
% % % % % \item we find as many models as desired of the theory
% % % % % $\Gamma\cup\C$.
% % % % % \end{enumerate}
% % % % %
% % % % % \subsection{Query Checking}
% % % % % Consider a trace $\tau=\tau_1,\ldots,\tau_n$ and a Declare template $\phi$. In
% % % % % general, templates have the form $\phi(X)$ or $\phi(X,Y)$, where $X$ and $Y$
% % % % % are activities to be instantiated depending on the requirements one needs to
% % % % % satisfy.
% % % % % %%
% % % % % While typically one deals with fully instantiated activities,
% % % % %
% % % % % Query checking is the problem of
% % % % %
% % % % %
% % % % % \subsection{Conformance Checking}
% % % % % Conformance checking is the problem of checking whether a given log, i.e., a
% % % % % set of traces, satisfies a Declare model. Here, we focus on the problem of
% % % % % checking only one trace, as the whole log can be checked by simply repeating
% % % % % the process for each trace.
% % % % %
% % % % % The problem can be easily expressed as a variant of log generation.
% % % % % The idea is to use the same theory as that for log generation, with the addition
% % % % % of a set of sentences constraining the trace to match the input trace. In this
% % % % % way, if the input trace satisfies the Declare constraints, then the theory has
% % % % % exactly one model, the input trace itself, which is then returned. If, instead,
% % % % % the trace does not satisfy the constraints, then the theory is inconsistent and
% % % % % no model is returned.
% % % % %
% % % % % In details, let $\tau$ be the input trace, with the $i$-th event consisting of
% % % % % an activity $a^i$ (we use superscripts to indicate the position in $\tau$),
% % % % % with associated attributes $att^i_1,\ldots,att^i_{\ell_i}$, with assigned
% % % % % values, respectively, $d^i_1,\ldots,d^i_\ell$. fix $b$ to the length of the
% % % % % input trace. We write the theory $\Gamma$ and the set of sentences
% % % % % $\C$ corresponding to the Declare constraints, as done for the case of log
% % % % % generation and, for every
% % % % % $i=1,\ldots,b$, add the following sentences:
% % % % % \begin{itemize}
% % % % % 	\item $act(\mathtt{e}_i)=\mathtt{a}^i$;
% % % % % 	\item for every $j=1,\ldots, \ell_i$,
% % % % % 			$att(\mathtt{e}_i,\mathtt{att}^i_j)=\mathtt{d}^i_j$.
% % % % % \end{itemize}
% % % % % %%
% % % % % Then, to check whether $\tau$ satisfies the Declare model, it is then enough to
% % % % % search for a model of the above theory. $\tau$ conforms to
% % % % % the Declare model if and only if some model exists.
% % % % %
% % % % %
% % % % %



\subsection{Conformance Checking}\label{subseq:alloy:conf-check}
To solve conformance checking in Alloy, we proceed as follows.
For every trace $\tau$ in the input set $L$, we write a theory
$\Gamma_\tau$ consisting of a set of FO sentences over $\sigma$,
which admits, for a suitable fixed interpretation domain $\Delta$, a
single model $\I_\tau$ which is also a model of $\Gamma$
corresponding to $\tau$. Once done so, we add the following
sentence to $\Gamma_\tau$, where $R$ is a $0$-ary relation
(i.e., a proposition):
$$\phi\doteq R\leftrightarrow \bigwedge_{\varphi\in \Phi} \varphi,$$
where $\Phi$ is the set of input formulas to be checked.
%%
Then, we run Alloy on the resulting theory, i.e., $\Gamma_\tau$
plus $\phi$, and the fixed interpretation domain $\Delta$,
and conclude that the trace $\tau$ satisfy
the input specification $\Phi$ iff the theory admits a model
where $R$ is true (by construction, if such a model exists
is unique).

\begin{example}\label{ex:alloy-c-check}
As an example, consider Ex.~\ref{ex:3} and trace $\tau$ of
Ex.~\ref{ex:1}.
The theory $\Gamma_\tau$ includes the following sentences:
\begin{itemize}
	\item $
		Seq(0,\cst a_1, 2)\land
		Seq(1,\cst a_2, 5)\land
		Seq(2,\cst a_3, 9)\land
		Seq(3,\cst a_3, 20)$
		
	\item $
		AttV(0,\cst v, \cst a)\land
		AttV(1,\cst w, \cst b)\land
		AttV(1,\cst y, \cst d)\land
		AttV(2,\cst s, \cst c)\land
		AttV(3,\cst s, \cst d)$
		
	\item $N(0)\land
			N(1)\land
			N(2)\land
			N(3)\land
			N(5)\land
			N(9)\land
			N(20)$		
			
	\item $\forall i,a,t.Seq(i,a,t)\rightarrow
		(i=0\land a= \cst a_1\land 2)\lor
		(i=1\land a= \cst a_2\land 5)\lor
		(i=2\land a= \cst a_3\land 9)\lor
		(i=3\land a= \cst a_3\land 20)$
		
	\item sentences analogous to the one above for
	$AttV$ and $N$ (omitted for brevity).
\end{itemize}
%%
For fixed $\Delta=\C=
			\set{0,1,2,3,5,9,20,
					\cst{a_1},\cst{a_2},\cst{a_3},
					\cst{v},\cst{w}, \cst{y},\cst{s},
					\cst{a},\cst{b},\cst{c},\cst{d}
				}
		$,
		it can be seen that the (only) model of $\Gamma_\tau$
		is also one for $\Gamma$ and represents trace
$\tau=\langle\cst{a}_1[
	\cst{t}=2,
	\cst{v}=\cst{a}
],
\cst{a}_2[
	\cst{t}=5,
	\cst{w}=\cst{b},
	\cst{y}=\cst{d}
],
\cst{a}_3[
	\cst{t}=9,
	\cst{s}=\cst{c}
],
\cst{a}_3[
	\cst{t}=20,
	\cst{s}=\cst{d}
]\rangle$.

Then, we define
$\phi=\textsc{Existence}(\cst a_2,\varphi)\land
\textsc{Response}(\cst a_1,\varphi_a,\cst a_2,\varphi_c)$,
for $\varphi$, $\varphi_a$, and $\varphi_c$ defined as
in Ex.~\ref{ex:3}.
%%
Finally, with Alloy, we search for a model of
$\Gamma_\tau\cup\set{\phi}$, giving $\Delta$ in input to Alloy.
In the case at hand, the returned model (there exists
exactly one) will have $R$ true, thus we conclude that
$\tau$ satisfies $\Phi$.
\end{example}



\subsection{Query Checking}
For query checking, we proceed in a way similar to
conformance checking (in fact, from a general point of
view, conformance checking can be seen as a special case
of query checking where the query has no free
variables, i.e., is boolean).

Given a trace $\tau\in L$, we produce the associated theory
(admitting one model only) $\Gamma_\tau$, as explained in
Sec.~\ref{subseq:alloy:conf-check}. Then, given the
query $\varphi(\vec x)$ with free variables
$\vec x$, we first add a new predicate symbol $Ans$
with arity $\card{\vec x}$ to $\sigma$, and then
add the following sentence to $\Gamma_\tau$:
$\forall \vec x.Ans(\vec x)\leftrightarrow \varphi(\vec x)$.
As a result, the (unique) model of the resulting theory
$\Gamma_\tau\cup\set{\varphi}$ is essentially the
same as that of $\Gamma$ but includes the additional relation
$Ans$ containing the tuples included in $\varphi^\I$.

Therefore, by feeding Alloy with the input consisting in
the theory $\Gamma_\tau\cup\set{\varphi}$ and the
interpretation domain associated with $\tau$, we can easily
answer the query.

\begin{example}
For an example of query checking, consider the problem
defined in Ex.~\ref{ex:q-check}, where the query
%%
$$\varphi=\forall i,t.(
			\forall\vec x.(
				(Seq(i,?x,t)\land \varphi_a)
					\rightarrow
$$

\vspace{-5mm}
$$
				(\exists i',t'.i'\geq i\land(
					\exists\vec x'.
						Seq(i',?y,t')\land\varphi_c)
				)
			)
		)
		$$
is evaluated against
trace~$\tau$ of Ex.~\ref{ex:1}:
$\tau=\langle\cst{a}_1[
	\cst{t}=2,
	\cst{v}=\cst{a}
],
\cst{a}_2[
	\cst{t}=5,
	\cst{w}=\cst{b},
	\cst{y}=\cst{d}
],
\cst{a}_3[
	\cst{t}=9,
	\cst{s}=\cst{c}
],
\cst{a}_3[
	\cst{t}=20,
	\cst{s}=\cst{d}
]\rangle$.

To answer the query, we simply feed Alloy with the theory
$\Gamma_\tau$ shown in Ex.~\ref{ex:alloy-c-check}
union the sentence
$\forall \vec x.Ans(\vec x)\leftrightarrow \varphi$,
and with the interpretation domain $\Delta$ associated with
$\tau$, and then ask Alloy to find a model of the
input theory.
The content of relation $Ans$ corresponds to the answer to
$\varphi$.
\end{example}


	
\section{Experiments}
We implemented the framework described in this paper in a tool available at \url{https://tinyurl.com/y5773gdz}. The efficiency of the tool was tested by performing several experiments measuring the execution times needed for the tool to perform tasks of different complexity.
Each measured execution time was averaged over 3 runs. The experiments were conducted on a single core of a 64-bit 2.2 Ghz Intel Core i5-5200U processor with 16GB of RAM. The SAT solver used was SAT4J.

In order to assess the performance of the log generation approach, we generated logs with different characteristics using different Declare models.
In a first set of experiments, we sampled the generation times that resulted by varying the following parameters:
\begin{inparaenum}[\itshape(i)\upshape]
\item the number of constraints in the Declare model (5, 10, 15, and 20 constraints),
\item the number of events per trace (5, 10, 15, 20, 25, 30, 40, and 50 events), and
\item the number of traces in the log (100, 250, 500, 1000, 2500, 5000, and 10\,000 traces).
\end{inparaenum}
The number of activities in the Declare models is fixed to 10. The constraints employed are all of type \emph{response}. The constraints are considered without data conditions, with activation conditions, and with activation and correlation conditions. 
%The results of our experimentation are summarized in Tables~1-7.
For space limitations, we cannot include the details of the wide experimentation carried out. In the following, we summarize the main findings.\footnote{The reader is referred to \url{https://tinyurl.com/wqmkwqx} for the detailed results of the experimentation.}

%\pgfplotstableread[col sep=comma,header=true]{experiments/5-CONSTRAINTS-NO-DATA.csv}\fileFIVEconstNODATA

\pgfplotstablecreatecol[copy column from table={\fileFIVEconstNODATA}{[index] 0}] {logSizeNODATAct5} {\fileFIVEconstNODATA}
\pgfplotstablecreatecol[copy column from table={\fileFIVEconstNODATA}{[index] 1}] {tr5NODATAct5} {\fileFIVEconstNODATA}
\pgfplotstablecreatecol[copy column from table={\fileFIVEconstNODATA}{[index] 2}] {tr10NODATAct5} {\fileFIVEconstNODATA}
\pgfplotstablecreatecol[copy column from table={\fileFIVEconstNODATA}{[index] 3}] {tr15NODATAct5} {\fileFIVEconstNODATA}
\pgfplotstablecreatecol[copy column from table={\fileFIVEconstNODATA}{[index] 4}] {tr20NODATAct5} {\fileFIVEconstNODATA}
\pgfplotstablecreatecol[copy column from table={\fileFIVEconstNODATA}{[index] 5}] {tr25NODATAct5} {\fileFIVEconstNODATA}
\pgfplotstablecreatecol[copy column from table={\fileFIVEconstNODATA}{[index] 6}] {tr30NODATAct5} {\fileFIVEconstNODATA}
\pgfplotstablecreatecol[copy column from table={\fileFIVEconstNODATA}{[index] 7}] {tr40NODATAct5} {\fileFIVEconstNODATA}
\pgfplotstablecreatecol[copy column from table={\fileFIVEconstNODATA}{[index] 8}] {tr50NODATAct5} {\fileFIVEconstNODATA}

\pgfplotstableread[col sep=comma,header=true]{experiments/10-CONSTRAINTS-NO-DATA.csv}\fileTENconstNODATA

\pgfplotstablecreatecol[copy column from table={\fileTENconstNODATA}{[index] 0}] {logSizeNODATAct10} {\fileTENconstNODATA}
\pgfplotstablecreatecol[copy column from table={\fileTENconstNODATA}{[index] 1}] {tr5NODATAct10} {\fileTENconstNODATA}
\pgfplotstablecreatecol[copy column from table={\fileTENconstNODATA}{[index] 2}] {tr10NODATAct10} {\fileTENconstNODATA}
\pgfplotstablecreatecol[copy column from table={\fileTENconstNODATA}{[index] 3}] {tr15NODATAct10} {\fileTENconstNODATA}
\pgfplotstablecreatecol[copy column from table={\fileTENconstNODATA}{[index] 4}] {tr20NODATAct10} {\fileTENconstNODATA}
\pgfplotstablecreatecol[copy column from table={\fileTENconstNODATA}{[index] 5}] {tr25NODATAct10} {\fileTENconstNODATA}
\pgfplotstablecreatecol[copy column from table={\fileTENconstNODATA}{[index] 6}] {tr30NODATAct10} {\fileTENconstNODATA}
\pgfplotstablecreatecol[copy column from table={\fileTENconstNODATA}{[index] 7}] {tr40NODATAct10}  {\fileTENconstNODATA}
\pgfplotstablecreatecol[copy column from table={\fileTENconstNODATA}{[index] 8}] {tr50NODATAct10}  {\fileTENconstNODATA}

\pgfplotstableread[col sep=comma,header=true]{experiments/15-CONSTRAINTS-NO-DATA.csv}\fileFIFTEENconstNODATA

\pgfplotstablecreatecol[copy column from table={\fileFIFTEENconstNODATA}{[index] 0}] {logSizeNODATAct15} {\fileFIFTEENconstNODATA}
\pgfplotstablecreatecol[copy column from table={\fileFIFTEENconstNODATA}{[index] 1}] {tr5NODATAct15} {\fileFIFTEENconstNODATA}
\pgfplotstablecreatecol[copy column from table={\fileFIFTEENconstNODATA}{[index] 2}] {tr10NODATAct15} {\fileFIFTEENconstNODATA}
\pgfplotstablecreatecol[copy column from table={\fileFIFTEENconstNODATA}{[index] 3}] {tr15NODATAct15} {\fileFIFTEENconstNODATA}
\pgfplotstablecreatecol[copy column from table={\fileFIFTEENconstNODATA}{[index] 4}] {tr20NODATAct15} {\fileFIFTEENconstNODATA}
\pgfplotstablecreatecol[copy column from table={\fileFIFTEENconstNODATA}{[index] 5}] {tr25NODATAct15} {\fileFIFTEENconstNODATA}
\pgfplotstablecreatecol[copy column from table={\fileFIFTEENconstNODATA}{[index] 6}] {tr30NODATAct15} {\fileFIFTEENconstNODATA}
\pgfplotstablecreatecol[copy column from table={\fileFIFTEENconstNODATA}{[index] 7}] {tr40NODATAct15}  {\fileFIFTEENconstNODATA}
\pgfplotstablecreatecol[copy column from table={\fileFIFTEENconstNODATA}{[index] 8}] {tr50NODATAct15}  {\fileFIFTEENconstNODATA}

\pgfplotstableread[col sep=comma,header=true]{experiments/20-CONSTRAINTS-NO-DATA.csv}\fileTWENTYconstNODATA

\pgfplotstablecreatecol[copy column from table={\fileTWENTYconstNODATA}{[index] 0}] {logSizeNODATAct20} {\fileTWENTYconstNODATA}
\pgfplotstablecreatecol[copy column from table={\fileTWENTYconstNODATA}{[index] 1}] {tr5NODATAct20} {\fileTWENTYconstNODATA}
\pgfplotstablecreatecol[copy column from table={\fileTWENTYconstNODATA}{[index] 2}] {tr10NODATAct20} {\fileTWENTYconstNODATA}
\pgfplotstablecreatecol[copy column from table={\fileTWENTYconstNODATA}{[index] 3}] {tr15NODATAct20} {\fileTWENTYconstNODATA}
\pgfplotstablecreatecol[copy column from table={\fileTWENTYconstNODATA}{[index] 4}] {tr20NODATAct20} {\fileTWENTYconstNODATA}
\pgfplotstablecreatecol[copy column from table={\fileTWENTYconstNODATA}{[index] 5}] {tr25NODATAct20} {\fileTWENTYconstNODATA}
\pgfplotstablecreatecol[copy column from table={\fileTWENTYconstNODATA}{[index] 6}] {tr30NODATAct20} {\fileTWENTYconstNODATA}
\pgfplotstablecreatecol[copy column from table={\fileTWENTYconstNODATA}{[index] 7}] {tr40NODATAct20}  {\fileTWENTYconstNODATA}
\pgfplotstablecreatecol[copy column from table={\fileTWENTYconstNODATA}{[index] 8}] {tr50NODATAct20}  {\fileTWENTYconstNODATA}

%\begin{small}
\begin{table}[b!]
\begin{center}
\resizebox{\columnwidth}{!}{
\newcolumntype{C}{>{\centering\arraybackslash}p{21mm}}
\pgfplotstabletypeset[
font=\normalsize,
	every head row/.style={
    output empty row,
		before row={%
              \toprule
              Trace length $\rightarrow$
            & 5
			& 10
            & 15
            & 20
            & 25
            & 30
            & 40
            & 50
            \\
		},
		after row={%
            \midrule
              \multicolumn{1}{c|}{\textbf{\emph{Log Size}} $\downarrow$}
            & \multicolumn{8}{c|}{\textbf{\emph{5 constraints}}}
            \\\midrule
		},
	},
	%
	columns={logSizeNODATAct5,tr5NODATAct5,tr10NODATAct5,tr15NODATAct5,tr20NODATAct5,tr25NODATAct5,tr30NODATAct5,
tr40NODATAct5,tr50NODATAct5},	
    columns/{logSizeNODATAct5}/.style={column name={},column type=C},	
    columns/{tr5NODATAct5}/.style={column name={},column type=C},
    columns/{tr10NODATAct5}/.style={column name={},column type=C},
    columns/{tr15NODATAct5}/.style={column name={},column type=C},
    columns/{tr20NODATAct5}/.style={column name={},column type=C},
    columns/{tr25NODATAct5}/.style={column name={},column type=C},
	columns/{tr30NODATAct5}/.style={column name={},column type=C},
    columns/{tr40NODATAct5}/.style={column name={},column type=C},
	columns/{tr50NODATAct5}/.style={column name={},column type=C},
	%%	
    %%every row 0 column 0/.style={postproc cell content/.style={@cell content={3-50}}},
    %%every row 1 column 0/.style={postproc cell content/.style={@cell content={51-75}}},
    %%every row 2 column 0/.style={postproc cell content/.style={@cell content={76-100}}},
    %%every row 3 column 0/.style={postproc cell content/.style={@cell content={101-128}}},
]{\fileFIVEconstNODATA}
}
%%%
%%%
\resizebox{\columnwidth}{!}{
\newcolumntype{C}{>{\centering\arraybackslash}p{21mm}}
\pgfplotstabletypeset[
font=\normalsize,
	every head row/.style={
    output empty row,
		after row={%
            \midrule
              \multicolumn{1}{c|}{\textbf{\emph{Log Size}} $\downarrow$}
            & \multicolumn{8}{c|}{\textbf{\emph{10 constraints}}}
            \\\midrule
		},
	},
	%
	columns={logSizeNODATAct10,tr5NODATAct10,tr10NODATAct10,tr15NODATAct10,tr20NODATAct10,tr25NODATAct10,tr30NODATAct10,
tr40NODATAct10,tr50NODATAct10},	
    columns/{logSizeNODATAct10}/.style={column name={},column type=C},	
    columns/{tr5NODATAct10}/.style={column name={},column type=C},
    columns/{tr10NODATAct10}/.style={column name={},column type=C},
    columns/{tr15NODATAct10}/.style={column name={},column type=C},
    columns/{tr20NODATAct10}/.style={column name={},column type=C},
    columns/{tr25NODATAct10}/.style={column name={},column type=C},
	columns/{tr30NODATAct10}/.style={column name={},column type=C},
    columns/{tr40NODATAct10}/.style={column name={},column type=C},
	columns/{tr50NODATAct10}/.style={column name={},column type=C},
	%%	
    %%every row 0 column 0/.style={postproc cell content/.style={@cell content={3-50}}},
    %%every row 1 column 0/.style={postproc cell content/.style={@cell content={51-75}}},
    %%every row 2 column 0/.style={postproc cell content/.style={@cell content={76-100}}},
    %%every row 3 column 0/.style={postproc cell content/.style={@cell content={101-128}}},
]{\fileTENconstNODATA}
}
%%%
%%%
\resizebox{\columnwidth}{!}{
\newcolumntype{C}{>{\centering\arraybackslash}p{21mm}}
\pgfplotstabletypeset[
font=\normalsize,
	every head row/.style={
    output empty row,
		after row={%
            \midrule
              \multicolumn{1}{c|}{\textbf{\emph{Log Size}} $\downarrow$}
            & \multicolumn{8}{c|}{\emph{\textbf{15 constraints}}}
            \\\midrule
		},
	},
	%
	columns={logSizeNODATAct15,tr5NODATAct15,tr10NODATAct15,tr15NODATAct15,tr20NODATAct15,tr25NODATAct15,tr30NODATAct15,
tr40NODATAct15,tr50NODATAct15},	
    columns/{logSizeNODATAct15}/.style={column name={},column type=C},	
    columns/{tr5NODATAct15}/.style={column name={},column type=C},
    columns/{tr10NODATAct15}/.style={column name={},column type=C},
    columns/{tr15NODATAct15}/.style={column name={},column type=C},
    columns/{tr20NODATAct15}/.style={column name={},column type=C},
    columns/{tr25NODATAct15}/.style={column name={},column type=C},
	columns/{tr30NODATAct15}/.style={column name={},column type=C},
    columns/{tr40NODATAct15}/.style={column name={},column type=C},
	columns/{tr50NODATAct15}/.style={column name={},column type=C},
	%%	
]{\fileFIFTEENconstNODATA}
}
%%%
%%%
\resizebox{\columnwidth}{!}{
\newcolumntype{C}{>{\centering\arraybackslash}p{21mm}}
\pgfplotstabletypeset[
font=\normalsize,
	every head row/.style={
    output empty row,
		after row={%
            \midrule
              \multicolumn{1}{c|}{\textbf{\emph{Log Size}} $\downarrow$}
            & \multicolumn{8}{c|}{\emph{\textbf{20 constraints}}}
            \\\midrule
		},
	},
	%
	columns={logSizeNODATAct20,tr5NODATAct20,tr10NODATAct20,tr15NODATAct20,tr20NODATAct20,tr25NODATAct20,tr30NODATAct20,
tr40NODATAct20,tr50NODATAct20},	
    columns/{logSizeNODATAct20}/.style={column name={},column type=C},	
    columns/{tr5NODATAct20}/.style={column name={},column type=C},
    columns/{tr10NODATAct20}/.style={column name={},column type=C},
    columns/{tr15NODATAct20}/.style={column name={},column type=C},
    columns/{tr20NODATAct20}/.style={column name={},column type=C},
    columns/{tr25NODATAct20}/.style={column name={},column type=C},
	columns/{tr30NODATAct20}/.style={column name={},column type=C},
    columns/{tr40NODATAct20}/.style={column name={},column type=C},
	columns/{tr50NODATAct20}/.style={column name={},column type=C},
	%%	
]{\fileTWENTYconstNODATA}
}
\caption{Time (in ms.) required for generating logs of different sizes containing traces of different lengths from Declare models of increasing sizes (without data conditions).}
\label{table:exp_results_real}
\end{center}
\end{table}
%\end{small} 
%\pgfplotstableread[col sep=comma,header=true]{experiments/5-CONSTRAINTS-DATA-ACT.csv}\fileFIVEconstDATAACT

\pgfplotstablecreatecol[copy column from table={\fileFIVEconstDATAACT}{[index] 0}] {logSizeDATAACTct5} {\fileFIVEconstDATAACT}
\pgfplotstablecreatecol[copy column from table={\fileFIVEconstDATAACT}{[index] 1}] {tr5DATAACTct5} {\fileFIVEconstDATAACT}
\pgfplotstablecreatecol[copy column from table={\fileFIVEconstDATAACT}{[index] 2}] {tr10DATAACTct5} {\fileFIVEconstDATAACT}
\pgfplotstablecreatecol[copy column from table={\fileFIVEconstDATAACT}{[index] 3}] {tr15DATAACTct5} {\fileFIVEconstDATAACT}
\pgfplotstablecreatecol[copy column from table={\fileFIVEconstDATAACT}{[index] 4}] {tr20DATAACTct5} {\fileFIVEconstDATAACT}
\pgfplotstablecreatecol[copy column from table={\fileFIVEconstDATAACT}{[index] 5}] {tr25DATAACTct5} {\fileFIVEconstDATAACT}
\pgfplotstablecreatecol[copy column from table={\fileFIVEconstDATAACT}{[index] 6}] {tr30DATAACTct5} {\fileFIVEconstDATAACT}
\pgfplotstablecreatecol[copy column from table={\fileFIVEconstDATAACT}{[index] 7}] {tr40DATAACTct5} {\fileFIVEconstDATAACT}
\pgfplotstablecreatecol[copy column from table={\fileFIVEconstDATAACT}{[index] 8}] {tr50DATAACTct5} {\fileFIVEconstDATAACT}

\pgfplotstableread[col sep=comma,header=true]{experiments/10-CONSTRAINTS-DATA-ACT.csv}\fileTENconstDATAACT

\pgfplotstablecreatecol[copy column from table={\fileTENconstDATAACT}{[index] 0}] {logSizeDATAACTct10} {\fileTENconstDATAACT}
\pgfplotstablecreatecol[copy column from table={\fileTENconstDATAACT}{[index] 1}] {tr5DATAACTct10} {\fileTENconstDATAACT}
\pgfplotstablecreatecol[copy column from table={\fileTENconstDATAACT}{[index] 2}] {tr10DATAACTct10} {\fileTENconstDATAACT}
\pgfplotstablecreatecol[copy column from table={\fileTENconstDATAACT}{[index] 3}] {tr15DATAACTct10} {\fileTENconstDATAACT}
\pgfplotstablecreatecol[copy column from table={\fileTENconstDATAACT}{[index] 4}] {tr20DATAACTct10} {\fileTENconstDATAACT}
\pgfplotstablecreatecol[copy column from table={\fileTENconstDATAACT}{[index] 5}] {tr25DATAACTct10} {\fileTENconstDATAACT}
\pgfplotstablecreatecol[copy column from table={\fileTENconstDATAACT}{[index] 6}] {tr30DATAACTct10} {\fileTENconstDATAACT}
\pgfplotstablecreatecol[copy column from table={\fileTENconstDATAACT}{[index] 7}] {tr40DATAACTct10}  {\fileTENconstDATAACT}
\pgfplotstablecreatecol[copy column from table={\fileTENconstDATAACT}{[index] 8}] {tr50DATAACTct10}  {\fileTENconstDATAACT}

\pgfplotstableread[col sep=comma,header=true]{experiments/15-CONSTRAINTS-DATA-ACT.csv}\fileFIFTEENconstDATAACT

\pgfplotstablecreatecol[copy column from table={\fileFIFTEENconstDATAACT}{[index] 0}] {logSizeDATAACTct15} {\fileFIFTEENconstDATAACT}
\pgfplotstablecreatecol[copy column from table={\fileFIFTEENconstDATAACT}{[index] 1}] {tr5DATAACTct15} {\fileFIFTEENconstDATAACT}
\pgfplotstablecreatecol[copy column from table={\fileFIFTEENconstDATAACT}{[index] 2}] {tr10DATAACTct15} {\fileFIFTEENconstDATAACT}
\pgfplotstablecreatecol[copy column from table={\fileFIFTEENconstDATAACT}{[index] 3}] {tr15DATAACTct15} {\fileFIFTEENconstDATAACT}
\pgfplotstablecreatecol[copy column from table={\fileFIFTEENconstDATAACT}{[index] 4}] {tr20DATAACTct15} {\fileFIFTEENconstDATAACT}
\pgfplotstablecreatecol[copy column from table={\fileFIFTEENconstDATAACT}{[index] 5}] {tr25DATAACTct15} {\fileFIFTEENconstDATAACT}
\pgfplotstablecreatecol[copy column from table={\fileFIFTEENconstDATAACT}{[index] 6}] {tr30DATAACTct15} {\fileFIFTEENconstDATAACT}
\pgfplotstablecreatecol[copy column from table={\fileFIFTEENconstDATAACT}{[index] 7}] {tr40DATAACTct15}  {\fileFIFTEENconstDATAACT}
\pgfplotstablecreatecol[copy column from table={\fileFIFTEENconstDATAACT}{[index] 8}] {tr50DATAACTct15}  {\fileFIFTEENconstDATAACT}

\pgfplotstableread[col sep=comma,header=true]{experiments/20-CONSTRAINTS-DATA-ACT.csv}\fileTWENTYconstDATAACT

\pgfplotstablecreatecol[copy column from table={\fileTWENTYconstDATAACT}{[index] 0}] {logSizeDATAACTct20} {\fileTWENTYconstDATAACT}
\pgfplotstablecreatecol[copy column from table={\fileTWENTYconstDATAACT}{[index] 1}] {tr5DATAACTct20} {\fileTWENTYconstDATAACT}
\pgfplotstablecreatecol[copy column from table={\fileTWENTYconstDATAACT}{[index] 2}] {tr10DATAACTct20} {\fileTWENTYconstDATAACT}
\pgfplotstablecreatecol[copy column from table={\fileTWENTYconstDATAACT}{[index] 3}] {tr15DATAACTct20} {\fileTWENTYconstDATAACT}
\pgfplotstablecreatecol[copy column from table={\fileTWENTYconstDATAACT}{[index] 4}] {tr20DATAACTct20} {\fileTWENTYconstDATAACT}
\pgfplotstablecreatecol[copy column from table={\fileTWENTYconstDATAACT}{[index] 5}] {tr25DATAACTct20} {\fileTWENTYconstDATAACT}
\pgfplotstablecreatecol[copy column from table={\fileTWENTYconstDATAACT}{[index] 6}] {tr30DATAACTct20} {\fileTWENTYconstDATAACT}
\pgfplotstablecreatecol[copy column from table={\fileTWENTYconstDATAACT}{[index] 7}] {tr40DATAACTct20}  {\fileTWENTYconstDATAACT}
\pgfplotstablecreatecol[copy column from table={\fileTWENTYconstDATAACT}{[index] 8}] {tr50DATAACTct20}  {\fileTWENTYconstDATAACT}

\begin{small}
\begin{table}[t!]
\begin{center}
\resizebox{\columnwidth}{!}{
\newcolumntype{C}{>{\centering\arraybackslash}p{21mm}}
\pgfplotstabletypeset[
font=\normalsize,
	every head row/.style={
    output empty row,
		before row={%
              \toprule
              Trace length $\rightarrow$
            & 5
			& 10
            & 15
            & 20
            & 25
            & 30
            & 40
            & 50
            \\
		},
		after row={%
            \midrule
              \multicolumn{1}{c|}{\textbf{\emph{Log Size}} $\downarrow$}
            & \multicolumn{8}{c|}{\textbf{\emph{5 constraints}}}
            \\\midrule
		},
	},
	%
	columns={logSizeDATAACTct5,tr5DATAACTct5,tr10DATAACTct5,tr15DATAACTct5,tr20DATAACTct5,tr25DATAACTct5,tr30DATAACTct5,
tr40DATAACTct5,tr50DATAACTct5},	
    columns/{logSizeDATAACTct5}/.style={column name={},column type=C},	
    columns/{tr5DATAACTct5}/.style={column name={},column type=C},
    columns/{tr10DATAACTct5}/.style={column name={},column type=C},
    columns/{tr15DATAACTct5}/.style={column name={},column type=C},
    columns/{tr20DATAACTct5}/.style={column name={},column type=C},
    columns/{tr25DATAACTct5}/.style={column name={},column type=C},
	columns/{tr30DATAACTct5}/.style={column name={},column type=C},
    columns/{tr40DATAACTct5}/.style={column name={},column type=C},
	columns/{tr50DATAACTct5}/.style={column name={},column type=C},
	%%	
    %%every row 0 column 0/.style={postproc cell content/.style={@cell content={3-50}}},
    %%every row 1 column 0/.style={postproc cell content/.style={@cell content={51-75}}},
    %%every row 2 column 0/.style={postproc cell content/.style={@cell content={76-100}}},
    %%every row 3 column 0/.style={postproc cell content/.style={@cell content={101-128}}},
]{\fileFIVEconstDATAACT}
}
%%%
%%%
\resizebox{\columnwidth}{!}{
\newcolumntype{C}{>{\centering\arraybackslash}p{21mm}}
\pgfplotstabletypeset[
font=\normalsize,
	every head row/.style={
    output empty row,
		after row={%
            \midrule
              \multicolumn{1}{c|}{\textbf{\emph{Log Size}} $\downarrow$}
            & \multicolumn{8}{c|}{\textbf{\emph{10 constraints}}}
            \\\midrule
		},
	},
	%
	columns={logSizeDATAACTct10,tr5DATAACTct10,tr10DATAACTct10,tr15DATAACTct10,tr20DATAACTct10,tr25DATAACTct10,tr30DATAACTct10,
tr40DATAACTct10,tr50DATAACTct10},	
    columns/{logSizeDATAACTct10}/.style={column name={},column type=C},	
    columns/{tr5DATAACTct10}/.style={column name={},column type=C},
    columns/{tr10DATAACTct10}/.style={column name={},column type=C},
    columns/{tr15DATAACTct10}/.style={column name={},column type=C},
    columns/{tr20DATAACTct10}/.style={column name={},column type=C},
    columns/{tr25DATAACTct10}/.style={column name={},column type=C},
	columns/{tr30DATAACTct10}/.style={column name={},column type=C},
    columns/{tr40DATAACTct10}/.style={column name={},column type=C},
	columns/{tr50DATAACTct10}/.style={column name={},column type=C},
	%%	
    %%every row 0 column 0/.style={postproc cell content/.style={@cell content={3-50}}},
    %%every row 1 column 0/.style={postproc cell content/.style={@cell content={51-75}}},
    %%every row 2 column 0/.style={postproc cell content/.style={@cell content={76-100}}},
    %%every row 3 column 0/.style={postproc cell content/.style={@cell content={101-128}}},
]{\fileTENconstDATAACT}
}
%%%
%%%
\resizebox{\columnwidth}{!}{
\newcolumntype{C}{>{\centering\arraybackslash}p{21mm}}
\pgfplotstabletypeset[
font=\normalsize,
	every head row/.style={
    output empty row,
		after row={%
            \midrule
              \multicolumn{1}{c|}{\textbf{\emph{Log Size}} $\downarrow$}
            & \multicolumn{8}{c|}{\emph{\textbf{15 constraints}}}
            \\\midrule
		},
	},
	%
	columns={logSizeDATAACTct15,tr5DATAACTct15,tr10DATAACTct15,tr15DATAACTct15,tr20DATAACTct15,tr25DATAACTct15,tr30DATAACTct15,
tr40DATAACTct15,tr50DATAACTct15},	
    columns/{logSizeDATAACTct15}/.style={column name={},column type=C},	
    columns/{tr5DATAACTct15}/.style={column name={},column type=C},
    columns/{tr10DATAACTct15}/.style={column name={},column type=C},
    columns/{tr15DATAACTct15}/.style={column name={},column type=C},
    columns/{tr20DATAACTct15}/.style={column name={},column type=C},
    columns/{tr25DATAACTct15}/.style={column name={},column type=C},
	columns/{tr30DATAACTct15}/.style={column name={},column type=C},
    columns/{tr40DATAACTct15}/.style={column name={},column type=C},
	columns/{tr50DATAACTct15}/.style={column name={},column type=C},
	%%	
]{\fileFIFTEENconstDATAACT}
}
%%%
%%%
\resizebox{\columnwidth}{!}{
\newcolumntype{C}{>{\centering\arraybackslash}p{21mm}}
\pgfplotstabletypeset[
font=\normalsize,
	every head row/.style={
    output empty row,
		after row={%
            \midrule
              \multicolumn{1}{c|}{\textbf{\emph{Log Size}} $\downarrow$}
            & \multicolumn{8}{c|}{\emph{\textbf{20 constraints}}}
            \\\midrule
		},
	},
	%
	columns={logSizeDATAACTct20,tr5DATAACTct20,tr10DATAACTct20,tr15DATAACTct20,tr20DATAACTct20,tr25DATAACTct20,tr30DATAACTct20,
tr40DATAACTct20,tr50DATAACTct20},	
    columns/{logSizeDATAACTct20}/.style={column name={},column type=C},	
    columns/{tr5DATAACTct20}/.style={column name={},column type=C},
    columns/{tr10DATAACTct20}/.style={column name={},column type=C},
    columns/{tr15DATAACTct20}/.style={column name={},column type=C},
    columns/{tr20DATAACTct20}/.style={column name={},column type=C},
    columns/{tr25DATAACTct20}/.style={column name={},column type=C},
	columns/{tr30DATAACTct20}/.style={column name={},column type=C},
    columns/{tr40DATAACTct20}/.style={column name={},column type=C},
	columns/{tr50DATAACTct20}/.style={column name={},column type=C},
	%%	
]{\fileTWENTYconstDATAACT}
}
\caption{Time (in ms.) required for generating logs of different sizes containing traces of different lengths from Declare models of increasing sizes (with activation conditions).}
\label{table:exp_results_real}
\end{center}
\end{table}
\end{small} 
%\pgfplotstableread[col sep=comma,header=true]{experiments/5-CONSTRAINTS-DATA-CORR.csv}\fileFIVEconstDATACORR

\pgfplotstablecreatecol[copy column from table={\fileFIVEconstDATACORR}{[index] 0}] {logSizeDATACORRct5} {\fileFIVEconstDATACORR}
\pgfplotstablecreatecol[copy column from table={\fileFIVEconstDATACORR}{[index] 1}] {tr5DATACORRct5} {\fileFIVEconstDATACORR}
\pgfplotstablecreatecol[copy column from table={\fileFIVEconstDATACORR}{[index] 2}] {tr10DATACORRct5} {\fileFIVEconstDATACORR}
\pgfplotstablecreatecol[copy column from table={\fileFIVEconstDATACORR}{[index] 3}] {tr15DATACORRct5} {\fileFIVEconstDATACORR}
\pgfplotstablecreatecol[copy column from table={\fileFIVEconstDATACORR}{[index] 4}] {tr20DATACORRct5} {\fileFIVEconstDATACORR}
\pgfplotstablecreatecol[copy column from table={\fileFIVEconstDATACORR}{[index] 5}] {tr25DATACORRct5} {\fileFIVEconstDATACORR}
\pgfplotstablecreatecol[copy column from table={\fileFIVEconstDATACORR}{[index] 6}] {tr30DATACORRct5} {\fileFIVEconstDATACORR}
\pgfplotstablecreatecol[copy column from table={\fileFIVEconstDATACORR}{[index] 7}] {tr40DATACORRct5} {\fileFIVEconstDATACORR}
\pgfplotstablecreatecol[copy column from table={\fileFIVEconstDATACORR}{[index] 8}] {tr50DATACORRct5} {\fileFIVEconstDATACORR}

\pgfplotstableread[col sep=comma,header=true]{experiments/10-CONSTRAINTS-DATA-CORR.csv}\fileTENconstDATACORR

\pgfplotstablecreatecol[copy column from table={\fileTENconstDATACORR}{[index] 0}] {logSizeDATACORRct10} {\fileTENconstDATACORR}
\pgfplotstablecreatecol[copy column from table={\fileTENconstDATACORR}{[index] 1}] {tr5DATACORRct10} {\fileTENconstDATACORR}
\pgfplotstablecreatecol[copy column from table={\fileTENconstDATACORR}{[index] 2}] {tr10DATACORRct10} {\fileTENconstDATACORR}
\pgfplotstablecreatecol[copy column from table={\fileTENconstDATACORR}{[index] 3}] {tr15DATACORRct10} {\fileTENconstDATACORR}
\pgfplotstablecreatecol[copy column from table={\fileTENconstDATACORR}{[index] 4}] {tr20DATACORRct10} {\fileTENconstDATACORR}
\pgfplotstablecreatecol[copy column from table={\fileTENconstDATACORR}{[index] 5}] {tr25DATACORRct10} {\fileTENconstDATACORR}
\pgfplotstablecreatecol[copy column from table={\fileTENconstDATACORR}{[index] 6}] {tr30DATACORRct10} {\fileTENconstDATACORR}
\pgfplotstablecreatecol[copy column from table={\fileTENconstDATACORR}{[index] 7}] {tr40DATACORRct10}  {\fileTENconstDATACORR}
\pgfplotstablecreatecol[copy column from table={\fileTENconstDATACORR}{[index] 8}] {tr50DATACORRct10}  {\fileTENconstDATACORR}

\pgfplotstableread[col sep=comma,header=true]{experiments/15-CONSTRAINTS-DATA-CORR.csv}\fileFIFTEENconstDATACORR

\pgfplotstablecreatecol[copy column from table={\fileFIFTEENconstDATACORR}{[index] 0}] {logSizeDATACORRct15} {\fileFIFTEENconstDATACORR}
\pgfplotstablecreatecol[copy column from table={\fileFIFTEENconstDATACORR}{[index] 1}] {tr5DATACORRct15} {\fileFIFTEENconstDATACORR}
\pgfplotstablecreatecol[copy column from table={\fileFIFTEENconstDATACORR}{[index] 2}] {tr10DATACORRct15} {\fileFIFTEENconstDATACORR}
\pgfplotstablecreatecol[copy column from table={\fileFIFTEENconstDATACORR}{[index] 3}] {tr15DATACORRct15} {\fileFIFTEENconstDATACORR}
\pgfplotstablecreatecol[copy column from table={\fileFIFTEENconstDATACORR}{[index] 4}] {tr20DATACORRct15} {\fileFIFTEENconstDATACORR}
\pgfplotstablecreatecol[copy column from table={\fileFIFTEENconstDATACORR}{[index] 5}] {tr25DATACORRct15} {\fileFIFTEENconstDATACORR}
\pgfplotstablecreatecol[copy column from table={\fileFIFTEENconstDATACORR}{[index] 6}] {tr30DATACORRct15} {\fileFIFTEENconstDATACORR}
\pgfplotstablecreatecol[copy column from table={\fileFIFTEENconstDATACORR}{[index] 7}] {tr40DATACORRct15}  {\fileFIFTEENconstDATACORR}
\pgfplotstablecreatecol[copy column from table={\fileFIFTEENconstDATACORR}{[index] 8}] {tr50DATACORRct15}  {\fileFIFTEENconstDATACORR}

\pgfplotstableread[col sep=comma,header=true]{experiments/20-CONSTRAINTS-DATA-CORR.csv}\fileTWENTYconstDATACORR

\pgfplotstablecreatecol[copy column from table={\fileTWENTYconstDATACORR}{[index] 0}] {logSizeDATACORRct20} {\fileTWENTYconstDATACORR}
\pgfplotstablecreatecol[copy column from table={\fileTWENTYconstDATACORR}{[index] 1}] {tr5DATACORRct20} {\fileTWENTYconstDATACORR}
\pgfplotstablecreatecol[copy column from table={\fileTWENTYconstDATACORR}{[index] 2}] {tr10DATACORRct20} {\fileTWENTYconstDATACORR}
\pgfplotstablecreatecol[copy column from table={\fileTWENTYconstDATACORR}{[index] 3}] {tr15DATACORRct20} {\fileTWENTYconstDATACORR}
\pgfplotstablecreatecol[copy column from table={\fileTWENTYconstDATACORR}{[index] 4}] {tr20DATACORRct20} {\fileTWENTYconstDATACORR}
\pgfplotstablecreatecol[copy column from table={\fileTWENTYconstDATACORR}{[index] 5}] {tr25DATACORRct20} {\fileTWENTYconstDATACORR}
\pgfplotstablecreatecol[copy column from table={\fileTWENTYconstDATACORR}{[index] 6}] {tr30DATACORRct20} {\fileTWENTYconstDATACORR}
\pgfplotstablecreatecol[copy column from table={\fileTWENTYconstDATACORR}{[index] 7}] {tr40DATACORRct20}  {\fileTWENTYconstDATACORR}
\pgfplotstablecreatecol[copy column from table={\fileTWENTYconstDATACORR}{[index] 8}] {tr50DATACORRct20}  {\fileTWENTYconstDATACORR}

\begin{small}
\begin{table}[t!]
\begin{center}
\resizebox{\columnwidth}{!}{
\newcolumntype{C}{>{\centering\arraybackslash}p{21mm}}
\pgfplotstabletypeset[
font=\normalsize,
	every head row/.style={
    output empty row,
		before row={%
              \toprule
              Trace length $\rightarrow$
            & 5
			& 10
            & 15
            & 20
            & 25
            & 30
            & 40
            & 50
            \\
		},
		after row={%
            \midrule
              \multicolumn{1}{c|}{\textbf{\emph{Log Size}} $\downarrow$}
            & \multicolumn{8}{c|}{\textbf{\emph{5 constraints}}}
            \\\midrule
		},
	},
	%
	columns={logSizeDATACORRct5,tr5DATACORRct5,tr10DATACORRct5,tr15DATACORRct5,tr20DATACORRct5,tr25DATACORRct5,tr30DATACORRct5,
tr40DATACORRct5,tr50DATACORRct5},	
    columns/{logSizeDATACORRct5}/.style={column name={},column type=C},	
    columns/{tr5DATACORRct5}/.style={column name={},column type=C},
    columns/{tr10DATACORRct5}/.style={column name={},column type=C},
    columns/{tr15DATACORRct5}/.style={column name={},column type=C},
    columns/{tr20DATACORRct5}/.style={column name={},column type=C},
    columns/{tr25DATACORRct5}/.style={column name={},column type=C},
	columns/{tr30DATACORRct5}/.style={column name={},column type=C},
    columns/{tr40DATACORRct5}/.style={column name={},column type=C},
	columns/{tr50DATACORRct5}/.style={column name={},column type=C},
	%%	
    %%every row 0 column 0/.style={postproc cell content/.style={@cell content={3-50}}},
    %%every row 1 column 0/.style={postproc cell content/.style={@cell content={51-75}}},
    %%every row 2 column 0/.style={postproc cell content/.style={@cell content={76-100}}},
    %%every row 3 column 0/.style={postproc cell content/.style={@cell content={101-128}}},
]{\fileFIVEconstDATACORR}
}
%%%
%%%
\resizebox{\columnwidth}{!}{
\newcolumntype{C}{>{\centering\arraybackslash}p{21mm}}
\pgfplotstabletypeset[
font=\normalsize,
	every head row/.style={
    output empty row,
		after row={%
            \midrule
              \multicolumn{1}{c|}{\textbf{\emph{Log Size}} $\downarrow$}
            & \multicolumn{8}{c|}{\textbf{\emph{10 constraints}}}
            \\\midrule
		},
	},
	%
	columns={logSizeDATACORRct10,tr5DATACORRct10,tr10DATACORRct10,tr15DATACORRct10,tr20DATACORRct10,tr25DATACORRct10,tr30DATACORRct10,
tr40DATACORRct10,tr50DATACORRct10},	
    columns/{logSizeDATACORRct10}/.style={column name={},column type=C},	
    columns/{tr5DATACORRct10}/.style={column name={},column type=C},
    columns/{tr10DATACORRct10}/.style={column name={},column type=C},
    columns/{tr15DATACORRct10}/.style={column name={},column type=C},
    columns/{tr20DATACORRct10}/.style={column name={},column type=C},
    columns/{tr25DATACORRct10}/.style={column name={},column type=C},
	columns/{tr30DATACORRct10}/.style={column name={},column type=C},
    columns/{tr40DATACORRct10}/.style={column name={},column type=C},
	columns/{tr50DATACORRct10}/.style={column name={},column type=C},
	%%	
    %%every row 0 column 0/.style={postproc cell content/.style={@cell content={3-50}}},
    %%every row 1 column 0/.style={postproc cell content/.style={@cell content={51-75}}},
    %%every row 2 column 0/.style={postproc cell content/.style={@cell content={76-100}}},
    %%every row 3 column 0/.style={postproc cell content/.style={@cell content={101-128}}},
]{\fileTENconstDATACORR}
}
%%%
%%%
\resizebox{\columnwidth}{!}{
\newcolumntype{C}{>{\centering\arraybackslash}p{21mm}}
\pgfplotstabletypeset[
font=\normalsize,
	every head row/.style={
    output empty row,
		after row={%
            \midrule
              \multicolumn{1}{c|}{\textbf{\emph{Log Size}} $\downarrow$}
            & \multicolumn{8}{c|}{\emph{\textbf{15 constraints}}}
            \\\midrule
		},
	},
	%
	columns={logSizeDATACORRct15,tr5DATACORRct15,tr10DATACORRct15,tr15DATACORRct15,tr20DATACORRct15,tr25DATACORRct15,tr30DATACORRct15,
tr40DATACORRct15,tr50DATACORRct15},	
    columns/{logSizeDATACORRct15}/.style={column name={},column type=C},	
    columns/{tr5DATACORRct15}/.style={column name={},column type=C},
    columns/{tr10DATACORRct15}/.style={column name={},column type=C},
    columns/{tr15DATACORRct15}/.style={column name={},column type=C},
    columns/{tr20DATACORRct15}/.style={column name={},column type=C},
    columns/{tr25DATACORRct15}/.style={column name={},column type=C},
	columns/{tr30DATACORRct15}/.style={column name={},column type=C},
    columns/{tr40DATACORRct15}/.style={column name={},column type=C},
	columns/{tr50DATACORRct15}/.style={column name={},column type=C},
	%%	
]{\fileFIFTEENconstDATACORR}
}
%%%
%%%
\resizebox{\columnwidth}{!}{
\newcolumntype{C}{>{\centering\arraybackslash}p{21mm}}
\pgfplotstabletypeset[
font=\normalsize,
	every head row/.style={
    output empty row,
		after row={%
            \midrule
              \multicolumn{1}{c|}{\textbf{\emph{Log Size}} $\downarrow$}
            & \multicolumn{8}{c|}{\emph{\textbf{20 constraints}}}
            \\\midrule
		},
	},
	%
	columns={logSizeDATACORRct20,tr5DATACORRct20,tr10DATACORRct20,tr15DATACORRct20,tr20DATACORRct20,tr25DATACORRct20,tr30DATACORRct20,
tr40DATACORRct20,tr50DATACORRct20},	
    columns/{logSizeDATACORRct20}/.style={column name={},column type=C},	
    columns/{tr5DATACORRct20}/.style={column name={},column type=C},
    columns/{tr10DATACORRct20}/.style={column name={},column type=C},
    columns/{tr15DATACORRct20}/.style={column name={},column type=C},
    columns/{tr20DATACORRct20}/.style={column name={},column type=C},
    columns/{tr25DATACORRct20}/.style={column name={},column type=C},
	columns/{tr30DATACORRct20}/.style={column name={},column type=C},
    columns/{tr40DATACORRct20}/.style={column name={},column type=C},
	columns/{tr50DATACORRct20}/.style={column name={},column type=C},
	%%	
]{\fileTWENTYconstDATACORR}
}
\caption{Time (in ms.) required for generating logs of different sizes containing traces of different lengths from Declare models of increasing sizes (with activation and correlation conditions).}
\label{table:exp_results_real}
\end{center}
\end{table}
\end{small} 
%\pgfplotstableread[col sep=comma,header=true]{experiments/5-ALPH-DATA-CORR.csv}\fileFIVEconstDATACORR

\pgfplotstablecreatecol[copy column from table={\fileFIVEconstDATACORR}{[index] 0}] {logSizeDATACORRct5} {\fileFIVEconstDATACORR}
\pgfplotstablecreatecol[copy column from table={\fileFIVEconstDATACORR}{[index] 1}] {tr20DATACORRct5} {\fileFIVEconstDATACORR}
\pgfplotstablecreatecol[copy column from table={\fileFIVEconstDATACORR}{[index] 2}] {tr25DATACORRct5} {\fileFIVEconstDATACORR}
\pgfplotstablecreatecol[copy column from table={\fileFIVEconstDATACORR}{[index] 3}] {tr30DATACORRct5} {\fileFIVEconstDATACORR}
\pgfplotstablecreatecol[copy column from table={\fileFIVEconstDATACORR}{[index] 4}] {tr40DATACORRct5} {\fileFIVEconstDATACORR}
\pgfplotstablecreatecol[copy column from table={\fileFIVEconstDATACORR}{[index] 5}] {tr50DATACORRct5} {\fileFIVEconstDATACORR}

\pgfplotstableread[col sep=comma,header=true]{experiments/10-ALPH-DATA-CORR.csv}\fileTENconstDATACORR

\pgfplotstablecreatecol[copy column from table={\fileTENconstDATACORR}{[index] 0}] {logSizeDATACORRct10} {\fileTENconstDATACORR}
\pgfplotstablecreatecol[copy column from table={\fileTENconstDATACORR}{[index] 1}] {tr20DATACORRct10} {\fileTENconstDATACORR}
\pgfplotstablecreatecol[copy column from table={\fileTENconstDATACORR}{[index] 2}] {tr25DATACORRct10} {\fileTENconstDATACORR}
\pgfplotstablecreatecol[copy column from table={\fileTENconstDATACORR}{[index] 3}] {tr30DATACORRct10} {\fileTENconstDATACORR}
\pgfplotstablecreatecol[copy column from table={\fileTENconstDATACORR}{[index] 4}] {tr40DATACORRct10}  {\fileTENconstDATACORR}
\pgfplotstablecreatecol[copy column from table={\fileTENconstDATACORR}{[index] 5}] {tr50DATACORRct10}  {\fileTENconstDATACORR}

\pgfplotstableread[col sep=comma,header=true]{experiments/15-ALPH-DATA-CORR.csv}\fileFIFTEENconstDATACORR

\pgfplotstablecreatecol[copy column from table={\fileFIFTEENconstDATACORR}{[index] 0}] {logSizeDATACORRct15} {\fileFIFTEENconstDATACORR}
\pgfplotstablecreatecol[copy column from table={\fileFIFTEENconstDATACORR}{[index] 1}] {tr20DATACORRct15} {\fileFIFTEENconstDATACORR}
\pgfplotstablecreatecol[copy column from table={\fileFIFTEENconstDATACORR}{[index] 2}] {tr25DATACORRct15} {\fileFIFTEENconstDATACORR}
\pgfplotstablecreatecol[copy column from table={\fileFIFTEENconstDATACORR}{[index] 3}] {tr30DATACORRct15} {\fileFIFTEENconstDATACORR}
\pgfplotstablecreatecol[copy column from table={\fileFIFTEENconstDATACORR}{[index] 4}] {tr40DATACORRct15}  {\fileFIFTEENconstDATACORR}
\pgfplotstablecreatecol[copy column from table={\fileFIFTEENconstDATACORR}{[index] 5}] {tr50DATACORRct15}  {\fileFIFTEENconstDATACORR}

\pgfplotstableread[col sep=comma,header=true]{experiments/20-ALPH-DATA-CORR.csv}\fileTWENTYconstDATACORR

\pgfplotstablecreatecol[copy column from table={\fileTWENTYconstDATACORR}{[index] 0}] {logSizeDATACORRct20} {\fileTWENTYconstDATACORR}
\pgfplotstablecreatecol[copy column from table={\fileTWENTYconstDATACORR}{[index] 1}] {tr20DATACORRct20} {\fileTWENTYconstDATACORR}
\pgfplotstablecreatecol[copy column from table={\fileTWENTYconstDATACORR}{[index] 2}] {tr25DATACORRct20} {\fileTWENTYconstDATACORR}
\pgfplotstablecreatecol[copy column from table={\fileTWENTYconstDATACORR}{[index] 3}] {tr30DATACORRct20} {\fileTWENTYconstDATACORR}
\pgfplotstablecreatecol[copy column from table={\fileTWENTYconstDATACORR}{[index] 4}] {tr40DATACORRct20}  {\fileTWENTYconstDATACORR}
\pgfplotstablecreatecol[copy column from table={\fileTWENTYconstDATACORR}{[index] 5}] {tr50DATACORRct20}  {\fileTWENTYconstDATACORR}

\begin{small}
\begin{table}[t!]
\begin{center}
\resizebox{\columnwidth}{!}{
\newcolumntype{C}{>{\centering\arraybackslash}p{25mm}}
\pgfplotstabletypeset[
font=\small,
	every head row/.style={
    output empty row,
		before row={%
              \toprule
              Trace length $\rightarrow$
            & 20
            & 25
            & 30
            & 40
            & 50
            \\
		},
		after row={%
            \midrule
              \multicolumn{1}{c|}{\textbf{\emph{Log Size}} $\downarrow$}
            & \multicolumn{5}{c|}{\textbf{\emph{5 activities}}}
            \\\midrule
		},
	},
	%
	columns={logSizeDATACORRct5,tr20DATACORRct5,tr25DATACORRct5,tr30DATACORRct5,tr40DATACORRct5,tr50DATACORRct5},	
    columns/{logSizeDATACORRct5}/.style={column name={},column type=C},	
    columns/{tr20DATACORRct5}/.style={column name={},column type=C},
    columns/{tr25DATACORRct5}/.style={column name={},column type=C},
	columns/{tr30DATACORRct5}/.style={column name={},column type=C},
    columns/{tr40DATACORRct5}/.style={column name={},column type=C},
	columns/{tr50DATACORRct5}/.style={column name={},column type=C},
	%%	
    %%every row 0 column 0/.style={postproc cell content/.style={@cell content={3-50}}},
    %%every row 1 column 0/.style={postproc cell content/.style={@cell content={51-75}}},
    %%every row 2 column 0/.style={postproc cell content/.style={@cell content={76-100}}},
    %%every row 3 column 0/.style={postproc cell content/.style={@cell content={101-128}}},
]{\fileFIVEconstDATACORR}
}
%%%
%%%
\resizebox{\columnwidth}{!}{
\newcolumntype{C}{>{\centering\arraybackslash}p{25mm}}
\pgfplotstabletypeset[
font=\small,
	every head row/.style={
    output empty row,
		after row={%
            \midrule
              \multicolumn{1}{c|}{\textbf{\emph{Log Size}} $\downarrow$}
            & \multicolumn{5}{c|}{\textbf{\emph{10 activities}}}
            \\\midrule
		},
	},
	%
	columns={logSizeDATACORRct10,tr20DATACORRct10,tr25DATACORRct10,tr30DATACORRct10,tr40DATACORRct10,tr50DATACORRct10},	
    columns/{logSizeDATACORRct10}/.style={column name={},column type=C},	
    columns/{tr20DATACORRct10}/.style={column name={},column type=C},
    columns/{tr25DATACORRct10}/.style={column name={},column type=C},
	columns/{tr30DATACORRct10}/.style={column name={},column type=C},
    columns/{tr40DATACORRct10}/.style={column name={},column type=C},
	columns/{tr50DATACORRct10}/.style={column name={},column type=C},
	%%	
    %%every row 0 column 0/.style={postproc cell content/.style={@cell content={3-50}}},
    %%every row 1 column 0/.style={postproc cell content/.style={@cell content={51-75}}},
    %%every row 2 column 0/.style={postproc cell content/.style={@cell content={76-100}}},
    %%every row 3 column 0/.style={postproc cell content/.style={@cell content={101-128}}},
]{\fileTENconstDATACORR}
}
%%%
%%%
\resizebox{\columnwidth}{!}{
\newcolumntype{C}{>{\centering\arraybackslash}p{25mm}}
\pgfplotstabletypeset[
font=\small,
	every head row/.style={
    output empty row,
		after row={%
            \midrule
              \multicolumn{1}{c|}{\textbf{\emph{Log Size}} $\downarrow$}
            & \multicolumn{5}{c|}{\emph{\textbf{15 activities}}}
            \\\midrule
		},
	},
	%
	columns={logSizeDATACORRct15,tr20DATACORRct15,tr25DATACORRct15,tr30DATACORRct15,tr40DATACORRct15,tr50DATACORRct15},
    columns/{logSizeDATACORRct15}/.style={column name={},column type=C},	
    columns/{tr20DATACORRct15}/.style={column name={},column type=C},
    columns/{tr25DATACORRct15}/.style={column name={},column type=C},
	columns/{tr30DATACORRct15}/.style={column name={},column type=C},
    columns/{tr40DATACORRct15}/.style={column name={},column type=C},
	columns/{tr50DATACORRct15}/.style={column name={},column type=C},
	%%	
]{\fileFIFTEENconstDATACORR}
}
%%%
%%%
\resizebox{\columnwidth}{!}{
\newcolumntype{C}{>{\centering\arraybackslash}p{25mm}}
\pgfplotstabletypeset[
font=\small,
	every head row/.style={
    output empty row,
		after row={%
            \midrule
              \multicolumn{1}{c|}{\textbf{\emph{Log Size}} $\downarrow$}
            & \multicolumn{5}{c|}{\emph{\textbf{20 activities}}}
            \\\midrule
		},
	},
	%
	columns={logSizeDATACORRct20,tr20DATACORRct20,tr25DATACORRct20,tr30DATACORRct20,tr40DATACORRct20,tr50DATACORRct20},	 columns/{logSizeDATACORRct20}/.style={column name={},column type=C},	
    columns/{tr20DATACORRct20}/.style={column name={},column type=C},
    columns/{tr25DATACORRct20}/.style={column name={},column type=C},
	columns/{tr30DATACORRct20}/.style={column name={},column type=C},
    columns/{tr40DATACORRct20}/.style={column name={},column type=C},
	columns/{tr50DATACORRct20}/.style={column name={},column type=C},
	%%	
]{\fileTWENTYconstDATACORR}
}
\caption{Time (in ms.) required for generating logs of different sizes containing traces of different lengths from Declare models of increasing alphabet sizes (without data conditions).}
\label{table:exp_results_real}
\end{center}
\end{table}
\end{small} 
%\pgfplotstableread[col sep=comma,header=true]{experiments/query1.csv}\fileQueryONE

\pgfplotstablecreatecol[copy column from table={\fileQueryONE}{[index] 0}] {alphSizeQueryONE} {\fileQueryONE}
\pgfplotstablecreatecol[copy column from table={\fileQueryONE}{[index] 1}] {tr20QueryONE} {\fileQueryONE}
\pgfplotstablecreatecol[copy column from table={\fileQueryONE}{[index] 2}] {tr25QueryONE} {\fileQueryONE}
\pgfplotstablecreatecol[copy column from table={\fileQueryONE}{[index] 3}] {tr30QueryONE} {\fileQueryONE}
\pgfplotstablecreatecol[copy column from table={\fileQueryONE}{[index] 4}] {tr40QueryONE} {\fileQueryONE}
\pgfplotstablecreatecol[copy column from table={\fileQueryONE}{[index] 5}] {tr50QueryONE} {\fileQueryONE}

\pgfplotstableread[col sep=comma,header=true]{experiments/query2.csv}\fileQueryTWO

\pgfplotstablecreatecol[copy column from table={\fileQueryTWO}{[index] 0}] {alphSizeQueryTWO} {\fileQueryTWO}
\pgfplotstablecreatecol[copy column from table={\fileQueryTWO}{[index] 1}] {tr20QueryTWO} {\fileQueryTWO}
\pgfplotstablecreatecol[copy column from table={\fileQueryTWO}{[index] 2}] {tr25QueryTWO} {\fileQueryTWO}
\pgfplotstablecreatecol[copy column from table={\fileQueryTWO}{[index] 3}] {tr30QueryTWO} {\fileQueryTWO}
\pgfplotstablecreatecol[copy column from table={\fileQueryTWO}{[index] 4}] {tr40QueryTWO}  {\fileQueryTWO}
\pgfplotstablecreatecol[copy column from table={\fileQueryTWO}{[index] 5}] {tr50QueryTWO}  {\fileQueryTWO}

\pgfplotstableread[col sep=comma,header=true]{experiments/query3.csv}\fileQueryTHREE

\pgfplotstablecreatecol[copy column from table={\fileQueryTHREE}{[index] 0}] {alphSizeQueryTHREE} {\fileQueryTHREE}
\pgfplotstablecreatecol[copy column from table={\fileQueryTHREE}{[index] 1}] {tr20QueryTHREE} {\fileQueryTHREE}
\pgfplotstablecreatecol[copy column from table={\fileQueryTHREE}{[index] 2}] {tr25QueryTHREE} {\fileQueryTHREE}
\pgfplotstablecreatecol[copy column from table={\fileQueryTHREE}{[index] 3}] {tr30QueryTHREE} {\fileQueryTHREE}
\pgfplotstablecreatecol[copy column from table={\fileQueryTHREE}{[index] 4}] {tr40QueryTHREE}  {\fileQueryTHREE}
\pgfplotstablecreatecol[copy column from table={\fileQueryTHREE}{[index] 5}] {tr50QueryTHREE}  {\fileQueryTHREE}

\pgfplotstableread[col sep=comma,header=true]{experiments/query4.csv}\fileQueryFOUR

\pgfplotstablecreatecol[copy column from table={\fileQueryFOUR}{[index] 0}] {alphSizeQueryFOUR} {\fileQueryFOUR}
\pgfplotstablecreatecol[copy column from table={\fileQueryFOUR}{[index] 1}] {tr20QueryFOUR} {\fileQueryFOUR}
\pgfplotstablecreatecol[copy column from table={\fileQueryFOUR}{[index] 2}] {tr25QueryFOUR} {\fileQueryFOUR}
\pgfplotstablecreatecol[copy column from table={\fileQueryFOUR}{[index] 3}] {tr30QueryFOUR} {\fileQueryFOUR}
\pgfplotstablecreatecol[copy column from table={\fileQueryFOUR}{[index] 4}] {tr40QueryFOUR}  {\fileQueryFOUR}
\pgfplotstablecreatecol[copy column from table={\fileQueryFOUR}{[index] 5}] {tr50QueryFOUR}  {\fileQueryFOUR}

\pgfplotstableread[col sep=comma,header=true]{experiments/query5.csv}\fileQueryFIVE

\pgfplotstablecreatecol[copy column from table={\fileQueryFIVE}{[index] 0}] {alphSizeQueryFIVE} {\fileQueryFIVE}
\pgfplotstablecreatecol[copy column from table={\fileQueryFIVE}{[index] 1}] {tr20QueryFIVE} {\fileQueryFIVE}
\pgfplotstablecreatecol[copy column from table={\fileQueryFIVE}{[index] 2}] {tr25QueryFIVE} {\fileQueryFIVE}
\pgfplotstablecreatecol[copy column from table={\fileQueryFIVE}{[index] 3}] {tr30QueryFIVE} {\fileQueryFIVE}
\pgfplotstablecreatecol[copy column from table={\fileQueryFIVE}{[index] 4}] {tr40QueryFIVE}  {\fileQueryFIVE}
\pgfplotstablecreatecol[copy column from table={\fileQueryFIVE}{[index] 5}] {tr50QueryFIVE}  {\fileQueryFIVE}

\pgfplotstableread[col sep=comma,header=true]{experiments/query6.csv}\fileQuerySIX

\pgfplotstablecreatecol[copy column from table={\fileQuerySIX}{[index] 0}] {alphSizeQuerySIX} {\fileQuerySIX}
\pgfplotstablecreatecol[copy column from table={\fileQuerySIX}{[index] 1}] {tr20QuerySIX} {\fileQuerySIX}
\pgfplotstablecreatecol[copy column from table={\fileQuerySIX}{[index] 2}] {tr25QuerySIX} {\fileQuerySIX}
\pgfplotstablecreatecol[copy column from table={\fileQuerySIX}{[index] 3}] {tr30QuerySIX} {\fileQuerySIX}
\pgfplotstablecreatecol[copy column from table={\fileQuerySIX}{[index] 4}] {tr40QuerySIX}  {\fileQuerySIX}
\pgfplotstablecreatecol[copy column from table={\fileQuerySIX}{[index] 5}] {tr50QuerySIX}  {\fileQuerySIX}

\pgfplotstableread[col sep=comma,header=true]{experiments/query7.csv}\fileQuerySEVEN

\pgfplotstablecreatecol[copy column from table={\fileQuerySEVEN}{[index] 0}] {alphSizeQuerySEVEN} {\fileQuerySEVEN}
\pgfplotstablecreatecol[copy column from table={\fileQuerySEVEN}{[index] 1}] {tr20QuerySEVEN} {\fileQuerySEVEN}
\pgfplotstablecreatecol[copy column from table={\fileQuerySEVEN}{[index] 2}] {tr25QuerySEVEN} {\fileQuerySEVEN}
\pgfplotstablecreatecol[copy column from table={\fileQuerySEVEN}{[index] 3}] {tr30QuerySEVEN} {\fileQuerySEVEN}
\pgfplotstablecreatecol[copy column from table={\fileQuerySEVEN}{[index] 4}] {tr40QuerySEVEN}  {\fileQuerySEVEN}
\pgfplotstablecreatecol[copy column from table={\fileQuerySEVEN}{[index] 5}] {tr50QuerySEVEN}  {\fileQuerySEVEN}

\pgfplotstableread[col sep=comma,header=true]{experiments/query8.csv}\fileQueryEIGHT

\pgfplotstablecreatecol[copy column from table={\fileQueryEIGHT}{[index] 0}] {alphSizeQueryEIGHT} {\fileQueryEIGHT}
\pgfplotstablecreatecol[copy column from table={\fileQueryEIGHT}{[index] 1}] {tr20QueryEIGHT} {\fileQueryEIGHT}
\pgfplotstablecreatecol[copy column from table={\fileQueryEIGHT}{[index] 2}] {tr25QueryEIGHT} {\fileQueryEIGHT}
\pgfplotstablecreatecol[copy column from table={\fileQueryEIGHT}{[index] 3}] {tr30QueryEIGHT} {\fileQueryEIGHT}
\pgfplotstablecreatecol[copy column from table={\fileQueryEIGHT}{[index] 4}] {tr40QueryEIGHT}  {\fileQueryEIGHT}
\pgfplotstablecreatecol[copy column from table={\fileQueryEIGHT}{[index] 5}] {tr50QueryEIGHT}  {\fileQueryEIGHT}

\pgfplotstableread[col sep=comma,header=true]{experiments/query9.csv}\fileQueryNINE

\pgfplotstablecreatecol[copy column from table={\fileQueryNINE}{[index] 0}] {alphSizeQueryNINE} {\fileQueryNINE}
\pgfplotstablecreatecol[copy column from table={\fileQueryNINE}{[index] 1}] {tr20QueryNINE} {\fileQueryNINE}
\pgfplotstablecreatecol[copy column from table={\fileQueryNINE}{[index] 2}] {tr25QueryNINE} {\fileQueryNINE}
\pgfplotstablecreatecol[copy column from table={\fileQueryNINE}{[index] 3}] {tr30QueryNINE} {\fileQueryNINE}
\pgfplotstablecreatecol[copy column from table={\fileQueryNINE}{[index] 4}] {tr40QueryNINE}  {\fileQueryNINE}
\pgfplotstablecreatecol[copy column from table={\fileQueryNINE}{[index] 5}] {tr50QueryNINE}  {\fileQueryNINE}

\pgfplotstableread[col sep=comma,header=true]{experiments/query10.csv}\fileQueryTEN

\pgfplotstablecreatecol[copy column from table={\fileQueryTEN}{[index] 0}] {alphSizeQueryTEN} {\fileQueryTEN}
\pgfplotstablecreatecol[copy column from table={\fileQueryTEN}{[index] 1}] {tr20QueryTEN} {\fileQueryTEN}
\pgfplotstablecreatecol[copy column from table={\fileQueryTEN}{[index] 2}] {tr25QueryTEN} {\fileQueryTEN}
\pgfplotstablecreatecol[copy column from table={\fileQueryTEN}{[index] 3}] {tr30QueryTEN} {\fileQueryTEN}
\pgfplotstablecreatecol[copy column from table={\fileQueryTEN}{[index] 4}] {tr40QueryTEN}  {\fileQueryTEN}
\pgfplotstablecreatecol[copy column from table={\fileQueryTEN}{[index] 5}] {tr50QueryTEN}  {\fileQueryTEN}

\pgfplotstableread[col sep=comma,header=true]{experiments/query11.csv}\fileQueryELEVEN

\pgfplotstablecreatecol[copy column from table={\fileQueryELEVEN}{[index] 0}] {alphSizeQueryELEVEN} {\fileQueryELEVEN}
\pgfplotstablecreatecol[copy column from table={\fileQueryELEVEN}{[index] 1}] {tr20QueryELEVEN} {\fileQueryELEVEN}
\pgfplotstablecreatecol[copy column from table={\fileQueryELEVEN}{[index] 2}] {tr25QueryELEVEN} {\fileQueryELEVEN}
\pgfplotstablecreatecol[copy column from table={\fileQueryELEVEN}{[index] 3}] {tr30QueryELEVEN} {\fileQueryELEVEN}
\pgfplotstablecreatecol[copy column from table={\fileQueryELEVEN}{[index] 4}] {tr40QueryELEVEN}  {\fileQueryELEVEN}
\pgfplotstablecreatecol[copy column from table={\fileQueryELEVEN}{[index] 5}] {tr50QueryELEVEN}  {\fileQueryELEVEN}

\pgfplotstableread[col sep=comma,header=true]{experiments/query12.csv}\fileQueryTWELVE

\pgfplotstablecreatecol[copy column from table={\fileQueryTWELVE}{[index] 0}] {alphSizeQueryTWELVE} {\fileQueryTWELVE}
\pgfplotstablecreatecol[copy column from table={\fileQueryTWELVE}{[index] 1}] {tr20QueryTWELVE} {\fileQueryTWELVE}
\pgfplotstablecreatecol[copy column from table={\fileQueryTWELVE}{[index] 2}] {tr25QueryTWELVE} {\fileQueryTWELVE}
\pgfplotstablecreatecol[copy column from table={\fileQueryTWELVE}{[index] 3}] {tr30QueryTWELVE} {\fileQueryTWELVE}
\pgfplotstablecreatecol[copy column from table={\fileQueryTWELVE}{[index] 4}] {tr40QueryTWELVE}  {\fileQueryTWELVE}
\pgfplotstablecreatecol[copy column from table={\fileQueryTWELVE}{[index] 5}] {tr50QueryTWELVE}  {\fileQueryTWELVE}


\begin{table}
\begin{center}
\scalebox{1.}[0.75]{
\resizebox{\columnwidth}{!}{
\newcolumntype{C}{>{\centering\arraybackslash}p{21mm}}
\pgfplotstabletypeset[
font=\scriptsize,
	every head row/.style={
    output empty row,
		before row={%
              \toprule
              Trace length $\rightarrow$
            & 20
            & 25
            & 30
            & 40
            & 50
            \\
		},
		after row={%
            \midrule
              \multicolumn{1}{c|}{\textbf{\emph{Alphabet Size}} $\downarrow$}
            & \multicolumn{5}{c|}{\textbf{Response[?X,?Y]}}
            \\\midrule
		},
	},
	%
	columns={alphSizeQueryONE,tr20QueryONE,tr25QueryONE,tr30QueryONE,tr40QueryONE,tr50QueryONE},	
    columns/{alphSizeQueryONE}/.style={column name={},column type=C},	
    columns/{tr20QueryONE}/.style={column name={},column type=C},
    columns/{tr25QueryONE}/.style={column name={},column type=C},
	columns/{tr30QueryONE}/.style={column name={},column type=C},
    columns/{tr40QueryONE}/.style={column name={},column type=C},
	columns/{tr50QueryONE}/.style={column name={},column type=C},
	%%	
    %%every row 0 column 0/.style={postproc cell content/.style={@cell content={3-50}}},
    %%every row 1 column 0/.style={postproc cell content/.style={@cell content={51-75}}},
    %%every row 2 column 0/.style={postproc cell content/.style={@cell content={76-100}}},
    %%every row 3 column 0/.style={postproc cell content/.style={@cell content={101-128}}},
]{\fileQueryONE}
}
}
%%%
%%%
\scalebox{1.}[0.75]{
\resizebox{\columnwidth}{!}{
\newcolumntype{C}{>{\centering\arraybackslash}p{21mm}}
\pgfplotstabletypeset[
font=\scriptsize,
	every head row/.style={
    output empty row,
		after row={%
            \midrule
              \multicolumn{1}{c|}{\textbf{\emph{Alphabet Size}} $\downarrow$}
            & \multicolumn{5}{c|}{\textbf{Existence[?X]}}
            \\\midrule
		},
	},
	%
	columns={alphSizeQueryTWO,tr20QueryTWO,tr25QueryTWO,tr30QueryTWO,tr40QueryTWO,tr50QueryTWO},	 columns/{alphSizeQueryTWO}/.style={column name={},column type=C},	
    columns/{tr20QueryTWO}/.style={column name={},column type=C},
    columns/{tr25QueryTWO}/.style={column name={},column type=C},
	columns/{tr30QueryTWO}/.style={column name={},column type=C},
    columns/{tr40QueryTWO}/.style={column name={},column type=C},
	columns/{tr50QueryTWO}/.style={column name={},column type=C},
	%%	
]{\fileQueryTWO}
}
}
%%%
%%%
\scalebox{1.}[0.75]{
\resizebox{\columnwidth}{!}{
\newcolumntype{C}{>{\centering\arraybackslash}p{21mm}}
\pgfplotstabletypeset[
font=\scriptsize,
	every head row/.style={
    output empty row,
		after row={%
            \midrule
              \multicolumn{1}{c|}{\textbf{\emph{Alphabet Size}} $\downarrow$}
            & \multicolumn{5}{c|}{\textbf{Absence[?X]}}
            \\\midrule
		},
	},
	%
	columns={alphSizeQueryTHREE,tr20QueryTHREE,tr25QueryTHREE,tr30QueryTHREE,tr40QueryTHREE,tr50QueryTHREE},	 columns/{alphSizeQueryTHREE}/.style={column name={},column type=C},	
    columns/{tr20QueryTHREE}/.style={column name={},column type=C},
    columns/{tr25QueryTHREE}/.style={column name={},column type=C},
	columns/{tr30QueryTHREE}/.style={column name={},column type=C},
    columns/{tr40QueryTHREE}/.style={column name={},column type=C},
	columns/{tr50QueryTHREE}/.style={column name={},column type=C},
	%%	
]{\fileQueryTHREE}
}
}
%%%
%%%
\scalebox{1.}[0.75]{
\resizebox{\columnwidth}{!}{
\newcolumntype{C}{>{\centering\arraybackslash}p{21mm}}
\pgfplotstabletypeset[
font=\scriptsize,
	every head row/.style={
    output empty row,
		after row={%
            \midrule
              \multicolumn{1}{c|}{\textbf{\emph{Alphabet Size}} $\downarrow$}
            & \multicolumn{5}{c|}{\textbf{RespondedExistence[?X,?Y]}}
            \\\midrule
		},
	},
	%
	columns={alphSizeQueryFOUR,tr20QueryFOUR,tr25QueryFOUR,tr30QueryFOUR,tr40QueryFOUR,tr50QueryFOUR},	 columns/{alphSizeQueryFOUR}/.style={column name={},column type=C},	
    columns/{tr20QueryFOUR}/.style={column name={},column type=C},
    columns/{tr25QueryFOUR}/.style={column name={},column type=C},
	columns/{tr30QueryFOUR}/.style={column name={},column type=C},
    columns/{tr40QueryFOUR}/.style={column name={},column type=C},
	columns/{tr50QueryFOUR}/.style={column name={},column type=C},
	%%	
]{\fileQueryFOUR}
}
}
%%%
%%%
\scalebox{1.}[0.75]{
\resizebox{\columnwidth}{!}{
\newcolumntype{C}{>{\centering\arraybackslash}p{21mm}}
\pgfplotstabletypeset[
font=\scriptsize,
	every head row/.style={
    output empty row,
		after row={%
            \midrule
              \multicolumn{1}{c|}{\textbf{\emph{Alphabet Size}} $\downarrow$}
            & \multicolumn{5}{c|}{\textbf{AlternateResponse[?X,?Y]}}
            \\\midrule
		},
	},
	%
	columns={alphSizeQueryFIVE,tr20QueryFIVE,tr25QueryFIVE,tr30QueryFIVE,tr40QueryFIVE,tr50QueryFIVE},	 columns/{alphSizeQueryFIVE}/.style={column name={},column type=C},	
    columns/{tr20QueryFIVE}/.style={column name={},column type=C},
    columns/{tr25QueryFIVE}/.style={column name={},column type=C},
	columns/{tr30QueryFIVE}/.style={column name={},column type=C},
    columns/{tr40QueryFIVE}/.style={column name={},column type=C},
	columns/{tr50QueryFIVE}/.style={column name={},column type=C},
	%%	
]{\fileQueryFIVE}
}
}
%%%
%%%
\scalebox{1.}[0.75]{
\resizebox{\columnwidth}{!}{
\newcolumntype{C}{>{\centering\arraybackslash}p{21mm}}
\pgfplotstabletypeset[
font=\scriptsize,
	every head row/.style={
    output empty row,
		after row={%
            \midrule
              \multicolumn{1}{c|}{\textbf{\emph{Alphabet Size}} $\downarrow$}
            & \multicolumn{5}{c|}{\textbf{ChainResponse[?X,?Y]}}
            \\\midrule
		},
	},
	%
	columns={alphSizeQuerySIX,tr20QuerySIX,tr25QuerySIX,tr30QuerySIX,tr40QuerySIX,tr50QuerySIX},	 columns/{alphSizeQuerySIX}/.style={column name={},column type=C},	
    columns/{tr20QuerySIX}/.style={column name={},column type=C},
    columns/{tr25QuerySIX}/.style={column name={},column type=C},
	columns/{tr30QuerySIX}/.style={column name={},column type=C},
    columns/{tr40QuerySIX}/.style={column name={},column type=C},
	columns/{tr50QuerySIX}/.style={column name={},column type=C},
	%%	
]{\fileQuerySIX}
}
}
%%%
%%%
\scalebox{1.}[0.75]{
\resizebox{\columnwidth}{!}{
\newcolumntype{C}{>{\centering\arraybackslash}p{21mm}}
\pgfplotstabletypeset[
font=\scriptsize,
	every head row/.style={
    output empty row,
		after row={%
            \midrule
              \multicolumn{1}{c|}{\textbf{\emph{Alphabet Size}} $\downarrow$}
            & \multicolumn{5}{c|}{\textbf{Precedence[?X,?Y]}}
            \\\midrule
		},
	},
	%
	columns={alphSizeQuerySEVEN,tr20QuerySEVEN,tr25QuerySEVEN,tr30QuerySEVEN,tr40QuerySEVEN,tr50QuerySEVEN},	 columns/{alphSizeQuerySEVEN}/.style={column name={},column type=C},	
    columns/{tr20QuerySEVEN}/.style={column name={},column type=C},
    columns/{tr25QuerySEVEN}/.style={column name={},column type=C},
	columns/{tr30QuerySEVEN}/.style={column name={},column type=C},
    columns/{tr40QuerySEVEN}/.style={column name={},column type=C},
	columns/{tr50QuerySEVEN}/.style={column name={},column type=C},
	%%	
]{\fileQuerySEVEN}
}
}
%%%
%%%
\scalebox{1.}[0.75]{
\resizebox{\columnwidth}{!}{
\newcolumntype{C}{>{\centering\arraybackslash}p{21mm}}
\pgfplotstabletypeset[
font=\scriptsize,
	every head row/.style={
    output empty row,
		after row={%
            \midrule
              \multicolumn{1}{c|}{\textbf{\emph{Alphabet Size}} $\downarrow$}
            & \multicolumn{5}{c|}{\textbf{AlternatePrecedence[?X,?Y]}}
            \\\midrule
		},
	},
	%
	columns={alphSizeQueryEIGHT,tr20QueryEIGHT,tr25QueryEIGHT,tr30QueryEIGHT,tr40QueryEIGHT,tr50QueryEIGHT},	 columns/{alphSizeQueryEIGHT}/.style={column name={},column type=C},	
    columns/{tr20QueryEIGHT}/.style={column name={},column type=C},
    columns/{tr25QueryEIGHT}/.style={column name={},column type=C},
	columns/{tr30QueryEIGHT}/.style={column name={},column type=C},
    columns/{tr40QueryEIGHT}/.style={column name={},column type=C},
	columns/{tr50QueryEIGHT}/.style={column name={},column type=C},
	%%	
]{\fileQueryEIGHT}
}
}
%%%
%%%
\scalebox{1.}[0.75]{
\resizebox{\columnwidth}{!}{
\newcolumntype{C}{>{\centering\arraybackslash}p{21mm}}
\pgfplotstabletypeset[
font=\scriptsize,
	every head row/.style={
    output empty row,
		after row={%
            \midrule
              \multicolumn{1}{c|}{\textbf{\emph{Alphabet Size}} $\downarrow$}
            & \multicolumn{5}{c|}{\textbf{ChainPrecedence[?X,?Y]}}
            \\\midrule
		},
	},
	%
	columns={alphSizeQueryNINE,tr20QueryNINE,tr25QueryNINE,tr30QueryNINE,tr40QueryNINE,tr50QueryNINE},	 columns/{alphSizeQueryNINE}/.style={column name={},column type=C},	
    columns/{tr20QueryNINE}/.style={column name={},column type=C},
    columns/{tr25QueryNINE}/.style={column name={},column type=C},
	columns/{tr30QueryNINE}/.style={column name={},column type=C},
    columns/{tr40QueryNINE}/.style={column name={},column type=C},
	columns/{tr50QueryNINE}/.style={column name={},column type=C},
	%%	
]{\fileQueryNINE}
}
}
%%%
%%%
\scalebox{1.}[0.75]{
\resizebox{\columnwidth}{!}{
\newcolumntype{C}{>{\centering\arraybackslash}p{21mm}}
\pgfplotstabletypeset[
font=\scriptsize,
	every head row/.style={
    output empty row,
		after row={%
            \midrule
              \multicolumn{1}{c|}{\textbf{\emph{Alphabet Size}} $\downarrow$}
            & \multicolumn{5}{c|}{\textbf{NotRespondedExistence[?X,?Y]}}
            \\\midrule
		},
	},
	%
	columns={alphSizeQueryTEN,tr20QueryTEN,tr25QueryTEN,tr30QueryTEN,tr40QueryTEN,tr50QueryTEN},	 columns/{alphSizeQueryTEN}/.style={column name={},column type=C},	
    columns/{tr20QueryTEN}/.style={column name={},column type=C},
    columns/{tr25QueryTEN}/.style={column name={},column type=C},
	columns/{tr30QueryTEN}/.style={column name={},column type=C},
    columns/{tr40QueryTEN}/.style={column name={},column type=C},
	columns/{tr50QueryTEN}/.style={column name={},column type=C},
	%%	
]{\fileQueryTEN}
}
}
%%%
%%%
\scalebox{1.}[0.75]{
\resizebox{\columnwidth}{!}{
\newcolumntype{C}{>{\centering\arraybackslash}p{21mm}}
\pgfplotstabletypeset[
font=\scriptsize,
	every head row/.style={
    output empty row,
		after row={%
            \midrule
              \multicolumn{1}{c|}{\textbf{\emph{Alphabet Size}} $\downarrow$}
            & \multicolumn{5}{c|}{\textbf{NotResponse[?X,?Y]}}
            \\\midrule
		},
	},
	%
	columns={alphSizeQueryELEVEN,tr20QueryELEVEN,tr25QueryELEVEN,tr30QueryELEVEN,tr40QueryELEVEN,tr50QueryELEVEN},	 columns/{alphSizeQueryELEVEN}/.style={column name={},column type=C},	
    columns/{tr20QueryELEVEN}/.style={column name={},column type=C},
    columns/{tr25QueryELEVEN}/.style={column name={},column type=C},
	columns/{tr30QueryELEVEN}/.style={column name={},column type=C},
    columns/{tr40QueryELEVEN}/.style={column name={},column type=C},
	columns/{tr50QueryELEVEN}/.style={column name={},column type=C},
	%%	
]{\fileQueryELEVEN}
}
}
%%%
%%%
\scalebox{1.}[0.75]{
\resizebox{\columnwidth}{!}{
\newcolumntype{C}{>{\centering\arraybackslash}p{21mm}}
\pgfplotstabletypeset[
font=\scriptsize,
	every head row/.style={
    output empty row,
		after row={%
            \midrule
              \multicolumn{1}{c|}{\textbf{\emph{Alphabet Size}} $\downarrow$}
            & \multicolumn{5}{c|}{\textbf{NotChainResponse[?X,?Y]}}
            \\\midrule
		},
	},
	%
	columns={alphSizeQueryTWELVE,tr20QueryTWELVE,tr25QueryTWELVE,tr30QueryTWELVE,tr40QueryTWELVE,tr50QueryTWELVE},	 columns/{alphSizeQueryTWELVE}/.style={column name={},column type=C},	
    columns/{tr20QueryTWELVE}/.style={column name={},column type=C},
    columns/{tr25QueryTWELVE}/.style={column name={},column type=C},
	columns/{tr30QueryTWELVE}/.style={column name={},column type=C},
    columns/{tr40QueryTWELVE}/.style={column name={},column type=C},
	columns/{tr50QueryTWELVE}/.style={column name={},column type=C},
	%%	
]{\fileQueryTWELVE}
}
}
\caption{Time (in ms.) required for executing queries (without data conditions) of different types on traces of different lengths and with alphabet of possible activities of different sizes.}
\label{table:exp_results_real}
\end{center}
\end{table}
%\pgfplotstableread[col sep=comma,header=true]{experiments/query25.csv}\fileQueryTWENTYFIVE

\pgfplotstablecreatecol[copy column from table={\fileQueryTWENTYFIVE}{[index] 0}] {alphSizeQueryTWENTYFIVE} {\fileQueryTWENTYFIVE}
\pgfplotstablecreatecol[copy column from table={\fileQueryTWENTYFIVE}{[index] 1}] {tr20QueryTWENTYFIVE} {\fileQueryTWENTYFIVE}
\pgfplotstablecreatecol[copy column from table={\fileQueryTWENTYFIVE}{[index] 2}] {tr25QueryTWENTYFIVE} {\fileQueryTWENTYFIVE}
\pgfplotstablecreatecol[copy column from table={\fileQueryTWENTYFIVE}{[index] 3}] {tr30QueryTWENTYFIVE} {\fileQueryTWENTYFIVE}
\pgfplotstablecreatecol[copy column from table={\fileQueryTWENTYFIVE}{[index] 4}] {tr40QueryTWENTYFIVE} {\fileQueryTWENTYFIVE}
\pgfplotstablecreatecol[copy column from table={\fileQueryTWENTYFIVE}{[index] 5}] {tr50QueryTWENTYFIVE} {\fileQueryTWENTYFIVE}

\pgfplotstableread[col sep=comma,header=true]{experiments/query14.csv}\fileQueryFOURTEEN

\pgfplotstablecreatecol[copy column from table={\fileQueryFOURTEEN}{[index] 0}] {alphSizeQueryFOURTEEN} {\fileQueryFOURTEEN}
\pgfplotstablecreatecol[copy column from table={\fileQueryFOURTEEN}{[index] 1}] {tr20QueryFOURTEEN} {\fileQueryFOURTEEN}
\pgfplotstablecreatecol[copy column from table={\fileQueryFOURTEEN}{[index] 2}] {tr25QueryFOURTEEN} {\fileQueryFOURTEEN}
\pgfplotstablecreatecol[copy column from table={\fileQueryFOURTEEN}{[index] 3}] {tr30QueryFOURTEEN} {\fileQueryFOURTEEN}
\pgfplotstablecreatecol[copy column from table={\fileQueryFOURTEEN}{[index] 4}] {tr40QueryFOURTEEN}  {\fileQueryFOURTEEN}
\pgfplotstablecreatecol[copy column from table={\fileQueryFOURTEEN}{[index] 5}] {tr50QueryFOURTEEN}  {\fileQueryFOURTEEN}

\pgfplotstableread[col sep=comma,header=true]{experiments/query15.csv}\fileQueryFIFTEEN

\pgfplotstablecreatecol[copy column from table={\fileQueryFIFTEEN}{[index] 0}] {alphSizeQueryFIFTEEN} {\fileQueryFIFTEEN}
\pgfplotstablecreatecol[copy column from table={\fileQueryFIFTEEN}{[index] 1}] {tr20QueryFIFTEEN} {\fileQueryFIFTEEN}
\pgfplotstablecreatecol[copy column from table={\fileQueryFIFTEEN}{[index] 2}] {tr25QueryFIFTEEN} {\fileQueryFIFTEEN}
\pgfplotstablecreatecol[copy column from table={\fileQueryFIFTEEN}{[index] 3}] {tr30QueryFIFTEEN} {\fileQueryFIFTEEN}
\pgfplotstablecreatecol[copy column from table={\fileQueryFIFTEEN}{[index] 4}] {tr40QueryFIFTEEN}  {\fileQueryFIFTEEN}
\pgfplotstablecreatecol[copy column from table={\fileQueryFIFTEEN}{[index] 5}] {tr50QueryFIFTEEN}  {\fileQueryFIFTEEN}

\pgfplotstableread[col sep=comma,header=true]{experiments/query26.csv}\fileQueryTWENTYSIX

\pgfplotstablecreatecol[copy column from table={\fileQueryTWENTYSIX}{[index] 0}] {alphSizeQueryTWENTYSIX} {\fileQueryTWENTYSIX}
\pgfplotstablecreatecol[copy column from table={\fileQueryTWENTYSIX}{[index] 1}] {tr20QueryTWENTYSIX} {\fileQueryTWENTYSIX}
\pgfplotstablecreatecol[copy column from table={\fileQueryTWENTYSIX}{[index] 2}] {tr25QueryTWENTYSIX} {\fileQueryTWENTYSIX}
\pgfplotstablecreatecol[copy column from table={\fileQueryTWENTYSIX}{[index] 3}] {tr30QueryTWENTYSIX} {\fileQueryTWENTYSIX}
\pgfplotstablecreatecol[copy column from table={\fileQueryTWENTYSIX}{[index] 4}] {tr40QueryTWENTYSIX}  {\fileQueryTWENTYSIX}
\pgfplotstablecreatecol[copy column from table={\fileQueryTWENTYSIX}{[index] 5}] {tr50QueryTWENTYSIX}  {\fileQueryTWENTYSIX}

\pgfplotstableread[col sep=comma,header=true]{experiments/query27.csv}\fileQueryTWENTYSEVEN

\pgfplotstablecreatecol[copy column from table={\fileQueryTWENTYSEVEN}{[index] 0}] {alphSizeQueryTWENTYSEVEN} {\fileQueryTWENTYSEVEN}
\pgfplotstablecreatecol[copy column from table={\fileQueryTWENTYSEVEN}{[index] 1}] {tr20QueryTWENTYSEVEN} {\fileQueryTWENTYSEVEN}
\pgfplotstablecreatecol[copy column from table={\fileQueryTWENTYSEVEN}{[index] 2}] {tr25QueryTWENTYSEVEN} {\fileQueryTWENTYSEVEN}
\pgfplotstablecreatecol[copy column from table={\fileQueryTWENTYSEVEN}{[index] 3}] {tr30QueryTWENTYSEVEN} {\fileQueryTWENTYSEVEN}
\pgfplotstablecreatecol[copy column from table={\fileQueryTWENTYSEVEN}{[index] 4}] {tr40QueryTWENTYSEVEN}  {\fileQueryTWENTYSEVEN}
\pgfplotstablecreatecol[copy column from table={\fileQueryTWENTYSEVEN}{[index] 5}] {tr50QueryTWENTYSEVEN}  {\fileQueryTWENTYSEVEN}

\pgfplotstableread[col sep=comma,header=true]{experiments/query28.csv}\fileQueryTWENTYEIGHT

\pgfplotstablecreatecol[copy column from table={\fileQueryTWENTYEIGHT}{[index] 0}] {alphSizeQueryTWENTYEIGHT} {\fileQueryTWENTYEIGHT}
\pgfplotstablecreatecol[copy column from table={\fileQueryTWENTYEIGHT}{[index] 1}] {tr20QueryTWENTYEIGHT} {\fileQueryTWENTYEIGHT}
\pgfplotstablecreatecol[copy column from table={\fileQueryTWENTYEIGHT}{[index] 2}] {tr25QueryTWENTYEIGHT} {\fileQueryTWENTYEIGHT}
\pgfplotstablecreatecol[copy column from table={\fileQueryTWENTYEIGHT}{[index] 3}] {tr30QueryTWENTYEIGHT} {\fileQueryTWENTYEIGHT}
\pgfplotstablecreatecol[copy column from table={\fileQueryTWENTYEIGHT}{[index] 4}] {tr40QueryTWENTYEIGHT}  {\fileQueryTWENTYEIGHT}
\pgfplotstablecreatecol[copy column from table={\fileQueryTWENTYEIGHT}{[index] 5}] {tr50QueryTWENTYEIGHT}  {\fileQueryTWENTYEIGHT}

\pgfplotstableread[col sep=comma,header=true]{experiments/query29.csv}\fileQueryTWENTYNINE

\pgfplotstablecreatecol[copy column from table={\fileQueryTWENTYNINE}{[index] 0}] {alphSizeQueryTWENTYNINE} {\fileQueryTWENTYNINE}
\pgfplotstablecreatecol[copy column from table={\fileQueryTWENTYNINE}{[index] 1}] {tr20QueryTWENTYNINE} {\fileQueryTWENTYNINE}
\pgfplotstablecreatecol[copy column from table={\fileQueryTWENTYNINE}{[index] 2}] {tr25QueryTWENTYNINE} {\fileQueryTWENTYNINE}
\pgfplotstablecreatecol[copy column from table={\fileQueryTWENTYNINE}{[index] 3}] {tr30QueryTWENTYNINE} {\fileQueryTWENTYNINE}
\pgfplotstablecreatecol[copy column from table={\fileQueryTWENTYNINE}{[index] 4}] {tr40QueryTWENTYNINE}  {\fileQueryTWENTYNINE}
\pgfplotstablecreatecol[copy column from table={\fileQueryTWENTYNINE}{[index] 5}] {tr50QueryTWENTYNINE}  {\fileQueryTWENTYNINE}

\pgfplotstableread[col sep=comma,header=true]{experiments/query30.csv}\fileQueryTHIRTY

\pgfplotstablecreatecol[copy column from table={\fileQueryTHIRTY}{[index] 0}] {alphSizeQueryTHIRTY} {\fileQueryTHIRTY}
\pgfplotstablecreatecol[copy column from table={\fileQueryTHIRTY}{[index] 1}] {tr20QueryTHIRTY} {\fileQueryTHIRTY}
\pgfplotstablecreatecol[copy column from table={\fileQueryTHIRTY}{[index] 2}] {tr25QueryTHIRTY} {\fileQueryTHIRTY}
\pgfplotstablecreatecol[copy column from table={\fileQueryTHIRTY}{[index] 3}] {tr30QueryTHIRTY} {\fileQueryTHIRTY}
\pgfplotstablecreatecol[copy column from table={\fileQueryTHIRTY}{[index] 4}] {tr40QueryTHIRTY}  {\fileQueryTHIRTY}
\pgfplotstablecreatecol[copy column from table={\fileQueryTHIRTY}{[index] 5}] {tr50QueryTHIRTY}  {\fileQueryTHIRTY}

\pgfplotstableread[col sep=comma,header=true]{experiments/query31.csv}\fileQueryTHIRTYONE

\pgfplotstablecreatecol[copy column from table={\fileQueryTHIRTYONE}{[index] 0}] {alphSizeQueryTHIRTYONE} {\fileQueryTHIRTYONE}
\pgfplotstablecreatecol[copy column from table={\fileQueryTHIRTYONE}{[index] 1}] {tr20QueryTHIRTYONE} {\fileQueryTHIRTYONE}
\pgfplotstablecreatecol[copy column from table={\fileQueryTHIRTYONE}{[index] 2}] {tr25QueryTHIRTYONE} {\fileQueryTHIRTYONE}
\pgfplotstablecreatecol[copy column from table={\fileQueryTHIRTYONE}{[index] 3}] {tr30QueryTHIRTYONE} {\fileQueryTHIRTYONE}
\pgfplotstablecreatecol[copy column from table={\fileQueryTHIRTYONE}{[index] 4}] {tr40QueryTHIRTYONE}  {\fileQueryTHIRTYONE}
\pgfplotstablecreatecol[copy column from table={\fileQueryTHIRTYONE}{[index] 5}] {tr50QueryTHIRTYONE}  {\fileQueryTHIRTYONE}

\pgfplotstableread[col sep=comma,header=true]{experiments/query32.csv}\fileQueryTHIRTYTWO

\pgfplotstablecreatecol[copy column from table={\fileQueryTHIRTYTWO}{[index] 0}] {alphSizeQueryTHIRTYTWO} {\fileQueryTHIRTYTWO}
\pgfplotstablecreatecol[copy column from table={\fileQueryTHIRTYTWO}{[index] 1}] {tr20QueryTHIRTYTWO} {\fileQueryTHIRTYTWO}
\pgfplotstablecreatecol[copy column from table={\fileQueryTHIRTYTWO}{[index] 2}] {tr25QueryTHIRTYTWO} {\fileQueryTHIRTYTWO}
\pgfplotstablecreatecol[copy column from table={\fileQueryTHIRTYTWO}{[index] 3}] {tr30QueryTHIRTYTWO} {\fileQueryTHIRTYTWO}
\pgfplotstablecreatecol[copy column from table={\fileQueryTHIRTYTWO}{[index] 4}] {tr40QueryTHIRTYTWO}  {\fileQueryTHIRTYTWO}
\pgfplotstablecreatecol[copy column from table={\fileQueryTHIRTYTWO}{[index] 5}] {tr50QueryTHIRTYTWO}  {\fileQueryTHIRTYTWO}

\pgfplotstableread[col sep=comma,header=true]{experiments/query33.csv}\fileQueryTHIRTYTHREE

\pgfplotstablecreatecol[copy column from table={\fileQueryTHIRTYTHREE}{[index] 0}] {alphSizeQueryTHIRTYTHREE} {\fileQueryTHIRTYTHREE}
\pgfplotstablecreatecol[copy column from table={\fileQueryTHIRTYTHREE}{[index] 1}] {tr20QueryTHIRTYTHREE} {\fileQueryTHIRTYTHREE}
\pgfplotstablecreatecol[copy column from table={\fileQueryTHIRTYTHREE}{[index] 2}] {tr25QueryTHIRTYTHREE} {\fileQueryTHIRTYTHREE}
\pgfplotstablecreatecol[copy column from table={\fileQueryTHIRTYTHREE}{[index] 3}] {tr30QueryTHIRTYTHREE} {\fileQueryTHIRTYTHREE}
\pgfplotstablecreatecol[copy column from table={\fileQueryTHIRTYTHREE}{[index] 4}] {tr40QueryTHIRTYTHREE}  {\fileQueryTHIRTYTHREE}
\pgfplotstablecreatecol[copy column from table={\fileQueryTHIRTYTHREE}{[index] 5}] {tr50QueryTHIRTYTHREE}  {\fileQueryTHIRTYTHREE}

\pgfplotstableread[col sep=comma,header=true]{experiments/query34.csv}\fileQueryTHIRTYFOUR

\pgfplotstablecreatecol[copy column from table={\fileQueryTHIRTYFOUR}{[index] 0}] {alphSizeQueryTHIRTYFOUR} {\fileQueryTHIRTYFOUR}
\pgfplotstablecreatecol[copy column from table={\fileQueryTHIRTYFOUR}{[index] 1}] {tr20QueryTHIRTYFOUR} {\fileQueryTHIRTYFOUR}
\pgfplotstablecreatecol[copy column from table={\fileQueryTHIRTYFOUR}{[index] 2}] {tr25QueryTHIRTYFOUR} {\fileQueryTHIRTYFOUR}
\pgfplotstablecreatecol[copy column from table={\fileQueryTHIRTYFOUR}{[index] 3}] {tr30QueryTHIRTYFOUR} {\fileQueryTHIRTYFOUR}
\pgfplotstablecreatecol[copy column from table={\fileQueryTHIRTYFOUR}{[index] 4}] {tr40QueryTHIRTYFOUR}  {\fileQueryTHIRTYFOUR}
\pgfplotstablecreatecol[copy column from table={\fileQueryTHIRTYFOUR}{[index] 5}] {tr50QueryTHIRTYFOUR}  {\fileQueryTHIRTYFOUR}


\begin{table}
\begin{center}
\scalebox{1.}[0.75]{
\resizebox{\columnwidth}{!}{
\newcolumntype{C}{>{\centering\arraybackslash}p{21mm}}
\pgfplotstabletypeset[
font=\scriptsize,
	every head row/.style={
    output empty row,
		before row={%
              \toprule
              Trace length $\rightarrow$
            & 20
            & 25
            & 30
            & 40
            & 50
            \\
		},
		after row={%
            \midrule
              \multicolumn{1}{c|}{\textbf{\emph{Alphabet Size}} $\downarrow$}
            & \multicolumn{5}{c|}{\textbf{Response[?X,?Y]$||$?}}
            \\\midrule
		},
	},
	%
	 columns={alphSizeQueryTWENTYFIVE,tr20QueryTWENTYFIVE,tr25QueryTWENTYFIVE,tr30QueryTWENTYFIVE,tr40QueryTWENTYFIVE,tr50QueryTWENTYFIVE},	
    columns/{alphSizeQueryTWENTYFIVE}/.style={column name={},column type=C},	
    columns/{tr20QueryTWENTYFIVE}/.style={column name={},column type=C},
    columns/{tr25QueryTWENTYFIVE}/.style={column name={},column type=C},
	columns/{tr30QueryTWENTYFIVE}/.style={column name={},column type=C},
    columns/{tr40QueryTWENTYFIVE}/.style={column name={},column type=C},
	columns/{tr50QueryTWENTYFIVE}/.style={column name={},column type=C},
	%%	
    %%every row 0 column 0/.style={postproc cell content/.style={@cell content={3-50}}},
    %%every row 1 column 0/.style={postproc cell content/.style={@cell content={51-75}}},
    %%every row 2 column 0/.style={postproc cell content/.style={@cell content={76-100}}},
    %%every row 3 column 0/.style={postproc cell content/.style={@cell content={101-128}}},
]{\fileQueryTWENTYFIVE}
}
}
%%%
%%%
\scalebox{1.}[0.75]{
\resizebox{\columnwidth}{!}{
\newcolumntype{C}{>{\centering\arraybackslash}p{21mm}}
\pgfplotstabletypeset[
font=\scriptsize,
	every head row/.style={
    output empty row,
		after row={%
            \midrule
              \multicolumn{1}{c|}{\textbf{\emph{Alphabet Size}} $\downarrow$}
            & \multicolumn{5}{c|}{\textbf{Existence[?X]$|$?}}
            \\\midrule
		},
	},
	%
	columns={alphSizeQueryFOURTEEN,tr20QueryFOURTEEN,tr25QueryFOURTEEN,tr30QueryFOURTEEN,tr40QueryFOURTEEN,tr50QueryFOURTEEN},	 columns/{alphSizeQueryFOURTEEN}/.style={column name={},column type=C},	
    columns/{tr20QueryFOURTEEN}/.style={column name={},column type=C},
    columns/{tr25QueryFOURTEEN}/.style={column name={},column type=C},
	columns/{tr30QueryFOURTEEN}/.style={column name={},column type=C},
    columns/{tr40QueryFOURTEEN}/.style={column name={},column type=C},
	columns/{tr50QueryFOURTEEN}/.style={column name={},column type=C},
	%%	
]{\fileQueryFOURTEEN}
}
}
%%%
%%%
\scalebox{1.}[0.75]{
\resizebox{\columnwidth}{!}{
\newcolumntype{C}{>{\centering\arraybackslash}p{21mm}}
\pgfplotstabletypeset[
font=\scriptsize,
	every head row/.style={
    output empty row,
		after row={%
            \midrule
              \multicolumn{1}{c|}{\textbf{\emph{Alphabet Size}} $\downarrow$}
            & \multicolumn{5}{c|}{\textbf{Absence[?X]$|$?}}
            \\\midrule
		},
	},
	%
	columns={alphSizeQueryFIFTEEN,tr20QueryFIFTEEN,tr25QueryFIFTEEN,tr30QueryFIFTEEN,tr40QueryFIFTEEN,tr50QueryFIFTEEN},	 columns/{alphSizeQueryFIFTEEN}/.style={column name={},column type=C},	
    columns/{tr20QueryFIFTEEN}/.style={column name={},column type=C},
    columns/{tr25QueryFIFTEEN}/.style={column name={},column type=C},
	columns/{tr30QueryFIFTEEN}/.style={column name={},column type=C},
    columns/{tr40QueryFIFTEEN}/.style={column name={},column type=C},
	columns/{tr50QueryFIFTEEN}/.style={column name={},column type=C},
	%%	
]{\fileQueryFIFTEEN}
}
}
%%%
%%%
\scalebox{1.}[0.75]{
\resizebox{\columnwidth}{!}{
\newcolumntype{C}{>{\centering\arraybackslash}p{21mm}}
\pgfplotstabletypeset[
font=\scriptsize,
	every head row/.style={
    output empty row,
		after row={%
            \midrule
              \multicolumn{1}{c|}{\textbf{\emph{Alphabet Size}} $\downarrow$}
            & \multicolumn{5}{c|}{\textbf{RespondedExistence[?X,?Y]$||$?}}
            \\\midrule
		},
	},
	%
	columns={alphSizeQueryTWENTYSIX,tr20QueryTWENTYSIX,tr25QueryTWENTYSIX,tr30QueryTWENTYSIX,tr40QueryTWENTYSIX,tr50QueryTWENTYSIX},	 columns/{alphSizeQueryTWENTYSIX}/.style={column name={},column type=C},	
    columns/{tr20QueryTWENTYSIX}/.style={column name={},column type=C},
    columns/{tr25QueryTWENTYSIX}/.style={column name={},column type=C},
	columns/{tr30QueryTWENTYSIX}/.style={column name={},column type=C},
    columns/{tr40QueryTWENTYSIX}/.style={column name={},column type=C},
	columns/{tr50QueryTWENTYSIX}/.style={column name={},column type=C},
	%%	
]{\fileQueryTWENTYSIX}
}
}
%%%
%%%
\scalebox{1.}[0.75]{
\resizebox{\columnwidth}{!}{
\newcolumntype{C}{>{\centering\arraybackslash}p{21mm}}
\pgfplotstabletypeset[
font=\scriptsize,
	every head row/.style={
    output empty row,
		after row={%
            \midrule
              \multicolumn{1}{c|}{\textbf{\emph{Alphabet Size}} $\downarrow$}
            & \multicolumn{5}{c|}{\textbf{AlternateResponse[?X,?Y]$||$?}}
            \\\midrule
		},
	},
	%
	 columns={alphSizeQueryTWENTYSEVEN,tr20QueryTWENTYSEVEN,tr25QueryTWENTYSEVEN,tr30QueryTWENTYSEVEN,tr40QueryTWENTYSEVEN,tr50QueryTWENTYSEVEN},	 columns/{alphSizeQueryTWENTYSEVEN}/.style={column name={},column type=C},	
    columns/{tr20QueryTWENTYSEVEN}/.style={column name={},column type=C},
    columns/{tr25QueryTWENTYSEVEN}/.style={column name={},column type=C},
	columns/{tr30QueryTWENTYSEVEN}/.style={column name={},column type=C},
    columns/{tr40QueryTWENTYSEVEN}/.style={column name={},column type=C},
	columns/{tr50QueryTWENTYSEVEN}/.style={column name={},column type=C},
	%%	
]{\fileQueryTWENTYSEVEN}
}
}
%%%
%%%
\scalebox{1.}[0.75]{
\resizebox{\columnwidth}{!}{
\newcolumntype{C}{>{\centering\arraybackslash}p{21mm}}
\pgfplotstabletypeset[
font=\scriptsize,
	every head row/.style={
    output empty row,
		after row={%
            \midrule
              \multicolumn{1}{c|}{\textbf{\emph{Alphabet Size}} $\downarrow$}
            & \multicolumn{5}{c|}{\textbf{ChainResponse[?X,?Y]$||$?}}
            \\\midrule
		},
	},
	%
	 columns={alphSizeQueryTWENTYEIGHT,tr20QueryTWENTYEIGHT,tr25QueryTWENTYEIGHT,tr30QueryTWENTYEIGHT,tr40QueryTWENTYEIGHT,tr50QueryTWENTYEIGHT},	 columns/{alphSizeQueryTWENTYEIGHT}/.style={column name={},column type=C},	
    columns/{tr20QueryTWENTYEIGHT}/.style={column name={},column type=C},
    columns/{tr25QueryTWENTYEIGHT}/.style={column name={},column type=C},
	columns/{tr30QueryTWENTYEIGHT}/.style={column name={},column type=C},
    columns/{tr40QueryTWENTYEIGHT}/.style={column name={},column type=C},
	columns/{tr50QueryTWENTYEIGHT}/.style={column name={},column type=C},
	%%	
]{\fileQueryTWENTYEIGHT}
}
}
%%%
%%%
\scalebox{1.}[0.75]{
\resizebox{\columnwidth}{!}{
\newcolumntype{C}{>{\centering\arraybackslash}p{21mm}}
\pgfplotstabletypeset[
font=\scriptsize,
	every head row/.style={
    output empty row,
		after row={%
            \midrule
              \multicolumn{1}{c|}{\textbf{\emph{Alphabet Size}} $\downarrow$}
            & \multicolumn{5}{c|}{\textbf{Precedence[?X,?Y]$||$?}}
            \\\midrule
		},
	},
	%
	 columns={alphSizeQueryTWENTYNINE,tr20QueryTWENTYNINE,tr25QueryTWENTYNINE,tr30QueryTWENTYNINE,tr40QueryTWENTYNINE,tr50QueryTWENTYNINE},	 columns/{alphSizeQueryTWENTYNINE}/.style={column name={},column type=C},	
    columns/{tr20QueryTWENTYNINE}/.style={column name={},column type=C},
    columns/{tr25QueryTWENTYNINE}/.style={column name={},column type=C},
	columns/{tr30QueryTWENTYNINE}/.style={column name={},column type=C},
    columns/{tr40QueryTWENTYNINE}/.style={column name={},column type=C},
	columns/{tr50QueryTWENTYNINE}/.style={column name={},column type=C},
	%%	
]{\fileQueryTWENTYNINE}
}
}
%%%
%%%
\scalebox{1.}[0.75]{
\resizebox{\columnwidth}{!}{
\newcolumntype{C}{>{\centering\arraybackslash}p{21mm}}
\pgfplotstabletypeset[
font=\scriptsize,
	every head row/.style={
    output empty row,
		after row={%
            \midrule
              \multicolumn{1}{c|}{\textbf{\emph{Alphabet Size}} $\downarrow$}
            & \multicolumn{5}{c|}{\textbf{AlternatePrecedence[?X,?Y]$||$?}}
            \\\midrule
		},
	},
	%
	columns={alphSizeQueryTHIRTY,tr20QueryTHIRTY,tr25QueryTHIRTY,tr30QueryTHIRTY,tr40QueryTHIRTY,tr50QueryTHIRTY},	 columns/{alphSizeQueryTHIRTY}/.style={column name={},column type=C},	
    columns/{tr20QueryTHIRTY}/.style={column name={},column type=C},
    columns/{tr25QueryTHIRTY}/.style={column name={},column type=C},
	columns/{tr30QueryTHIRTY}/.style={column name={},column type=C},
    columns/{tr40QueryTHIRTY}/.style={column name={},column type=C},
	columns/{tr50QueryTHIRTY}/.style={column name={},column type=C},
	%%	
]{\fileQueryTHIRTY}
}
}
%%%
%%%
\scalebox{1.}[0.75]{
\resizebox{\columnwidth}{!}{
\newcolumntype{C}{>{\centering\arraybackslash}p{21mm}}
\pgfplotstabletypeset[
font=\scriptsize,
	every head row/.style={
    output empty row,
		after row={%
            \midrule
              \multicolumn{1}{c|}{\textbf{\emph{Alphabet Size}} $\downarrow$}
            & \multicolumn{5}{c|}{\textbf{ChainPrecedence[?X,?Y]$||$?}}
            \\\midrule
		},
	},
	%
	columns={alphSizeQueryTHIRTYONE,tr20QueryTHIRTYONE,tr25QueryTHIRTYONE,tr30QueryTHIRTYONE,tr40QueryTHIRTYONE,tr50QueryTHIRTYONE},	 columns/{alphSizeQueryTHIRTYONE}/.style={column name={},column type=C},	
    columns/{tr20QueryTHIRTYONE}/.style={column name={},column type=C},
    columns/{tr25QueryTHIRTYONE}/.style={column name={},column type=C},
	columns/{tr30QueryTHIRTYONE}/.style={column name={},column type=C},
    columns/{tr40QueryTHIRTYONE}/.style={column name={},column type=C},
	columns/{tr50QueryTHIRTYONE}/.style={column name={},column type=C},
	%%	
]{\fileQueryTHIRTYONE}
}
}
%%%
%%%
\scalebox{1.}[0.75]{
\resizebox{\columnwidth}{!}{
\newcolumntype{C}{>{\centering\arraybackslash}p{21mm}}
\pgfplotstabletypeset[
font=\scriptsize,
	every head row/.style={
    output empty row,
		after row={%
            \midrule
              \multicolumn{1}{c|}{\textbf{\emph{Alphabet Size}} $\downarrow$}
            & \multicolumn{5}{c|}{\textbf{NotRespondedExistence[?X,?Y]$||$?}}
            \\\midrule
		},
	},
	%
	columns={alphSizeQueryTHIRTYTWO,tr20QueryTHIRTYTWO,tr25QueryTHIRTYTWO,tr30QueryTHIRTYTWO,tr40QueryTHIRTYTWO,tr50QueryTHIRTYTWO},	 columns/{alphSizeQueryTHIRTYTWO}/.style={column name={},column type=C},	
    columns/{tr20QueryTHIRTYTWO}/.style={column name={},column type=C},
    columns/{tr25QueryTHIRTYTWO}/.style={column name={},column type=C},
	columns/{tr30QueryTHIRTYTWO}/.style={column name={},column type=C},
    columns/{tr40QueryTHIRTYTWO}/.style={column name={},column type=C},
	columns/{tr50QueryTHIRTYTWO}/.style={column name={},column type=C},
	%%	
]{\fileQueryTHIRTYTWO}
}
}
%%%
%%%
\scalebox{1.}[0.75]{
\resizebox{\columnwidth}{!}{
\newcolumntype{C}{>{\centering\arraybackslash}p{21mm}}
\pgfplotstabletypeset[
font=\scriptsize,
	every head row/.style={
    output empty row,
		after row={%
            \midrule
              \multicolumn{1}{c|}{\textbf{\emph{Alphabet Size}} $\downarrow$}
            & \multicolumn{5}{c|}{\textbf{NotResponse[?X,?Y]$||$?}}
            \\\midrule
		},
	},
	%
	 columns={alphSizeQueryTHIRTYTHREE,tr20QueryTHIRTYTHREE,tr25QueryTHIRTYTHREE,tr30QueryTHIRTYTHREE,tr40QueryTHIRTYTHREE,tr50QueryTHIRTYTHREE},	 columns/{alphSizeQueryTHIRTYTHREE}/.style={column name={},column type=C},	
    columns/{tr20QueryTHIRTYTHREE}/.style={column name={},column type=C},
    columns/{tr25QueryTHIRTYTHREE}/.style={column name={},column type=C},
	columns/{tr30QueryTHIRTYTHREE}/.style={column name={},column type=C},
    columns/{tr40QueryTHIRTYTHREE}/.style={column name={},column type=C},
	columns/{tr50QueryTHIRTYTHREE}/.style={column name={},column type=C},
	%%	
]{\fileQueryTHIRTYTHREE}
}
}
%%%
%%%
\scalebox{1.}[0.75]{
\resizebox{\columnwidth}{!}{
\newcolumntype{C}{>{\centering\arraybackslash}p{21mm}}
\pgfplotstabletypeset[
font=\scriptsize,
	every head row/.style={
    output empty row,
		after row={%
            \midrule
              \multicolumn{1}{c|}{\textbf{\emph{Alphabet Size}} $\downarrow$}
            & \multicolumn{5}{c|}{\textbf{NotChainResponse[?X,?Y]$||$?}}
            \\\midrule
		},
	},
	%
	 columns={alphSizeQueryTHIRTYFOUR,tr20QueryTHIRTYFOUR,tr25QueryTHIRTYFOUR,tr30QueryTHIRTYFOUR,tr40QueryTHIRTYFOUR,tr50QueryTHIRTYFOUR},	 columns/{alphSizeQueryTHIRTYFOUR}/.style={column name={},column type=C},	
    columns/{tr20QueryTHIRTYFOUR}/.style={column name={},column type=C},
    columns/{tr25QueryTHIRTYFOUR}/.style={column name={},column type=C},
	columns/{tr30QueryTHIRTYFOUR}/.style={column name={},column type=C},
    columns/{tr40QueryTHIRTYFOUR}/.style={column name={},column type=C},
	columns/{tr50QueryTHIRTYFOUR}/.style={column name={},column type=C},
	%%	
]{\fileQueryTHIRTYFOUR}
}
}
\caption{Time (in ms.) required for executing queries (with activation conditions) of different types on traces of different lengths and with alphabet of possible activities of different sizes.}
\label{table:exp_results_real}
\end{center}
\end{table}
%\end{tiny} 
%\pgfplotstableread[col sep=comma,header=true]{experiments/query13.csv}\fileQueryTHIRTEEN

\pgfplotstablecreatecol[copy column from table={\fileQueryTHIRTEEN}{[index] 0}] {alphSizeQueryTHIRTEEN} {\fileQueryTHIRTEEN}
\pgfplotstablecreatecol[copy column from table={\fileQueryTHIRTEEN}{[index] 1}] {tr20QueryTHIRTEEN} {\fileQueryTHIRTEEN}
\pgfplotstablecreatecol[copy column from table={\fileQueryTHIRTEEN}{[index] 2}] {tr25QueryTHIRTEEN} {\fileQueryTHIRTEEN}
\pgfplotstablecreatecol[copy column from table={\fileQueryTHIRTEEN}{[index] 3}] {tr30QueryTHIRTEEN} {\fileQueryTHIRTEEN}
\pgfplotstablecreatecol[copy column from table={\fileQueryTHIRTEEN}{[index] 4}] {tr40QueryTHIRTEEN} {\fileQueryTHIRTEEN}
\pgfplotstablecreatecol[copy column from table={\fileQueryTHIRTEEN}{[index] 5}] {tr50QueryTHIRTEEN} {\fileQueryTHIRTEEN}

\pgfplotstableread[col sep=comma,header=true]{experiments/query16.csv}\fileQuerySIXTEEN

\pgfplotstablecreatecol[copy column from table={\fileQuerySIXTEEN}{[index] 0}] {alphSizeQuerySIXTEEN} {\fileQuerySIXTEEN}
\pgfplotstablecreatecol[copy column from table={\fileQuerySIXTEEN}{[index] 1}] {tr20QuerySIXTEEN} {\fileQuerySIXTEEN}
\pgfplotstablecreatecol[copy column from table={\fileQuerySIXTEEN}{[index] 2}] {tr25QuerySIXTEEN} {\fileQuerySIXTEEN}
\pgfplotstablecreatecol[copy column from table={\fileQuerySIXTEEN}{[index] 3}] {tr30QuerySIXTEEN} {\fileQuerySIXTEEN}
\pgfplotstablecreatecol[copy column from table={\fileQuerySIXTEEN}{[index] 4}] {tr40QuerySIXTEEN}  {\fileQuerySIXTEEN}
\pgfplotstablecreatecol[copy column from table={\fileQuerySIXTEEN}{[index] 5}] {tr50QuerySIXTEEN}  {\fileQuerySIXTEEN}

\pgfplotstableread[col sep=comma,header=true]{experiments/query17.csv}\fileQuerySEVENTEEN

\pgfplotstablecreatecol[copy column from table={\fileQuerySEVENTEEN}{[index] 0}] {alphSizeQuerySEVENTEEN} {\fileQuerySEVENTEEN}
\pgfplotstablecreatecol[copy column from table={\fileQuerySEVENTEEN}{[index] 1}] {tr20QuerySEVENTEEN} {\fileQuerySEVENTEEN}
\pgfplotstablecreatecol[copy column from table={\fileQuerySEVENTEEN}{[index] 2}] {tr25QuerySEVENTEEN} {\fileQuerySEVENTEEN}
\pgfplotstablecreatecol[copy column from table={\fileQuerySEVENTEEN}{[index] 3}] {tr30QuerySEVENTEEN} {\fileQuerySEVENTEEN}
\pgfplotstablecreatecol[copy column from table={\fileQuerySEVENTEEN}{[index] 4}] {tr40QuerySEVENTEEN}  {\fileQuerySEVENTEEN}
\pgfplotstablecreatecol[copy column from table={\fileQuerySEVENTEEN}{[index] 5}] {tr50QuerySEVENTEEN}  {\fileQuerySEVENTEEN}

\pgfplotstableread[col sep=comma,header=true]{experiments/query18.csv}\fileQueryEIGHTEEN

\pgfplotstablecreatecol[copy column from table={\fileQueryEIGHTEEN}{[index] 0}] {alphSizeQueryEIGHTEEN} {\fileQueryEIGHTEEN}
\pgfplotstablecreatecol[copy column from table={\fileQueryEIGHTEEN}{[index] 1}] {tr20QueryEIGHTEEN} {\fileQueryEIGHTEEN}
\pgfplotstablecreatecol[copy column from table={\fileQueryEIGHTEEN}{[index] 2}] {tr25QueryEIGHTEEN} {\fileQueryEIGHTEEN}
\pgfplotstablecreatecol[copy column from table={\fileQueryEIGHTEEN}{[index] 3}] {tr30QueryEIGHTEEN} {\fileQueryEIGHTEEN}
\pgfplotstablecreatecol[copy column from table={\fileQueryEIGHTEEN}{[index] 4}] {tr40QueryEIGHTEEN}  {\fileQueryEIGHTEEN}
\pgfplotstablecreatecol[copy column from table={\fileQueryEIGHTEEN}{[index] 5}] {tr50QueryEIGHTEEN}  {\fileQueryEIGHTEEN}

\pgfplotstableread[col sep=comma,header=true]{experiments/query19.csv}\fileQueryNINETEEN

\pgfplotstablecreatecol[copy column from table={\fileQueryNINETEEN}{[index] 0}] {alphSizeQueryNINETEEN} {\fileQueryNINETEEN}
\pgfplotstablecreatecol[copy column from table={\fileQueryNINETEEN}{[index] 1}] {tr20QueryNINETEEN} {\fileQueryNINETEEN}
\pgfplotstablecreatecol[copy column from table={\fileQueryNINETEEN}{[index] 2}] {tr25QueryNINETEEN} {\fileQueryNINETEEN}
\pgfplotstablecreatecol[copy column from table={\fileQueryNINETEEN}{[index] 3}] {tr30QueryNINETEEN} {\fileQueryNINETEEN}
\pgfplotstablecreatecol[copy column from table={\fileQueryNINETEEN}{[index] 4}] {tr40QueryNINETEEN}  {\fileQueryNINETEEN}
\pgfplotstablecreatecol[copy column from table={\fileQueryNINETEEN}{[index] 5}] {tr50QueryNINETEEN}  {\fileQueryNINETEEN}

\pgfplotstableread[col sep=comma,header=true]{experiments/query20.csv}\fileQueryTWENTY

\pgfplotstablecreatecol[copy column from table={\fileQueryTWENTY}{[index] 0}] {alphSizeQueryTWENTY} {\fileQueryTWENTY}
\pgfplotstablecreatecol[copy column from table={\fileQueryTWENTY}{[index] 1}] {tr20QueryTWENTY} {\fileQueryTWENTY}
\pgfplotstablecreatecol[copy column from table={\fileQueryTWENTY}{[index] 2}] {tr25QueryTWENTY} {\fileQueryTWENTY}
\pgfplotstablecreatecol[copy column from table={\fileQueryTWENTY}{[index] 3}] {tr30QueryTWENTY} {\fileQueryTWENTY}
\pgfplotstablecreatecol[copy column from table={\fileQueryTWENTY}{[index] 4}] {tr40QueryTWENTY}  {\fileQueryTWENTY}
\pgfplotstablecreatecol[copy column from table={\fileQueryTWENTY}{[index] 5}] {tr50QueryTWENTY}  {\fileQueryTWENTY}

\pgfplotstableread[col sep=comma,header=true]{experiments/query21.csv}\fileQueryTWENTYONE

\pgfplotstablecreatecol[copy column from table={\fileQueryTWENTYONE}{[index] 0}] {alphSizeQueryTWENTYONE} {\fileQueryTWENTYONE}
\pgfplotstablecreatecol[copy column from table={\fileQueryTWENTYONE}{[index] 1}] {tr20QueryTWENTYONE} {\fileQueryTWENTYONE}
\pgfplotstablecreatecol[copy column from table={\fileQueryTWENTYONE}{[index] 2}] {tr25QueryTWENTYONE} {\fileQueryTWENTYONE}
\pgfplotstablecreatecol[copy column from table={\fileQueryTWENTYONE}{[index] 3}] {tr30QueryTWENTYONE} {\fileQueryTWENTYONE}
\pgfplotstablecreatecol[copy column from table={\fileQueryTWENTYONE}{[index] 4}] {tr40QueryTWENTYONE}  {\fileQueryTWENTYONE}
\pgfplotstablecreatecol[copy column from table={\fileQueryTWENTYONE}{[index] 5}] {tr50QueryTWENTYONE}  {\fileQueryTWENTYONE}

\pgfplotstableread[col sep=comma,header=true]{experiments/query22.csv}\fileQueryTWENTYTWO

\pgfplotstablecreatecol[copy column from table={\fileQueryTWENTYTWO}{[index] 0}] {alphSizeQueryTWENTYTWO} {\fileQueryTWENTYTWO}
\pgfplotstablecreatecol[copy column from table={\fileQueryTWENTYTWO}{[index] 1}] {tr20QueryTWENTYTWO} {\fileQueryTWENTYTWO}
\pgfplotstablecreatecol[copy column from table={\fileQueryTWENTYTWO}{[index] 2}] {tr25QueryTWENTYTWO} {\fileQueryTWENTYTWO}
\pgfplotstablecreatecol[copy column from table={\fileQueryTWENTYTWO}{[index] 3}] {tr30QueryTWENTYTWO} {\fileQueryTWENTYTWO}
\pgfplotstablecreatecol[copy column from table={\fileQueryTWENTYTWO}{[index] 4}] {tr40QueryTWENTYTWO}  {\fileQueryTWENTYTWO}
\pgfplotstablecreatecol[copy column from table={\fileQueryTWENTYTWO}{[index] 5}] {tr50QueryTWENTYTWO}  {\fileQueryTWENTYTWO}

\pgfplotstableread[col sep=comma,header=true]{experiments/query23.csv}\fileQueryTWENTYTHREE

\pgfplotstablecreatecol[copy column from table={\fileQueryTWENTYTHREE}{[index] 0}] {alphSizeQueryTWENTYTHREE} {\fileQueryTWENTYTHREE}
\pgfplotstablecreatecol[copy column from table={\fileQueryTWENTYTHREE}{[index] 1}] {tr20QueryTWENTYTHREE} {\fileQueryTWENTYTHREE}
\pgfplotstablecreatecol[copy column from table={\fileQueryTWENTYTHREE}{[index] 2}] {tr25QueryTWENTYTHREE} {\fileQueryTWENTYTHREE}
\pgfplotstablecreatecol[copy column from table={\fileQueryTWENTYTHREE}{[index] 3}] {tr30QueryTWENTYTHREE} {\fileQueryTWENTYTHREE}
\pgfplotstablecreatecol[copy column from table={\fileQueryTWENTYTHREE}{[index] 4}] {tr40QueryTWENTYTHREE}  {\fileQueryTWENTYTHREE}
\pgfplotstablecreatecol[copy column from table={\fileQueryTWENTYTHREE}{[index] 5}] {tr50QueryTWENTYTHREE}  {\fileQueryTWENTYTHREE}

\pgfplotstableread[col sep=comma,header=true]{experiments/query24.csv}\fileQueryTWENTYFOUR

\pgfplotstablecreatecol[copy column from table={\fileQueryTWENTYFOUR}{[index] 0}] {alphSizeQueryTWENTYFOUR} {\fileQueryTWENTYFOUR}
\pgfplotstablecreatecol[copy column from table={\fileQueryTWENTYFOUR}{[index] 1}] {tr20QueryTWENTYFOUR} {\fileQueryTWENTYFOUR}
\pgfplotstablecreatecol[copy column from table={\fileQueryTWENTYFOUR}{[index] 2}] {tr25QueryTWENTYFOUR} {\fileQueryTWENTYFOUR}
\pgfplotstablecreatecol[copy column from table={\fileQueryTWENTYFOUR}{[index] 3}] {tr30QueryTWENTYFOUR} {\fileQueryTWENTYFOUR}
\pgfplotstablecreatecol[copy column from table={\fileQueryTWENTYFOUR}{[index] 4}] {tr40QueryTWENTYFOUR}  {\fileQueryTWENTYFOUR}
\pgfplotstablecreatecol[copy column from table={\fileQueryTWENTYFOUR}{[index] 5}] {tr50QueryTWENTYFOUR}  {\fileQueryTWENTYFOUR}

\begin{tiny}
\begin{table}
\begin{center}
\scalebox{1.}[0.9]{
\resizebox{\columnwidth}{!}{
\newcolumntype{C}{>{\centering\arraybackslash}p{21mm}}
\pgfplotstabletypeset[
font=\scriptsize,
	every head row/.style={
    output empty row,
		before row={%
              \toprule
              Trace length $\rightarrow$
            & 20
            & 25
            & 30
            & 40
            & 50
            \\
		},
		after row={%
            \midrule
              \multicolumn{1}{c|}{\textbf{\emph{Alphabet Size}} $\downarrow$}
            & \multicolumn{5}{c|}{\textbf{Response[?X,?Y]$|$?$|$?}}
            \\\midrule
		},
	},
	%
	columns={alphSizeQueryTHIRTEEN,tr20QueryTHIRTEEN,tr25QueryTHIRTEEN,tr30QueryTHIRTEEN,tr40QueryTHIRTEEN,tr50QueryTHIRTEEN},	
    columns/{alphSizeQueryTHIRTEEN}/.style={column name={},column type=C},	
    columns/{tr20QueryTHIRTEEN}/.style={column name={},column type=C},
    columns/{tr25QueryTHIRTEEN}/.style={column name={},column type=C},
	columns/{tr30QueryTHIRTEEN}/.style={column name={},column type=C},
    columns/{tr40QueryTHIRTEEN}/.style={column name={},column type=C},
	columns/{tr50QueryTHIRTEEN}/.style={column name={},column type=C},
	%%	
    %%every row 0 column 0/.style={postproc cell content/.style={@cell content={3-50}}},
    %%every row 1 column 0/.style={postproc cell content/.style={@cell content={51-75}}},
    %%every row 2 column 0/.style={postproc cell content/.style={@cell content={76-100}}},
    %%every row 3 column 0/.style={postproc cell content/.style={@cell content={101-128}}},
]{\fileQueryTHIRTEEN}
}
}
%%%
%%%
\scalebox{1.}[0.9]{
\resizebox{\columnwidth}{!}{
\newcolumntype{C}{>{\centering\arraybackslash}p{21mm}}
\pgfplotstabletypeset[
font=\scriptsize,
	every head row/.style={
    output empty row,
		after row={%
            \midrule
              \multicolumn{1}{c|}{\textbf{\emph{Alphabet Size}} $\downarrow$}
            & \multicolumn{5}{c|}{\textbf{RespondedExistence[?X,?Y]$|$?$|$?}}
            \\\midrule
		},
	},
	%
	columns={alphSizeQuerySIXTEEN,tr20QuerySIXTEEN,tr25QuerySIXTEEN,tr30QuerySIXTEEN,tr40QuerySIXTEEN,tr50QuerySIXTEEN},	 columns/{alphSizeQuerySIXTEEN}/.style={column name={},column type=C},	
    columns/{tr20QuerySIXTEEN}/.style={column name={},column type=C},
    columns/{tr25QuerySIXTEEN}/.style={column name={},column type=C},
	columns/{tr30QuerySIXTEEN}/.style={column name={},column type=C},
    columns/{tr40QuerySIXTEEN}/.style={column name={},column type=C},
	columns/{tr50QuerySIXTEEN}/.style={column name={},column type=C},
	%%	
]{\fileQuerySIXTEEN}
}
}
%%%
%%%
\scalebox{1.}[0.9]{
\resizebox{\columnwidth}{!}{
\newcolumntype{C}{>{\centering\arraybackslash}p{21mm}}
\pgfplotstabletypeset[
font=\scriptsize,
	every head row/.style={
    output empty row,
		after row={%
            \midrule
              \multicolumn{1}{c|}{\textbf{\emph{Alphabet Size}} $\downarrow$}
            & \multicolumn{5}{c|}{\textbf{AlternateResponse[?X,?Y]$|$?$|$?}}
            \\\midrule
		},
	},
	%
	columns={alphSizeQuerySEVENTEEN,tr20QuerySEVENTEEN,tr25QuerySEVENTEEN,tr30QuerySEVENTEEN,tr40QuerySEVENTEEN,tr50QuerySEVENTEEN},	 columns/{alphSizeQuerySEVENTEEN}/.style={column name={},column type=C},	
    columns/{tr20QuerySEVENTEEN}/.style={column name={},column type=C},
    columns/{tr25QuerySEVENTEEN}/.style={column name={},column type=C},
	columns/{tr30QuerySEVENTEEN}/.style={column name={},column type=C},
    columns/{tr40QuerySEVENTEEN}/.style={column name={},column type=C},
	columns/{tr50QuerySEVENTEEN}/.style={column name={},column type=C},
	%%	
]{\fileQuerySEVENTEEN}
}
}
%%%
%%%
\scalebox{1.}[0.9]{
\resizebox{\columnwidth}{!}{
\newcolumntype{C}{>{\centering\arraybackslash}p{21mm}}
\pgfplotstabletypeset[
font=\scriptsize,
	every head row/.style={
    output empty row,
		after row={%
            \midrule
              \multicolumn{1}{c|}{\textbf{\emph{Alphabet Size}} $\downarrow$}
            & \multicolumn{5}{c|}{\textbf{ChainResponse[?X,?Y]$|$?$|$?}}
            \\\midrule
		},
	},
	%
	columns={alphSizeQueryEIGHTEEN,tr20QueryEIGHTEEN,tr25QueryEIGHTEEN,tr30QueryEIGHTEEN,tr40QueryEIGHTEEN,tr50QueryEIGHTEEN},	 columns/{alphSizeQueryEIGHTEEN}/.style={column name={},column type=C},	
    columns/{tr20QueryEIGHTEEN}/.style={column name={},column type=C},
    columns/{tr25QueryEIGHTEEN}/.style={column name={},column type=C},
	columns/{tr30QueryEIGHTEEN}/.style={column name={},column type=C},
    columns/{tr40QueryEIGHTEEN}/.style={column name={},column type=C},
	columns/{tr50QueryEIGHTEEN}/.style={column name={},column type=C},
	%%	
]{\fileQueryEIGHTEEN}
}
}
%%%
%%%
\scalebox{1.}[0.9]{
\resizebox{\columnwidth}{!}{
\newcolumntype{C}{>{\centering\arraybackslash}p{21mm}}
\pgfplotstabletypeset[
font=\scriptsize,
	every head row/.style={
    output empty row,
		after row={%
            \midrule
              \multicolumn{1}{c|}{\textbf{\emph{Alphabet Size}} $\downarrow$}
            & \multicolumn{5}{c|}{\textbf{Precedence[?X,?Y]$|$?$|$?}}
            \\\midrule
		},
	},
	%
	columns={alphSizeQueryNINETEEN,tr20QueryNINETEEN,tr25QueryNINETEEN,tr30QueryNINETEEN,tr40QueryNINETEEN,tr50QueryNINETEEN},	 columns/{alphSizeQueryNINETEEN}/.style={column name={},column type=C},	
    columns/{tr20QueryNINETEEN}/.style={column name={},column type=C},
    columns/{tr25QueryNINETEEN}/.style={column name={},column type=C},
	columns/{tr30QueryNINETEEN}/.style={column name={},column type=C},
    columns/{tr40QueryNINETEEN}/.style={column name={},column type=C},
	columns/{tr50QueryNINETEEN}/.style={column name={},column type=C},
	%%	
]{\fileQueryNINETEEN}
}
}
%%%
%%%
\scalebox{1.}[0.9]{
\resizebox{\columnwidth}{!}{
\newcolumntype{C}{>{\centering\arraybackslash}p{21mm}}
\pgfplotstabletypeset[
font=\scriptsize,
	every head row/.style={
    output empty row,
		after row={%
            \midrule
              \multicolumn{1}{c|}{\textbf{\emph{Alphabet Size}} $\downarrow$}
            & \multicolumn{5}{c|}{\textbf{AlternatePrecedence[?X,?Y]$|$?$|$?}}
            \\\midrule
		},
	},
	%
	columns={alphSizeQueryTWENTY,tr20QueryTWENTY,tr25QueryTWENTY,tr30QueryTWENTY,tr40QueryTWENTY,tr50QueryTWENTY},	 columns/{alphSizeQueryTWENTY}/.style={column name={},column type=C},	
    columns/{tr20QueryTWENTY}/.style={column name={},column type=C},
    columns/{tr25QueryTWENTY}/.style={column name={},column type=C},
	columns/{tr30QueryTWENTY}/.style={column name={},column type=C},
    columns/{tr40QueryTWENTY}/.style={column name={},column type=C},
	columns/{tr50QueryTWENTY}/.style={column name={},column type=C},
	%%	
]{\fileQueryTWENTY}
}
}
%%%
%%%
\scalebox{1.}[0.9]{
\resizebox{\columnwidth}{!}{
\newcolumntype{C}{>{\centering\arraybackslash}p{21mm}}
\pgfplotstabletypeset[
font=\scriptsize,
	every head row/.style={
    output empty row,
		after row={%
            \midrule
              \multicolumn{1}{c|}{\textbf{\emph{Alphabet Size}} $\downarrow$}
            & \multicolumn{5}{c|}{\textbf{ChainPrecedence[?X,?Y]$|$?$|$?}}
            \\\midrule
		},
	},
	%
	columns={alphSizeQueryTWENTYONE,tr20QueryTWENTYONE,tr25QueryTWENTYONE,tr30QueryTWENTYONE,tr40QueryTWENTYONE,tr50QueryTWENTYONE},	 columns/{alphSizeQueryTWENTYONE}/.style={column name={},column type=C},	
    columns/{tr20QueryTWENTYONE}/.style={column name={},column type=C},
    columns/{tr25QueryTWENTYONE}/.style={column name={},column type=C},
	columns/{tr30QueryTWENTYONE}/.style={column name={},column type=C},
    columns/{tr40QueryTWENTYONE}/.style={column name={},column type=C},
	columns/{tr50QueryTWENTYONE}/.style={column name={},column type=C},
	%%	
]{\fileQueryTWENTYONE}
}
}
%%%
%%%
\scalebox{1.}[0.9]{
\resizebox{\columnwidth}{!}{
\newcolumntype{C}{>{\centering\arraybackslash}p{21mm}}
\pgfplotstabletypeset[
font=\scriptsize,
	every head row/.style={
    output empty row,
		after row={%
            \midrule
              \multicolumn{1}{c|}{\textbf{\emph{Alphabet Size}} $\downarrow$}
            & \multicolumn{5}{c|}{\textbf{NotRespondedExistence[?X,?Y]$|$?$|$?}}
            \\\midrule
		},
	},
	%
	columns={alphSizeQueryTWENTYTWO,tr20QueryTWENTYTWO,tr25QueryTWENTYTWO,tr30QueryTWENTYTWO,tr40QueryTWENTYTWO,tr50QueryTWENTYTWO},	 columns/{alphSizeQueryTWENTYTWO}/.style={column name={},column type=C},	
    columns/{tr20QueryTWENTYTWO}/.style={column name={},column type=C},
    columns/{tr25QueryTWENTYTWO}/.style={column name={},column type=C},
	columns/{tr30QueryTWENTYTWO}/.style={column name={},column type=C},
    columns/{tr40QueryTWENTYTWO}/.style={column name={},column type=C},
	columns/{tr50QueryTWENTYTWO}/.style={column name={},column type=C},
	%%	
]{\fileQueryTWENTYTWO}
}
}
%%%
%%%
\scalebox{1.}[0.9]{
\resizebox{\columnwidth}{!}{
\newcolumntype{C}{>{\centering\arraybackslash}p{21mm}}
\pgfplotstabletypeset[
font=\scriptsize,
	every head row/.style={
    output empty row,
		after row={%
            \midrule
              \multicolumn{1}{c|}{\textbf{\emph{Alphabet Size}} $\downarrow$}
            & \multicolumn{5}{c|}{\textbf{NotResponse[?X,?Y]$|$?$|$?}}
            \\\midrule
		},
	},
	%
	 columns={alphSizeQueryTWENTYTHREE,tr20QueryTWENTYTHREE,tr25QueryTWENTYTHREE,tr30QueryTWENTYTHREE,tr40QueryTWENTYTHREE,tr50QueryTWENTYTHREE},	 columns/{alphSizeQueryTWENTYTHREE}/.style={column name={},column type=C},	
    columns/{tr20QueryTWENTYTHREE}/.style={column name={},column type=C},
    columns/{tr25QueryTWENTYTHREE}/.style={column name={},column type=C},
	columns/{tr30QueryTWENTYTHREE}/.style={column name={},column type=C},
    columns/{tr40QueryTWENTYTHREE}/.style={column name={},column type=C},
	columns/{tr50QueryTWENTYTHREE}/.style={column name={},column type=C},
	%%	
]{\fileQueryTWENTYTHREE}
}
}
%%%
%%%
\scalebox{1.}[0.9]{
\resizebox{\columnwidth}{!}{
\newcolumntype{C}{>{\centering\arraybackslash}p{21mm}}
\pgfplotstabletypeset[
font=\scriptsize,
	every head row/.style={
    output empty row,
		after row={%
            \midrule
              \multicolumn{1}{c|}{\textbf{\emph{Alphabet Size}} $\downarrow$}
            & \multicolumn{5}{c|}{\textbf{NotChainResponse[?X,?Y]$|$?$|$?}}
            \\\midrule
		},
	},
	%
	 columns={alphSizeQueryTWENTYFOUR,tr20QueryTWENTYFOUR,tr25QueryTWENTYFOUR,tr30QueryTWENTYFOUR,tr40QueryTWENTYFOUR,tr50QueryTWENTYFOUR},	 columns/{alphSizeQueryTWENTYFOUR}/.style={column name={},column type=C},	
    columns/{tr20QueryTWENTYFOUR}/.style={column name={},column type=C},
    columns/{tr25QueryTWENTYFOUR}/.style={column name={},column type=C},
	columns/{tr30QueryTWENTYFOUR}/.style={column name={},column type=C},
    columns/{tr40QueryTWENTYFOUR}/.style={column name={},column type=C},
	columns/{tr50QueryTWENTYFOUR}/.style={column name={},column type=C},
	%%	
]{\fileQueryTWENTYFOUR}
}
}
\caption{Time (in ms.) required for executing queries (with activation and correlation conditions) of different types on traces of different lengths and with alphabet of possible activities of different sizes.}
\label{table:exp_results_real}
\end{center}
\end{table}
\end{tiny} 


When using a Declare model including constraints without data conditions, in the worst case, namely to generate a log of $10\,000$ traces of length 50 from a model containing 20 constraints, the execution time is slightly above 2 minutes. When using constraints with activation conditions, the execution time in the worst case is of around 4.5 minutes. In the case of constraints with activation and correlation conditions the execution time in the worst case is of 5.5 minutes.

We also tested the trend of the execution times when varying the alphabet size (5, 10, 15, and 20 activities) of the Declare models (without data conditions). The number of constraints in these experiments is fixed to 20. As expected, for increasing alphabet sizes, the execution times increase. In the worst case, namely to generate a log of $10\,000$ traces of length 50 from a model containing 20 activities, the execution time is of around 6 minutes.

Finally, we executed a set of experiments to test the performance of query checking.\footnote{We did not assess the conformance checking tool explicitly since, as already mentioned, conformance checking is a special case of query checking.} We sampled the query times by varying the following parameters:
\begin{inparaenum}[\itshape(i)\upshape]
\item the type of query,
\item the number of events in the input trace (20, 25, 30, 40, and 50 events), and
\item the alphabet of possible activities (5, 10, 15, and 20 activities).
\end{inparaenum}
The queries cover all the standard Declare constraints and are considered without data conditions, with activation conditions, and with activation and correlation conditions. For all query types the execution times range from few milliseconds to 10 seconds. The only exception is the case of the queries of type \emph{alternate} that require, in the worst case, more than 1.5 minutes.

\section{Related Work and Conclusions}
In the literature, different extensions of Declare have been proposed. Some of these extensions have been used as a basis to develop Process Mining algorithms.
In \cite{Westergaard.Maggi/CoopIS2012:LookingintoFuture,DBLP:journals/ijcis/MaggiW14}, the authors introduce \emph{Timed Declare}, an extension of Declare whose semantics is based on a metric temporal logic that allows Declare constraints to be enriched with temporal conditions on timestamps. In \cite{DBLP:conf/bir/Maggi14}, such extension is used for the discovery of Timed Declare constraints.
%\cite{DBLP:conf/ruleml/BernardiCFM14,DBLP:journals/is/BernardiCFM16} presents an approach for the discovery of Declare rules characterizing the lifecycle of non-atomic activities in a log.

In the work proposed in \cite{Maggi.etal/BPM2013:DiscoveringDataAware}, another extension of Declare has been proposed whose semantics is defined by using a first-order variant of LTL that allows Declare constraints to be enriched with conditions on data. This semantics has been used in \cite{Maggi.etal/BPM2013:DiscoveringDataAware,DBLP:conf/bpm/BoseMA13} for the development of process discovery algorithms that produce data-aware Declare constraints from logs. In \cite{DBLP:journals/eswa/BurattinMS16}, MP-Declare is introduced and a technique for conformance checking based on MP-Declare is presented. Techniques for the discovery of MP-Declare constraints from event logs are presented in \cite{DBLP:conf/bpm/LenoDM18,Schoenig.etal/ICSOC2016:DiscoveryMultiperspectiveDeclare}.

%The presented technique is based on the translation of \textsc{Declare} templates into SQL queries on a relational database instance, where the event log has previously been stored. The query answer assigns the free variables with those tasks that lead to the satisfaction of the constraint in the event log. The methodology has later been extended towards multi-perspective \textsc{Declare} discovery~\cite{Schoenig.etal/ICSOC2016:DiscoveryMultiperspectiveDeclare}, to include data in the formulation of constraints.

In this paper, we presented a framework based on SAT to solve problems in the context of Declarative Process Mining.
We showed how three specific Declarative Process Mining problems (namely log generation, conformance checking and temporal query checking) can be expressed as suitable FO theories and solved by compilation into SAT using Alloy. The Alloy output, i.e., the SAT instance, can be provided as input to any SAT solver, thus making any performance optimization in SAT directly
available to Declarative Process Mining. While existing approaches for Declarative Process Mining are tailored for Declare, our approach can deal with any property expressed as a FOL constraint.

As future work, we would like to compare the performance of our tool using different SAT solvers. We are currently working on an approach based on SAT for monitoring the compliance of ongoing process executions wrt a set of MP-Declare constraints.  


%\section*{References}

\bibliography{references,mainbib,main-bib,libraryFiltered,diciccio,bpm,biblio,bib}

\end{document} 