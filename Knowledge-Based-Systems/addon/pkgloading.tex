% For language-specific hyphenations etc.
\usepackage[english]{babel}
% For nice links
\usepackage{url}
% For playing with colors in tabular environments
\usepackage{colortbl}
% For math symbols, such as \nexists
\usepackage{amssymb}
% For more math symbols, such as \mapsfrom
\usepackage{stmaryrd}
% For advanced graphics
%% For including figures, graphicx.sty has been loaded in
%% elsarticle.cls. If you prefer to use the old commands
%% please give \usepackage{epsfig}
% For equations, arrays of equations, defining operator names, etc.
\usepackage{amsmath}
% For cursive math
\usepackage{mathrsfs}
% For math symbols, such as \nexists
\usepackage{amssymb}
%% For math environments, such as "definition"
%\usepackage{amsthm}
%\theoremstyle{definition}
%\newdefinition{definition}{Definition}[section]
%\newtheorem{theorem}{Theorem}[section]
% For enumerating the line numbers
\usepackage[left]{lineno}
% For side notes, missing figures and inline to-do's
\usepackage[textsize=scriptsize,backgroundcolor=yellow!40]{todonotes}
% For specifying kewords and acronyms
\usepackage[nonumberlist,acronym,sanitize=none]{glossaries}
\glsdisablehyper
% For commenting out some parts of the text
\usepackage{comment}
% For building hyperlinks
\usepackage[pdftex, colorlinks=true, hyperfootnotes=true, hyperindex=true,
            plainpages=false, pagebackref=false, pdfpagelabels=true, pdfstartview=FitH,
            linkcolor=blue, citecolor=blue, urlcolor=blue,
            bookmarks, bookmarksopen, bookmarksdepth=3]{hyperref}
% For smart references
\usepackage[capitalise,nameinlink]{cleveref}
% To have "Figure 3(a)" in place of "Figure 3a" and  "Table 3(a)" in place of "Table 3a"
\captionsetup[subfigure]{subrefformat=simple,labelformat=simple}
  \renewcommand\thesubfigure{(\alph{subfigure})}
\captionsetup[subtable]{subrefformat=simple,labelformat=simple}
  \renewcommand\thesubtable{(\alph{subtable})}
\Crefname{algocf}{Algorithm}{Algorithms}
% TikZ/Pgf advanced graphics
\usepackage{tkz-base}
\usetikzlibrary{decorations.pathmorphing,trees,snakes,arrows,shapes,automata}
% To use inline and other fancy list-like environments (e.g., inparaenum)
\usepackage{paralist}
% To divide a text line into multiple columns
\usepackage{multicol}
% To create good-looking book-style tables
\usepackage{booktabs}
% To play around with list environments
\usepackage{enumitem}
% To create multirow cells in tables
\usepackage{multirow}
% To create rotated cells in tables
\usepackage{rotating}
% To create enumerated lists, whose numbering is reversed
\usepackage{etaremune}
% To make algorithmic nice-looking pseudocode
\usepackage[ruled,linesnumbered,algo2e]{algorithm2e}
%% For creating side-notes
% \usepackage{marginnote}
% For superimposing symbols over one another within math env.
\usepackage{mathtools}
% For strange math symbols like \Dashv
\usepackage{mathabx}
% For LaTeX if/then statements
\usepackage{ifthen}
% For strike-through cancellations
\usepackage[normalem]{ulem}
%% The lineno packages adds line numbers. Start line numbering with
%% \begin{linenumbers}, end it with \end{linenumbers}. Or switch it on
%% for the whole article with \linenumbers.
\usepackage{lineno}
% For highlighted text
\usepackage{soul}
% To add dummy text
\usepackage{lipsum}
% To have newlines in cells, with commands such as \makecell or \thead
\usepackage{makecell}
% For a decent formatting of numbers (and a wonderful system for numeric columns in tables, ``S'')
\usepackage{siunitx}
% To enable text protrusion
\usepackage{microtype}